

%%%%%%%%%%%%%%%%%%%%%%% Next Slide %%%%%%%%%%%%%%%%%%%%%%%

\begin{slide}
\section{Properties of Logarithms}

\begin{PauseHighLight}
  \begin{itemize}
  \item There are many logarithms defined as the inverse of an exponent
    \begin{align*}
      \log_a(a^x) &=x & a^{\log_a(x)}= x
    \end{align*}
    Note that $a>0$\pause
  \item We can write $b = a^{\log_a(b)}$ so
    \begin{align*}
      \log_a\bra{b^c} = \log_a\bra{(a^{\log_a(b)})^c}\pause =
      \log_a\bra{a^{c\,\log_a(b)}}\pauseb = c\,\log_a(b)\pauseb
    \end{align*}
  \item Note that $\log_a\bra{\frac{1}{x}}= \log_a\bra{x^{-1}}\pause = -\log_a(x)$\pauseb
  \item Because $a^0 =1$ then $\log_a(1)= \log_a(a^0) = 0\,\log_a(a)=
    0$\pause
  \end{itemize}
\end{PauseHighLight}

\end{slide}


%%%%%%%%%%%%%%%%%%%%%%% Next Slide %%%%%%%%%%%%%%%%%%%%%%%

\begin{slide}
  \section[-1]{More Properties of Logarithms}

\begin{PauseHighLight}
  \begin{itemize}
  \item Since $a^b \, a^c= a^{b+c}$
    \begin{align*}
      \logg{a^b \, a^c} &= \logg{a^{b+c}}\pause
                          = (b+c)\,\log(a)\pauseb \\
      &= b\,\log(a) +
      c\,\log(a)\pauseb = \log(a^b) + \log(a^c)\pauseb
    \end{align*}
  \item Let $x=a^b>0$ and $y=a^c>0$ then $\log(x\,y) =
    \log(x)+\log(y)$\pause
  \item Because $\log(b^{-1})=-\log(b)$ then $\logg{\frac{a}{b}} =
    \log(a)-\log(b)$\pause
  \item Note that $\log_a(x) = \log_a(b^{\log_b(x)}) = \log_b(x) \, \log_a(b)$\pause
  \item Most properties of logarithm apply to all logarithms, but only
    for the natural logarithm does $\dd \ln(x)/\dd x = 1/x$\pause
 \item Throughout the lecture course I have used $\log(x)$ to denote
    $\ln(x)$\pause
  \end{itemize}
\end{PauseHighLight}

\end{slide}
