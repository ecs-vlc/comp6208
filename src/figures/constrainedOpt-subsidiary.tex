% Created 2021-01-28 Thu 17:24
% Intended LaTeX compiler: pdflatex
\documentclass[11pt]{article}
\usepackage[utf8]{inputenc}
\usepackage[T1]{fontenc}
\usepackage{graphicx}
\usepackage{grffile}
\usepackage{longtable}
\usepackage{wrapfig}
\usepackage{rotating}
\usepackage[normalem]{ulem}
\usepackage{amsmath}
\usepackage{textcomp}
\usepackage{amssymb}
\usepackage{capt-of}
\usepackage{hyperref}
\usepackage{minted}
\usepackage[a4paper,margin=20mm]{geometry}
\usepackage{amsmath}
\usepackage{amsfonts}
\usepackage{stmaryrd}
\usepackage{bm}
\usepackage{minted}
\usemintedstyle{emacs}
\usepackage[T1]{fontenc}
\usepackage[scaled]{beraserif}
\usepackage[scaled]{berasans}
\usepackage[scaled]{beramono}
\newcommand{\tr}{\textsf{T}}
\newcommand{\grad}{\bm{\nabla}}
\newcommand{\av}[2][]{\mathbb{E}_{#1\!}\left[ #2 \right]}
\newcommand{\Prob}[2][]{\mathbb{P}_{#1\!}\left[ #2 \right]}
\newcommand{\logg}[1]{\log\!\left( #1 \right)}
\newcommand{\pred}[1]{\left\llbracket { \small #1} \right\rrbracket}
\newcommand{\e}[1]{{\rm e}^{#1}}
\newcommand{\dd}{\mathrm{d}}
\DeclareMathAlphabet{\mat}{OT1}{cmss}{bx}{n}
\newcommand{\normal}[2]{\mathcal{N}\!\left(#1 \big| #2 \right)}
\newcounter{eqCounter}
\setcounter{eqCounter}{0}
\newcommand{\explanation}{\setcounter{eqCounter}{0}\renewcommand{\labelenumi}{(\arabic{enumi})}}
\newcommand{\eq}[1][=]{\stepcounter{eqCounter}\stackrel{\text{\tiny(\arabic{eqCounter})}}{#1}}
\newcommand{\argmax}{\mathop{\mathrm{argmax}}}
\newcommand{\Dist}[2][Binom]{\mathrm{#1}\left( \strut {#2} \right)}
\author{Adam Prügel-Bennett}
\date{\today}
\title{Advanced Machine Learning Subsidary Notes\\\medskip
\large Lecture 11: Constrained Optimisation}
\hypersetup{
 pdfauthor={Adam Prügel-Bennett},
 pdftitle={Advanced Machine Learning Subsidary Notes},
 pdfkeywords={},
 pdfsubject={},
 pdfcreator={Emacs 26.3 (Org mode 9.1.9)}, 
 pdflang={English}}
\begin{document}

\maketitle

\section{Keywords}
\label{sec:orgfede038}
\begin{itemize}
\item Lagrangians, Inequalities, KKT, Linear Programming,
\end{itemize}

\section{Main Points}
\label{sec:orgc9f8acd}

\subsection{Equality Constraints}
\label{sec:org51f0e69}
\begin{itemize}
\item If we want to minimise \(f(\bm{x})\) subject to the constraint
\(g(\bm{x})=0\) this is equivalent to solving the problem
$$ \min_{\bm{x}} \max_{\alpha} \mathcal{L}(\bm{x},\alpha) $$
where \(\mathcal{L}(\bm{x},\alpha)\) is a \emph{Lagrangian} given by
$$ \mathcal{L}(\bm{x},\alpha) = f(\bm{x}) - \alpha\,g(\bm{x}) $$
\item \(\alpha\) is a Lagrange multiplier that is determined by the joint
optimisation problem
\item Note that we seek a saddle-point
\begin{itemize}
\item We minimise with respect to \(\bm{x}\) and maximise with respect
to \(\alpha\)
\item We can't escape this
\begin{itemize}
\item if we multiply \(\alpha\) by \(-1\) we just changing the directions
of the \(\alpha\) axis but still have a saddle-point
\item if we multiply both terms by \(-1\) then we would end up
minimising with respect to \(\alpha\), but maximising with
respect to \(\bm{x}\)
\end{itemize}
\end{itemize}
\item The solution to our problem must satisfy
\begin{align*}
\grad  \mathcal{L}(\bm{x},\alpha) &=
\grad f(\bm{x}) - \alpha\,\grad g(\bm{x}) = 0, &
\frac{\partial \mathcal{L}(\bm{x},\alpha)}{\partial \alpha} &=
g(\bm{x}) = 0
\end{align*}
\begin{itemize}
\item The second equation ensures that we sit on the constraint
\item The first equation says that the gradient of \(f(\bm{x})\) must be
perpendicular to the constraint
\item This is necessary for the solution to be a (local) minimum
(i.e. we can not get an improvement by moving along the constraint)
\item There can be multiple solutions: these equations at a satisfied
for any local minima on the constraint
\end{itemize}
\item If we have multiple equality constraints we just use multiple
Lagrange multipliers
\end{itemize}

\subsection{Inequality constraints}
\label{sec:org609c360}
\begin{itemize}
\item If we are minimising \(f(\bm{x})\) subject to an inequality
constraint \(g(\bm{x})\geq0\) then one of two things can happen
\begin{enumerate}
\item Either we have a (local) minimum of \(f(\bm{x})\) that satisfies
the constraint or
\item We have a local minimum on the constraint
\end{enumerate}
\item We can therefore solve this problem by again using a Lagrangian
$$ \mathcal{L}(\bm{x},\alpha) = f(\bm{x}) - \alpha\,g(\bm{x}) $$
with the additional constraint \(\alpha\geq0\)
\begin{itemize}
\item If the minimum lies in a region that satisfies the constraint
then we just set \(\alpha=0\) and minimise \(f(\bm{x})\)
\item If the solution lies on the constraint we again have  \(\grad
       f(\bm{x}) = \alpha\,\grad g(\bm{x})\), but now \(\alpha>0\) which
means that not only is there no improving direction along the
constraint, but also the negative-gradient of \(f(\bm{x})\)
points in the direction where \(g(\bm{x})\) becomes smaller,
i.e. in the region that violates the constraint
\begin{itemize}
\item note if \(\alpha<0\) we could find a better solution moving
away from the constraint into the feasible region
\end{itemize}
\item \emph{Karush-Kuhn-Tucker (KKT) Conditions}
\begin{itemize}
\item For inequality constraints we require either
\begin{enumerate}
\item \(\alpha=0\) and there is an unconstrained minimum in the
regions \(g(\bm{x})\geq0\) or
\item \(\alpha>0\) and the solution lies on \(g(\bm{x})=0\)
\end{enumerate}
\end{itemize}
\end{itemize}
\item If we have multiple inequality constraints we just introduce a
Lagrange multiply for each constraint with \(\alpha\geq0\)
\end{itemize}

\subsection{Duality}
\label{sec:orgf3478b2}
\begin{itemize}
\item Our problem of solving \(f(\bm{x})\) subject to some constraints is
known as the \emph{primal problem}
\item If our problem is sufficiently simple we can sometimes find a
solution \(\bm{x}^*(\bm{\alpha})\) that satisfies
$$ \grad  \mathcal{L}(\bm{x}^*(\bm{\alpha}),\bm{\alpha}) = 0 $$
\item This leaves us with the \emph{dual problem}
$$ \max_{\bm{\alpha}} \;
     \mathcal{L}(\bm{x}^*(\bm{\alpha}),\bm{\alpha}) $$
possible with constraints on \(\bm{\alpha}\) (e.g. \(\alpha_i\geq0\))
that arise from KKT conditions
\item \textbf{Linear Programming}
\begin{itemize}
\item In linear programming we are trying to find a value of \(\bm{x}\)
that minimises a linear objective function \(\bm{c}^\tr\bm{x}\)
subject to linear constraints \(\mat{M}\,\bm{x} = \bm{b}\)
(and/or \(\mat{M}\bm{x}\geq \bm{b}\))
\item We can write this as a Lagrange problem
$$ \mathcal{L} = \bm{c}^\tr \bm{x}  - \bm{\alpha}^\tr \left(
       \mat{M}\bm{x} - \bm{b}\right) $$
(subject to constraints \(\bm{\alpha}\geq0\) if we have
inequality constraints in the primal problem)
\item We can rearrange the Lagrangian as
$$ \mathcal{L} = \bm{\alpha}^\tr  \bm{b} + \left( \bm{c}^\tr - \bm{\alpha}^\tr \mat{M} \right) \bm{x} $$
\item Using the identity \(\bm{a}^\tr\bm{b} = \bm{b}^\tr \bm{a}\) w can rewrite this
as
$$ \mathcal{L} =  \bm{b}^\tr \bm{\alpha} - \bm{x}^\tr \left(\mat{M}^\tr \bm{\alpha} - \bm{c} \right) $$
\item But we can now treat \(\bm{x}\) as a Lagrange multiplier so we
get the \emph{dual problem}
$$ \max_{\bm{\alpha}} \; \bm{b}^\tr \bm{\alpha} $$
subject to the constraint
$$ \mat{M}^\tr \bm{\alpha} =  \bm{c} $$
\item If the original constraints were inequality constraints the \(\bm{\alpha}\geq0\)
\item The dimensionality of the dual problem can sometimes be much
lower than that of the primal problem making it easier to solve
\end{itemize}
\item \textbf{Quadratic Program}
\begin{itemize}
\item In a quadratic program we have to minimise a quadratic loss
\(\bm{x}^\tr \mat{Q}\,\bm{x}\) subject to linear constraints
\(\mat{M}\,\bm{x} =\mat{b}\) (or \(\mat{M}\,\bm{x} \geq \mat{b}\))
\item For there to be a unique minimum \(\mat{Q}\) must be positive
definite (which is sometimes written \(\mat{Q}\succ0\))
\item We can write a Lagrangian
$$ \mathcal{L}(\bm{x},\bm{\alpha}) = \bm{x}^\tr \mat{Q}\, \bm{x} -
      \bm{\alpha}^\tr \left( \mat{M}\,\bm{x} - \bm{b}\right) $$
\item The solution is given by  \(\max\limits_{\bm{\alpha}}
        \min\limits_{\bm{x}}  \mathcal{L}(\bm{x},\bm{\alpha})\)
\item If the constraints are inequality constraints then \(\alpha_i\geq0\)
\item The minimum with respect to \(\bm{x}\) is given by 
$$ \grad_{\bm{x}} \mathcal{L}(\bm{x},\bm{\alpha}) =  2\, \mat{Q}\,
        \bm{x} + \mat{M}^\tr \bm{\alpha} = 0 $$
\item So that \(\bm{x}^* = \frac{1}{2} \mat{Q}^{-1} \mat{M}^\tr\)
\item Substituting this back into the Lagrangian we get the dual problem
 $$ \max_{\bm{\alpha}}
      -\frac{1}{4} \bm{\alpha}^\tr \mat{M} \mat{Q}^{-1} \mat{M}^\tr
      \bm{\alpha} + \bm{\alpha}^\tr \bm{b} $$
with \(\alpha_i\geq0\) if we started with inequality constraints
\begin{itemize}
\item note in the derivation that we end up with two terms
proportional to \(\bm{\alpha}^\tr \mat{M} \mat{Q}^{-1} \mat{M}^\tr
          \bm{\alpha}\) one partially cancelling the other
\end{itemize}
\end{itemize}
\end{itemize}



\section{Exercises}
\label{sec:orga75f437}

\subsection{Quadratic with a linear constraint}
\label{sec:orgdc9fdf5}
\begin{itemize}
\item Consider minimising \(f(\bm{x}) = \tfrac{1}{2} \bm{x}^\tr \mat{Q} \bm{x}\)
subject to the constraint \(\bm{a}^\tr\bm{x} = b\)
\begin{enumerate}
\item Write a Lagrangian for this problem
\item Find the minimum of the Lagrangian with respect to \(\bm{x}\)
\item Write down and solve the dual problem
\item Hence write down a solution to the primal problem
\end{enumerate}
\item See answers, but also experiments
\end{itemize}

\subsection{Saddle Point}
\label{sec:org0e8fdc3}
\begin{itemize}
\item Strangely (for me at least) the optimum of a constrained
optimisation problem is given by the saddle-point of the Lagrangian
\item Consider the problems of minimising \(x^2/2\) subject to the
constraint \(x=1\)
\begin{enumerate}
\item Write down the Lagrangian
\item Calculate the Hessian matrix (matrix of second derivatives)
\item Compute the eigenvalues of the Hessian (show that they have
different signs everywhere so there are  no maxima or minima)
\end{enumerate}
\item See answers
\end{itemize}

\section{Experiments}
\label{sec:org21d1566}

\subsection{Quadratic with a linear constraint}
\label{sec:org1b69bfd}
\begin{itemize}
\item Let \(X\) be a \(10\times5\) random matrix with elements drawn from
\(\mathcal{N}(0,1)\)
\item Let \(\mat{Q} = \mat{X}^\tr \mat{X}\)
\begin{itemize}
\item Check that this is positive definite
\end{itemize}
\item Let \(f(\bm{x}) = \tfrac{1}{2} \bm{x}^\tr \mat{Q} \bm{x}\)
\item Let \(\bm{a}\) be a random vector with 5 elements drawn from \(\mathcal{N}(0,1)\)
\item We want to minimise \(f(\bm{x})\) subject to the constraint \(\bm{a}^\tr\bm{x}=1\)
\item Work out the Lagrangian, \(L(\bm{x},\alpha)\) for this system
\item Write an iterative gradient solver that
\begin{enumerate}
\item Makes steps \(\bm{x}\leftarrow \bm{x} - r \grad L(\bm{x},\alpha)\)
\item Makes steps \(\alpha \leftarrow \alpha + r \frac{\partial
        L(\bm{x},\alpha)}\partial \alpha}\)
\end{enumerate}
\item Note you will have to tune the learning step \(r\)
\item Compare the solution you find by running your algorithm until
convergence with the exact result (see exercise and/or answer)
\end{itemize}

\begin{minted}[]{matlab}
function [x,alpha] = optimise(Q,a,r)
  x = rand(5,1);                                 % initialise x
  alpha = 0;                                       % initialise alpha
  for t=1:1000
    x = x - r*(Q*x-alpha*a);                  % x = x - r  \grad L
    alpha = alpha + r*(-(a'*x-1));          % alpha = alpha + r dL/d\alpha
  endfor
endfunction

[x,alpha] = optimise(Q,a,0.1)

alphaTheory = 1/(a'*Q*a)
xTheory = inv(Q)*a*alphaTheory

\end{minted}




\section{Answers}
\label{sec:orgd89063c}

\subsection{Quadratic with a linear constraint}
\label{sec:orgbf5bcb5}
\begin{enumerate}
\item The Lagrangian is given by
$$ \mat{L}(\bm{x},\alpha) = \frac{1}{2}\, \bm{x}^\tr \mat{Q}
      \bm{x} - \alpha \,(\bm{a}^\tr\bm{x} - b) $$
\item Minimising with respect to \(\bm{x}\) we get
$$ \grad  \mat{L}(\bm{x},\alpha) =  \mat{Q}\,  \bm{x} + \alpha\,
      \bm{a} = 0 $$
or \(\bm{x} = \alpha\, \mat{Q}^{-1} \bm{a}\)
\item Thus the dual problem is
$$ \max_\alpha - \frac{1}{2} \alpha^2 \, \bm{a}^\tr\, \mat{Q}^{-1}\bm{a} + \alpha\,b $$
\begin{itemize}
\item The solution to the dual problem is
$$ \alpha = \frac{b}{\bm{a}^\tr\, \mat{Q}^{-1}\bm{a}} $$
\end{itemize}
\item Thus the solution to the primal problem is
$$ \bm{x} = \frac{b\, \mat{Q}^{-1} \bm{a}}{\bm{a}^\tr\,
      \mat{Q}^{-1}\bm{a}} $$
\begin{itemize}
\item Note that in most quadratic programming problems we are dealing
with many inequality constraints so solving the dual problem in
closed form isn't necessarily easy
\end{itemize}
\end{enumerate}

\subsection{Saddle Point}
\label{sec:orged22394}
\begin{itemize}
\item Just do it
\begin{enumerate}
\item The Lagrangian is given bye
$$ \mathcal{L} = \frac{x^2}{2} - \alpha\, (x-1) $$
\item The Hessian is given by
$$ \mat{H} = \begin{pmatrix} 1 & -1 \\ -1 &0 \end{pmatrix} $$
\item The traces \(T = 1\) and the determinant \(D = -1\) So that
$$ \lambda = \frac{T \pm \sqrt{T^2 - 4\, D}}{2} = \frac{1
	\pm{5}}{2} = \{1.618, -0.618\} $$
If you prefer you can compute the eigenvalues numerically
\end{enumerate}
\item Note that whatever we do the determinant will be negative leading
to a negative eigenvalue (the determinant is equal to the product
of eigenvalues).   This would be true if we were maximising
\(-x^2/2\).  You can change the constraints or the objective
function, but you will still get eigenvalues of different signs.
\end{itemize}
\end{document}
