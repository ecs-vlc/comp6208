%Master File:lectures.tex

\lesson{Kernel Trick}
\vspace{-1cm}
\begin{center}
  \includegraphics[height=12cm]{svm-data1}\
\end{center}

\keywords{The Kernel Trick, SVMs, Regression}


%%%%%%%%%%%%%%%%%%%%%%% Next Slide %%%%%%%%%%%%%%%%%%%%%%%
\renewcommand{\Outline}{%
\begin{slide}
\section[-1]{Outline}

\begin{minipage}{8cm}\raggedright
  \begin{enumerate}
    \outlineitem{The Kernel Trick}{kernels}
    \outlineitem{Positive Semi-Definite Kernels}{psdk}
    \outlineitem{Kernel Properties}{eigen}
    \outlineitem{Beyond Classification}{beyond}
  \end{enumerate}
\end{minipage}\hfill
\begin{minipage}{15cm}
  \includegraphics[width=15cm]{svm-data1}
\end{minipage}
\end{slide}
\addtocounter{outlineitem}{1}
}

\setcounter{outlineitem}{1}

%%%%%%%%%%%%%%%%%%%%%%% Next Slide %%%%%%%%%%%%%%%%%%%%%%%
\Outline %  Kernels
%%%%%%%%%%%%%%%%%%%%%%% Next Slide %%%%%%%%%%%%%%%%%%%%%%%



\begin{slide}
\section[-2]{SVM Kernels}

\begin{PauseHighLight}
  \begin{itemize}
  \item SVM Kernels are functions of two variables that can be factorised
    \begin{align*}
      K(\bm{x},\bm{y}) = \bm{\phi}(\bm{x})^\tr\bm{\phi}(\bm{y}) =
      \sum_{i} \phi^{(i)}(\bm{x})\,\phi^{(i)}(\bm{y})
    \end{align*}
  \item where $\bm{\phi}(\bm{x}) = (\phi^{(1)}(\bm{x}),\, \phi^{(2)}(\bm{x}),\,
    \ldots)^\tr$ and $\phi^{(i)}(\bm{x})$ are real valued functions of
    $\bm{x}$\pause
  \item $K(\bm{x},\bm{y})$ will be positive semi-definite (because it
    is an inner-product)\pause
  \item Furthermore, any positive semi-definite function will
    factorise\pause
  \item This factorisation is not always obvious (we return to this
    later)\pause
  \end{itemize}
\end{PauseHighLight}

\end{slide}


%%%%%%%%%%%%%%%%%%%%%%% Next Slide %%%%%%%%%%%%%%%%%%%%%%%

\begin{slide}
\section[-2]{Dual Form}

\begin{PauseHighLight}
  \begin{itemize}
  \item Recall that the dual problem for an SVM is
    \begin{align*}
      \max_{\bm{\alpha}} \sum_{k=1}^m \alpha_k - \frac{1}{2}
      \sum_{k,l=1}^m \alpha_k\,\alpha_l\, y_k\,y_l\,
      \bm{\phi}(\bm{x}_k)^\tr\,\bm{\phi}(\bm{x}_l)
    \end{align*}
  \item subject to $\sum\limits_{k=1}^m y_k\,\alpha_k=0$ and $0\leq \alpha_k (\leq C)$\pause
  \item But since $K(\bm{x}_k,\bm{x}_l) =
    \bm{\phi}(\bm{x}_k)^\tr\,\bm{\phi}(\bm{x}_l)$ the dual problem
    becomes
    \begin{align*}
      \max_{\bm{\alpha}} \sum_{k=1}^m \alpha_k - \frac{1}{2}
      \sum_{k,l=1}^m \alpha_k\,\alpha_l\, y_k\,y_l\,K(\bm{x}_k,\bm{x}_l)\pause
    \end{align*}
  \item This is the \emph{kernel trick}\pause---we never have to
    compute $\bm{\phi}(\bm{x})$!\pauseb
  \end{itemize}
\end{PauseHighLight}


\end{slide}


%%%%%%%%%%%%%%%%%%%%%%% Next Slide %%%%%%%%%%%%%%%%%%%%%%%

\begin{slide}
\section[-2]{Classifying New Data}

\begin{PauseHighLight}
  \begin{itemize}
  \item Having trained the SVM we now have to use it\pause
  \item Given a new input $\bm{x}$ we decide on the class
    \begin{align*}
      &y=\mathrm{\mathop{sgn}}\!\left(\bm{w}^\tr\bm{\phi}(\bm{x})-b\right)\pause
        & \text{but}&& \bm{w} = \sum_{k=1}^m
    \alpha_k\,y_k \, \bm{\phi}(\bm{x}_k)\pause
    \end{align*}
  \item In the dual representation this becomes
    \begin{align*}
      \mathrm{\mathop{sgn}}\!\left(\sum_{k=1}^m \alpha_k\, y_k \,
      K(\bm{x}_k,\bm{x})-b\right)
    \end{align*}
    where we only need to sum over the non-zero $\alpha_k$ (i.e. the
    support vectors SVs)\pause
  \end{itemize}
\end{PauseHighLight}

\end{slide}

%%%%%%%%%%%%%%%%%%%%%%% Next Slide %%%%%%%%%%%%%%%%%%%%%%%

%%%%%%%%%%%%%%%%%%%%%%% Next Slide %%%%%%%%%%%%%%%%%%%%%%%
\Outline %  Kernels
%%%%%%%%%%%%%%%%%%%%%%% Next Slide %%%%%%%%%%%%%%%%%%%%%%%

\begin{slide}
\section[-1]{Recap on Eigen Systems}

\begin{PauseHighLight}
  \begin{itemize}
  \item Recall for a symmetric ($n\times n$) matrix $\mat{M}$ an
    eigenvector, $\bm{v}$
    \begin{align*}
      \mat{M} \, \bm{v} = \lambda \bm{v}\pause
    \end{align*}
  \item There are $n$ independent eigenvectors $\bm{v}^{(i)}$ with
    real eigenvalues $\lambda^{(i)}$\pause
  \item The eigenvectors are orthogonal so that
    $\bm{v}^{(i)\tr}\bm{v}^{(j)}=0$  if $i\neq j$\pause
  \item Forming a matrix of eigenvectors
    $\mat{V}=(\bm{v}^{(1)},\,\bm{v}^{(2)},\, \ldots\,\bm{v}^{(n)})$ the matrix
    satisfies
    \begin{align*}
      \mat{V}^\tr \mat{V} = \mat{V}\,\mat{V}^\tr = \mat{I}\pause
    \end{align*}
  \item Such matrices are said to be orthogonal\pause
      \end{itemize}
\end{PauseHighLight}

\end{slide}

%%%%%%%%%%%%%%%%%%%%%%% Next Slide %%%%%%%%%%%%%%%%%%%%%%%

\begin{slide}
\section[-2]{Eigen Decomposition}

\begin{PauseHighLight}\squeeze
  \begin{itemize}
  \item From the eigenvalue equation $\mat{M} \, \bm{v}^{(k)} = \lambda^{(k)}
    \bm{v}^{(k)}$
     \vspace*{-0.5em}
    \begin{align*}
      \mat{M} \mat{V} &= \mat{V} \, \mat{\Lambda} & \mbox{where}\quad
      \mat{\Lambda} &= {\small
      \begin{pmatrix}
        \lambda_1 & 0 & \cdots & 0 \\
        0 & \lambda_2 & \cdots & 0 \\
        \vdots & \vdots & \ddots & \vdots \\
        0 & 0 & \cdots & \lambda_n
      \end{pmatrix}}\pause
    \end{align*}                         \vspace*{-2em}
  \item Multiplying on the right by $\mat{V}^\tr$ we get
    \begin{align*}
      \mat{M} &= \mat{V} \, \mat{\Lambda} \mat{V}^\tr = \sum_{k=1}^n
      \lambda^{(k)}\, \bm{v}^{(k)}\, \bm{v}^{(k)\tr}\pause 
    \end{align*}
    Or \vspace*{-1em}
    \begin{align*}
      M_{ij} = \sum_{k=1}^n \lambda^{(k)}\, v^{(k)}_i\, v^{(k)}_j
      \pause = \sum_{k=1}^n u^{(k)}_i\, u^{(k)}_j = \langle \bm{u}_i, \bm{u}_j \rangle
    \end{align*}
    $u^{(k)}_i = \sqrt{\lambda^{(k)}}\,v^{(k)}_i$\pauseb
  \end{itemize}
\end{PauseHighLight}

\end{slide}

%%%%%%%%%%%%%%%%%%%%%%% Next Slide %%%%%%%%%%%%%%%%%%%%%%%

\begin{slide}
\section{Eigenfunctions}

\begin{PauseHighLight}
  \begin{itemize}
  \item By analogy for a symmetric function of two variables we can
    define an \textit{eigenfunction}
    \begin{align*}
      \int K(\bm{x}, \bm{y})\, \psi(\bm{y}) \, \dd \, \bm{y} = \lambda
      \, \psi(\bm{x})\pause
    \end{align*}
  \item In general there will be a denumerable set of eigenfunctions
    $\psi^{(k)}(\bm{x})$ where
    \begin{align*}
      K(\bm{x}, \bm{y}) = \sum_k \lambda^{(k)}\,
      \psi^{(k)}(\bm{x})\,\psi^{(k)}(\bm{y})\pause
    \end{align*}
  \item This is known as Mercer's theorem\pause
  \end{itemize}
\end{PauseHighLight}

\end{slide}

%%%%%%%%%%%%%%%%%%%%%%% Next Slide %%%%%%%%%%%%%%%%%%%%%%%

\begin{slide}
\section[-2]{Limit Process}

\pb\pause
\begin{itemize}
\item Consider sampling a function at a set of
  points\pauseh\pauselevel{=1}
  \begin{center}
    \multipdf[width=0.8\linewidth]{sampleFunction}\pause
  \end{center}
\item In the limit where the number of sample points goes to infinity
  the vector more closely approximates a function\pauseh
\item Instead of the indices being numbers we use $k\leftarrow x_k$\pauseh
\end{itemize}

\end{slide}

%%%%%%%%%%%%%%%%%%%%%%% Next Slide %%%%%%%%%%%%%%%%%%%%%%%

\begin{slide}
\section[-2]{Linear Operators}

\begin{PauseHighLight}
  \begin{itemize}\squeeze
  \item Recall a linear function $\mathcal{T}[f(x)]$ can be
    represented by a kernel
    \begin{align*}
      \mathcal{T}[f(x)] = \int_{y\in\mathcal{I}} K(x,y)\,f(y)\,\dd
      y\pause \approx \Delta \sum_{j=1}^n K(x, y_j)\, f(y_j)\pauseb
    \end{align*}
    \begin{center}
      \multipdf[width=0.7\linewidth]{approxKernel}\pauseb
    \end{center}
  This is just a matrix equation with $M_{ij}=\Delta\,K(x_i, y_j)$\pause
  \end{itemize}
\end{PauseHighLight}

\end{slide}


%%%%%%%%%%%%%%%%%%%%%%% Next Slide %%%%%%%%%%%%%%%%%%%%%%%

\begin{slide}
\section[-2]{SVM Kernels}

\begin{PauseHighLight}
  \begin{itemize}
  \item If we define $\phi^{(k)}(\bm{x}) = \sqrt{\lambda^{(k)}} \psi^{(k)}(\bm{x})$
    then
    \begin{align*}
      K(\bm{x}, \bm{y}) &= \sum_k \lambda^{(k)}\,
      \psi^{(k)}(\bm{x})\,\psi^{(k)}(\bm{y})
      = \sum_k \phi^{(k)}(\bm{x})\,\phi^{(k)}(\bm{y})\pause
    \end{align*}
  \item This is the definition of a SVM kernel we started with\pause
  \item Note that for $\phi^{(k)}(\bm{x})$ to be real $\lambda^{(k)}\geq 0$ for
    all $k$\pause
  \item If $\lambda^{(k)} < 0$ the ``distance'' between points in the
    extended feature space can be negative!\pause
  \item If we use a kernel that isn't positive semi-definite then the
    Hessian of the dual objective function will not be negative
    semi-definite and there will be a maximum where $\bm{\alpha}$
    diverges\pause
  \end{itemize}
\end{PauseHighLight}

\end{slide}



%%%%%%%%%%%%%%%%%%%%%%% Next Slide %%%%%%%%%%%%%%%%%%%%%%%
\Outline %  Kernels
%%%%%%%%%%%%%%%%%%%%%%% Next Slide %%%%%%%%%%%%%%%%%%%%%%%

\begin{slide}
\section{Positive Semi-Definite Kernels}

\begin{PauseHighLight}
  \begin{itemize}
  \item Kernels (or matrices) that have eigenvalues $\lambda^{(k)}\geq0$ are
    called positive semi-definite\pause
  \item (If the eigenvalues are strictly positive $\lambda^{(k)}>0$ the
    kernels or matrices are called positive definite)\pause
  \item Positive semi-definite kernels can always be decomposed into a
    sum of real functions
    \begin{align*}
      K(\bm{x}, \bm{y}) &= \sum_k \phi^{(k)}(\bm{x})\,\phi^{(k)}(\bm{y})\pause
    \end{align*}
  \end{itemize}
\end{PauseHighLight}


\end{slide}

%%%%%%%%%%%%%%%%%%%%%%% Next Slide %%%%%%%%%%%%%%%%%%%%%%%

\begin{slide}
\section[-1]{Properties of Positive Semi-Definiteness}

\begin{PauseHighLight}
  \begin{itemize}
  \item Since 
    \begin{align*}
      K(\bm{x}, \bm{y}) = \sum_k \phi^{(k)}(\bm{x})\,\phi^{(k)}(\bm{y})\pause
    \end{align*}
  \item An immediate consequence is that for any function $f(\bm{x})$
   \begin{align*}
      \hspace*{-2em}\int f(\bm{x})\, K(\bm{x}, \bm{y})\, f(\bm{y})\,\dd\bm{x}\, \dd
       \bm{y} &= \int f(\bm{x})\, \sum_k
      \phi^{(k)}(\bm{x})\,\phi^{(k)}(\bm{y})\, f(\bm{y})\,\dd\bm{x}\, \dd
      \bm{y}\pause  \\
      &=\sum_k \left( \int f(\bm{x})\, \phi^{(k)}(\bm{x})\,\dd\bm{x}
      \right)^2 \geq 0\pause
    \end{align*}
  \end{itemize}
\end{PauseHighLight}


\end{slide}

%%%%%%%%%%%%%%%%%%%%%%% Next Slide %%%%%%%%%%%%%%%%%%%%%%%

\begin{slide}
\section{Positive Semi-Definiteness}

\begin{PauseHighLight}
  \begin{itemize}
  \item The following statements are equivalent
    \begin{itemize}
    \item $K(\bm{x}, \bm{y})$ is positive semi-definite (written
      $K(\bm{x}, \bm{y})\succeq 0$)\pause
    \item The eigenvalues of $K(\bm{x}, \bm{y})$ are non-negative\pause
    \item The kernel can be written
      \begin{align*}
        K(\bm{x}, \bm{y}) = \sum_k \phi^{(k)}(\bm{x})\,\phi^{(k)}(\bm{y})
      \end{align*}
      where the $\phi^{(k)}(\bm{x})$'s are real functions\pause
    \item For any real function $f(x)$
      \begin{align*}
        \int f(\bm{x})\, K(\bm{x}, \bm{y}) \, f(\bm{y})
         \, \dd \, \bm{x}  \, \dd \, \bm{y} \geq 0\pause
      \end{align*}
    \end{itemize}
  \end{itemize}
\end{PauseHighLight}

\end{slide}


%%%%%%%%%%%%%%%%%%%%%%% Next Slide %%%%%%%%%%%%%%%%%%%%%%%

\begin{slide}
\section[-1]{Adding Kernels}

\begin{PauseHighLight}
  \begin{itemize}
  \item We can construct SVM kernels from other kernels\pause
  \item If $K_1(\bm{x}, \bm{y})$ and  $K_2(\bm{x}, \bm{y})$ are valid
    kernels then so is $K_3(\bm{x}, \bm{y}) = K_1(\bm{x}, \bm{y}) +
    K_2(\bm{x}, \bm{y})$
    \begin{align*}
      \hspace*{-1em}Q &= \int f(\bm{x})\, K_3(\bm{x}, \bm{y}) \, f(\bm{y})
      \, \dd \bm{x}  \, \dd \bm{y} \\
      &=\int f(\bm{x})\, \left( \strut K_1(\bm{x}, \bm{y}) + K_2(\bm{x},
      \bm{y}) \right)  f(\bm{y}) \, \dd \bm{x}  \, \dd \bm{y} \\
      &=\int f(\bm{x})\, K_1(\bm{x}, \bm{y}) \, f(\bm{y})
      \, \dd \bm{x}  \, \dd \bm{y} +
      \int f(\bm{x})\, K_2(\bm{x}, \bm{y}) \, f(\bm{y})
      \, \dd \bm{x}  \, \dd \bm{y}
      \geq 0   \pause
    \end{align*}
  \item If $K(\bm{x}, \bm{y})$ is a valid kernel so is $c\,K(\bm{x},
    \bm{y})$ for $c>0$\pause
  \end{itemize}
\end{PauseHighLight}

\end{slide}

%%%%%%%%%%%%%%%%%%%%%%% Next Slide %%%%%%%%%%%%%%%%%%%%%%%

\begin{slide}
\section{Product of Kernels}

\begin{PauseHighLight}
  \begin{itemize}
  \item If $K_1(\bm{x}, \bm{y})$ and  $K_2(\bm{x}, \bm{y})$ are valid
    kernels then so is $K_3(\bm{x}, \bm{y}) = K_1(\bm{x}, \bm{y})\,
    K_2(\bm{x}, \bm{y})$\pause
  \item Writing $K_1(\bm{x}, \bm{y}) = \sum_i \phi^{(i)}_1(\bm{x})\,
    \phi^{(i)}_1(\bm{y})$ and $K_2(\bm{x}, \bm{y}) = \sum_j \phi^{(j)}_2(\bm{x})\,
    \phi^{(j)}_2(\bm{y})$ then
    \begin{align*}
      K_3(\bm{x}, \bm{y})
      = \sum_{i,j} \phi^{(i)}_1(\bm{x})\,\phi^{(i)}_1(\bm{y})\,
      \phi^{(j)}_2(\bm{x})\,\phi^{(j)}_2(\bm{y}) \pause
      = \sum_{i,j} \phi^{(ij)}_3(\bm{x})\,
      \phi^{(ij)}_3(\bm{y})
    \end{align*}
    where $\phi^{(ij)}_3(\bm{x}) = \phi^{(i)}_1(\bm{x})\,\phi^{(j)}_2(\bm{x})$\pauseb
  \end{itemize}
\end{PauseHighLight}

\end{slide}

%%%%%%%%%%%%%%%%%%%%%%% Next Slide %%%%%%%%%%%%%%%%%%%%%%%

\begin{slide}
\section{Exponentiating Kernels}

\begin{PauseHighLight}
  \begin{itemize}
  \item If $K(\bm{x}, \bm{y})$ is a valid kernel so is 
    $K^n(\bm{x}, \bm{y})$ (by induction)\pause
    \begin{itemize}\squeeze
    \item Assume $K(\bm{x}, \bm{y}) \succeq 0$ this satisfies base case\pause
    \item If $K^{n-1} (\bm{x}, \bm{y})\succeq 0$ then
      \begin{align*}
        K^{n}(\bm{x}, \bm{y}) = K^{n-1} (\bm{x}, \bm{y}) \, K(\bm{x}, \bm{y})\succeq 0\pause
      \end{align*}
    \end{itemize}
  \item and $\exp(K(\bm{x}, \bm{y}))$ is also a valid kernel since
    \begin{align*}
      \e{K(\bm{x}, \bm{y})} = \sum_k \frac{1}{i!} K^i(\bm{x}, \bm{y})
      = 1 + K(\bm{x}, \bm{y}) + \frac{1}{2} K^2(\bm{x}, \bm{y}) + \cdots
    \end{align*}
    but each term in the sum is a kernel\pause
  \end{itemize}
\end{PauseHighLight}

\end{slide}

%%%%%%%%%%%%%%%%%%%%%%% Next Slide %%%%%%%%%%%%%%%%%%%%%%%

\begin{slide}
\section{Gaussian Kernel}

\begin{PauseHighLight}
  \begin{itemize}
  \item Now $\bm{x}^\tr\bm{y}$ is a valid kernel because it is of the
    form $\sum_k \phi^{(k)}(\bm{x})\,\phi^{(k)}(\bm{y})$ where $\phi^{(k)}(\bm{x}) =
    x_k$\pause
  \item For $\gamma>0$ we have $2\,\gamma\,\bm{x}^\tr\bm{y}\succeq 0$
  \item Thus $\exp(2\,\gamma\,\bm{x}^\tr\bm{y}) \succeq 0$\pause
  \item Since $\exp(\gamma\,\bm{x}^\tr\bm{x})$ and
    $\exp(\gamma\,\bm{y}^\tr\bm{y})$ are positive numbers
    \begin{align*}
      \frac{\e{2\,\gamma\,\bm{x}^\tr\bm{y}}}{\e{\gamma\,\bm{x}^\tr\bm{x}}\,
        \e{\gamma\,\bm{y}^\tr\bm{y}}} &= \e{-\gamma\,\bm{x}^\tr\bm{x} +
        2\,\gamma\,\bm{x}^\tr\bm{y} - \gamma\,\bm{y}^\tr\bm{y}} \\
      &= \e{-\gamma\,\|\bm{x}-\bm{y}\|^2} \succeq 0\pause
    \end{align*}
  \end{itemize}
\end{PauseHighLight}

\end{slide}


%%%%%%%%%%%%%%%%%%%%%%% Next Slide %%%%%%%%%%%%%%%%%%%%%%%

\begin{slide}
\section{Other Kernels}

\begin{PauseHighLight}
  \begin{itemize}
  \item The success of SVMs has meant that researchers try to increase
    the area of application\pause
  \item The condition that a SVM kernel must be positive semi-definite
    is quite restrictive\pause
  \item There has been an industry of research finding smart kernels for
    solving complicated problems\pause
  \item The key to finding new kernels is to use the properties of
    kernels to build more complicated kernels from simpler ones\pause
  \end{itemize}
\end{PauseHighLight}

\end{slide}

%%%%%%%%%%%%%%%%%%%%%%% Next Slide %%%%%%%%%%%%%%%%%%%%%%%

\begin{slide}
\section{String Kernels}

\begin{PauseHighLight}
  \begin{itemize}
  \item One area where SVMs have become very important is in document
    classification\pause
  \item This requires comparing strings\pause
  \item There are a large number of kernels developed to do this\pause
  \end{itemize}
\end{PauseHighLight}

\end{slide}

%%%%%%%%%%%%%%%%%%%%%%% Next Slide %%%%%%%%%%%%%%%%%%%%%%%

\begin{slide}
\section[-1]{Spectrum Kernel}

\begin{PauseHighLight}
  \begin{itemize}
  \item A simple way to compare documents is to collect a histogram of
    all occurrences of substrings of length $p$\pause
  \item This is known as a $p$-spectrum\pause
  \item A $p$-spectrum kernel counts the number of common substrings
    {\small
    \begin{align*}
      s &= \mbox{\tt statistics} & \mathcal{S}_3(s) &= \{\mathtt{sta, tat, ati, tis,
      ist, sti, tic, ics}\}
      \\
      t &= \mbox{\tt computation} & \mathcal{S}_3(t) &= \{\mathtt{com, omp,
      mpu, put, uta, tat, ati, tio, ion}\}\pause
    \end{align*}}
  \item $K(s,t) = 2$ (``\texttt{tat}'' and ``\texttt{ati}'')\pause
  \end{itemize}
\end{PauseHighLight}

\end{slide}

%%%%%%%%%%%%%%%%%%%%%%% Next Slide %%%%%%%%%%%%%%%%%%%%%%%

\begin{slide}
\section{All Subsequences Kernel}

\begin{PauseHighLight}
  \begin{itemize}
  \item A more sophisticated kernel is to count all of the common
    subsequences that occur in two documents\pause
  \item Naively this would take an exponential amount of time to
    compute\pause
  \item Using clever dynamic-programming techniques this can be done
    relatively efficiently\pause
  \item This can even be extended to include sub-sequence matches with
    possible gaps between words\pause
  \end{itemize}
\end{PauseHighLight}

\end{slide}

%%%%%%%%%%%%%%%%%%%%%%% Next Slide %%%%%%%%%%%%%%%%%%%%%%%

\begin{slide}
\section{Other Kernel Applications}

\begin{PauseHighLight}
  \begin{itemize}
  \item String kernels for comparing subsequences are used in
    bioinformatics\pause
  \item Kernels have been developed for comparing trees (e.g. for
    computer program evaluation, XML, etc.)\pause
  \item Kernels have also been developed for comparing graphs (e.g. for
    comparing chemicals based on their molecular graph)\pause
  \end{itemize}
\end{PauseHighLight}

\end{slide}

%%%%%%%%%%%%%%%%%%%%%%% Next Slide %%%%%%%%%%%%%%%%%%%%%%%

\begin{slide}
\section{Fisher Kernels}

\begin{PauseHighLight}
  \begin{itemize}
  \item In an attempt to build kernels that capture more domain
    knowledge, kernels are constructed from other learning
    machines\pause
  \item An example of this are ``Fisher kernels'' whose features come
    from an Hidden Markov Model (HMM) trained on the data\pause
  \item These tend to have better discriminative power than the
    underlying model (HMM), and has a better feature set than a SVM
    using a generic kernel\pause
  \end{itemize}
\end{PauseHighLight}

\end{slide}





%%%%%%%%%%%%%%%%%%%%%%% Next Slide %%%%%%%%%%%%%%%%%%%%%%%
\Outline % Beyond
%%%%%%%%%%%%%%%%%%%%%%% Next Slide %%%%%%%%%%%%%%%%%%%%%%%



\begin{slide}
\section[-2]{Regression with Margins}

\begin{PauseHighLight}
  \begin{itemize}
  \item SVMs can be modified to perform regression
    \begin{center}
      \includegraphics[width=0.9\linewidth]{svm-regress}
    \end{center}
  \end{itemize}
\end{PauseHighLight}

\end{slide}

%%%%%%%%%%%%%%%%%%%%%%% Next Slide %%%%%%%%%%%%%%%%%%%%%%%

\begin{slide}
\section{Error Functions}

\begin{PauseHighLight}
  \begin{itemize}
  \item Can introduce slack variables with different errors
    \begin{center}
      \includegraphics[width=0.9\linewidth]{svm-regression}
    \end{center}
  \item This can be transformed to a quadratic programming problem
  \end{itemize}
\end{PauseHighLight}

\end{slide}

\begin{slide}

%%%%%%%%%%%%%%%%%%%%%%% Next Slide %%%%%%%%%%%%%%%%%%%%%%%

\section{Ridge Regression Using Kernels}

\begin{PauseHighLight}
  \begin{itemize}
  \item We can also solve regression problems without using margins\pause
  \item To solve a regression problem once again the problem is set up
    as a quadratic programming problem
    \begin{align*}
      \min_{\bm{w}} \lambda\,\| \bm{w} \|^2 + \sum_{i=1}^m \left( y_i -
      \bm{w}^\tr \bm{\phi}(\bm{x}_i) \right)^2\pause
    \end{align*}
  \item the $\| \bm{w} \|^2$ is a regularisation term\pause
  \item By assuming $\bm{w} = \sum_i \alpha_i \,\bm{\phi}(\bm{x}_i)$ we
    obtain a quadratic equation for the $\alpha_i$'s which we can
    solve\pause
  \end{itemize}
\end{PauseHighLight}

\end{slide}

%%%%%%%%%%%%%%%%%%%%%%% Next Slide %%%%%%%%%%%%%%%%%%%%%%%

\begin{slide}
\section{Kernel Methods}

\begin{PauseHighLight}
  \begin{itemize}
  \item Kernel methods where we project into an extended feature space
    is also used with algorithms
    \begin{itemize}
    \item Kernel Fisher discriminant analysis (KFDA)
    \item Kernel principle component analysis (KPCA)
    \item Kernel canonical correlation analysis (KCCA)
    \item Gaussian Processes\pause
    \end{itemize}
  \item These are also extremely power machine learning algorithms\pause
  \end{itemize}
\end{PauseHighLight}

\end{slide}


%%%%%%%%%%%%%%%%%%%%%%% Next Slide %%%%%%%%%%%%%%%%%%%%%%%

\begin{slide}
\section[-1]{Summary}

\begin{PauseHighLight}
  \begin{itemize}
  \item SVMs require a positive definite kernel function\pause
  \item These can be built from simpler function\pause
  \item There is an important industry of people creating new kernels
    for different application\pause
  \item SVMs are just one example of a host of machine that
    \begin{itemize}
    \item use the kernel trick
    \item often use linear constraints
    \item tend to be convex optimisation problems\pause
    \end{itemize}
  \end{itemize}
\end{PauseHighLight}

\end{slide}





%%% Local Variables:
%%% TeX-master: "lectures"
%%% End:
