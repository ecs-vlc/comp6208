% Created 2021-01-28 Thu 17:25
% Intended LaTeX compiler: pdflatex
\documentclass[11pt]{article}
\usepackage[utf8]{inputenc}
\usepackage[T1]{fontenc}
\usepackage{graphicx}
\usepackage{grffile}
\usepackage{longtable}
\usepackage{wrapfig}
\usepackage{rotating}
\usepackage[normalem]{ulem}
\usepackage{amsmath}
\usepackage{textcomp}
\usepackage{amssymb}
\usepackage{capt-of}
\usepackage{hyperref}
\usepackage{minted}
\usepackage[a4paper,margin=20mm]{geometry}
\usepackage{amsmath}
\usepackage{amsfonts}
\usepackage{stmaryrd}
\usepackage{bm}
\usepackage{minted}
\usemintedstyle{emacs}
\usepackage[T1]{fontenc}
\usepackage[scaled]{beraserif}
\usepackage[scaled]{berasans}
\usepackage[scaled]{beramono}
\newcommand{\tr}{\textsf{T}}
\newcommand{\grad}{\bm{\nabla}}
\newcommand{\av}[2][]{\mathbb{E}_{#1\!}\left[ #2 \right]}
\newcommand{\Prob}[2][]{\mathbb{P}_{#1\!}\left[ #2 \right]}
\newcommand{\logg}[1]{\log\!\left( #1 \right)}
\newcommand{\pred}[1]{\left\llbracket { \small #1} \right\rrbracket}
\newcommand{\e}[1]{{\rm e}^{#1}}
\newcommand{\dd}{\mathrm{d}}
\DeclareMathAlphabet{\mat}{OT1}{cmss}{bx}{n}
\newcommand{\normal}[2]{\mathcal{N}\!\left(#1 \big| #2 \right)}
\newcounter{eqCounter}
\setcounter{eqCounter}{0}
\newcommand{\explanation}{\setcounter{eqCounter}{0}\renewcommand{\labelenumi}{(\arabic{enumi})}}
\newcommand{\eq}[1][=]{\stepcounter{eqCounter}\stackrel{\text{\tiny(\arabic{eqCounter})}}{#1}}
\newcommand{\argmax}{\mathop{\mathrm{argmax}}}
\newcommand{\Dist}[2][Binom]{\mathrm{#1}\left( \strut {#2} \right)}
\author{Adam Prügel-Bennett}
\date{\today}
\title{Advanced Machine Learning Subsidary Notes\\\medskip
\large Lecture 12: Convexity}
\hypersetup{
 pdfauthor={Adam Prügel-Bennett},
 pdftitle={Advanced Machine Learning Subsidary Notes},
 pdfkeywords={},
 pdfsubject={},
 pdfcreator={Emacs 26.3 (Org mode 9.1.9)}, 
 pdflang={English}}
\begin{document}

\maketitle


\section{Keywords}
\label{sec:org2242e1b}
\begin{itemize}
\item Convex sets, convex functions, Jensen's inequality
\end{itemize}

\section{Main Points}
\label{sec:org01fde79}

\subsection{Convex Sets}
\label{sec:org451130e}
\begin{itemize}
\item We are familiar geometrically with convex regions
\item To define convexity we need to define an intermediate point
\(\bm{z} = a\,\bm{x} + (1-a)\,\bm{y}\)
\begin{itemize}
\item we requires \(a\in[0,1]\) (i.e. \(x\) is in the interval between 0 and 1)  for \(\bm{z}\) to be between \(\bm{x}\) and \(\bm{y}\)
\item to define convexity we only need to have addition and scalar multiplication
\end{itemize}
\item We can the define convexity in a very general way: a set
\(\mathcal{S}\) is convex if for every pair of points
\(\bm{x},\bm{y}\in \mathcal{S}\) and for every possible \(a\in[0,1]\) then
\(\bm{z} = a\,\bm{x} + (1-a)\,\bm{y} \in \mathcal{S}\)
\item We can apply convexity to sets of complicated objects
\item For example the set of positive semi-definite matrices forms a convex set
\begin{itemize}
\item This follows from the fact the sum of two positive
semi-definite matrices is also positive semi-definite and if we
multiply a positive semi-definite matrix by a positive number
then the matrix is still positive semi-definite
\end{itemize}
\end{itemize}

\subsection{Convex Functions}
\label{sec:org4ebd56a}
\begin{itemize}
\item We can define a function, \(f(\bm{x})\), to be a \emph{convex function}
if for all pairs of points in the domain of the function and all
\(a\in[0,1]\) $$ f(a \,\bm{x} + (1-a)\, \bm{y}) \leq a \,
     f(\bm{x}) + (1-a) \, f(\bm{y}) $$
\item This means that the function sits on or below the linear chord
connecting any two points in the domain of the function
\item The epigraph of a function is the area that lies on or above the
functions
\begin{itemize}
\item The epigraph of a convex function forms a convex region
\item If the epigraph of a function forms a convex region then the
function is convex
\end{itemize}
\item We can define \emph{convex-down} or \emph{concave} functions by inverting the constraint
$$  f(a \,\bm{x} + (1-a)\, \bm{y}) \geq  a \, f(\bm{x}) + (1-a) \, f(\bm{y}) $$
\begin{itemize}
\item for clarity I will sometimes refer to "convex" functions as
\emph{convex-up} functions
\item convex-down functions have similar (but opposite) properties to
convex up functions
\end{itemize}
\item A function where for every pair of points and for any \(a\) such
that \(0<a<1\) (i.e. a lies strictly between 0 and 1) then if
$$  f(a \,\bm{x} + (1-a)\, \bm{y}) < a \, f(\bm{x}) + (1-a) \, f(\bm{y}) $$
then function is said to be \emph{strictly convex}
\item Linear functions \(f(\bm{x}) = a\,\bm{x} +c\) are both convex-up
and convex-down functions
\begin{itemize}
\item For a function to be a strictly convex function it cannot have
a linear section
\end{itemize}
\item Convex functions lie on or above their tangent plane
\begin{itemize}
\item The tangent plane to a function \(f(\bm{x})\) at a point
\(\bm{x}_0\) is the plane orthogonal to the gradient, \(\grad
       f(\bm{x}_0)\) that goes through the point \(\bm{x}_0\)
\end{itemize}
\item A necessary and sufficient condition for a function to be convex
is that its second derivative is non-negative or for
multi-dimensional functions the Hessian is positive semi-definite
\begin{itemize}
\item If the second derivative is positive (i.e. always greater
than 0) or the Hessian is positive definite then the function
is strictly positive
\end{itemize}
\item Examples
\begin{itemize}
\item Convex-up Functions
\begin{itemize}
\item \(f(x) = x^2\)  is strictly convex since \(f''(x)=2>0\)
\item \(f(x) = x^{-2}\)  is strictly convex since \(f''(x)=2\,x^{-4}>0\)
\item \(f(x) = x^4\)  is convex since \(f''(x)=12\,x^2\geq0\)
\item \(f(x) = \e{c\,x}\) is strictly convex for all \(c\) as \(f''(x)=c^2\,\e{c\,x}>0\)
\item \(f(\bm{x}) = \|\bm{x}\|^2\)  is strictly convex since \(\mat{H}(x)=\mat{I}\succ0\)
\item \(f(x) = |x|\) is convex since for \(a\in[0,1]\) we have \(|a\,x +
         (1-a)\,y| \leq a\,|x| + (1-a)\,|y|\) with equality only when
\(x\,y\geq 0\)
\end{itemize}
\item Convex-down Functions
\begin{enumerate}
\item \(f(x) = -x^2\)  is strictly convex-down since \(f''(x)=-2<0\)
\item \(f(x) = \sqrt{x}\)  (for \(x>0\)) is strictly convex-down since \(f''(x)=-x^{-3/2}/4<0\)
\item \(f(x) = \log(x)\)  is strictly convex-down since \(f''(x)=-1/x^2<0\)
\end{enumerate}
\end{itemize}
\item A function \(f(\bm{x})\) that is constrained to a convex domain
(\(\bm{x}\in\mathcal{S}\), where \(\mathcal{S}\) is a convex set) is
convex in that domain if for all pairs
\(\bm{x},\bm{y}\in\mathcal{S}\) and all \(a\in[0,1]\) we have
 $$  f(a \,\bm{x} + (1-a)\, \bm{y}) \leq  a \, f(\bm{x}) + (1-a)  \, f(\bm{y}) $$
\begin{itemize}
\item This is just a more precise definition of a convex function
\item Note that by limiting the domain of a function some non-convex
functions may be convex over that domain
\begin{itemize}
\item e.g. \(\cos(x)\) is convex in the interval \([-\pi/2,3\pi/3]\)
\end{itemize}
\item A convex function constrained to lie in a convex set will still
be convex
\item Any combination of linear constraints will form a convex set
\item Therefore convex functions restricted to satisfy linear
constraints will be convex
\end{itemize}
\item The minimum of a convex function will form a convex set
\begin{itemize}
\item There can be no local minima
\item For a strictly convex function the minimum will be unique
\end{itemize}
\item The sum of convex functions will be convex
\item \textbf{Linear regression}
\begin{itemize}
\item The loss function of linear regression is convex
$$ L(\bm{w}) = \| \mat{X}\bm{w} - \bm{y} \|^2 $$
\begin{itemize}
\item The Hessian is \(\mat{X}^\tr\mat{X}\) which is positive
semi-definite which is a sufficient condition for \(L(\bm{w})\)
to be convex
\end{itemize}
\item Both the \(L_2\) regulariser and the \(L_1\) regulariser are convex
\item The \(L_2\) regulariser is strictly convex so there will be a
unique solution
\item Many machine learning algorithms are chosen because they
involve minimising a convex function leading to a unique minimum
\end{itemize}
\end{itemize}

\subsection{Jensen's Inequality}
\label{sec:orga1151ea}
\begin{itemize}
\item For any convex-up function, if \(\bm{x}\) is a random variable then
$$ \av{f(\bm{x})} \geq f(\av{\bm{x}}) $$
\begin{itemize}
\item \(\av{\cdots}\) denotes the expectation
\end{itemize}
\item For any convex-down function
$$ \av{f(\bm{x})} \leq f(\av{\bm{x}}) $$
\item These are known as \emph{Jensen's Inequality}
\item \textbf{Proof}
\begin{itemize}
\item We can prove this starting from the fact that \(f(\bm{x})\) lies
above the tangent plane at any point
$$ f(\bm{x}) \geq f(\bm{x}^*) + (\bm{x}-\bm{x}^*)^\tr \grad
       f(\bm{x}^*) $$
\item This has to be true at the point \(\bm{x}^*=\av{\bm{x}}\)
$$ f(\bm{x}) \geq f(\av{\bm{x}}) + (\bm{x}-\av{\bm{x}})^\tr
       \grad f(\av{\bm{x}}) $$
\item Taking expectations of both sides of the equation
$$ \av{f(\bm{x})} \geq f(\av{\bm{x}}) +
	 (\av{\bm{x}}-\av{\bm{x}})^\tr \grad f(\av{\bm{x}})\\
	 &= f(\av{\bm{x}}) \hspace{2cm} \square $$
\end{itemize}
\item Using Jensen's Inequality
\begin{itemize}
\item Consider the strictly convex function \(f(\bm{x})=x^2\) by
Jensen's inequality
$$ \av{x^2} \geq \av{x}^2$$
\begin{itemize}
\item or \(\av{x^2}-\av{x}^2 \geq 0\)
\begin{itemize}
\item the left-hand side is the variance so we see variances are non-negative
\item because \(f(\bm{x})=x^2\) is strictly convex we only get
equality where \(\bm{x}\) doesn't vary at all
\end{itemize}
\end{itemize}
\item Consider the Kullback-Liebler (KL) divergence defined for
discrete probability probability distributions defined over the
same range as
$$ \mathcal{KL}(f \| g) = - \sum_i f_i \,  \logg{\frac{g_i}{f_i}} $$
\begin{itemize}
\item This is often used to measure how different distribution are
from each other
\item Note if \(g_i=f_i\) then \(\mathcal{KL}(f \| g) =0\) since \(\log(1)=0\)
\item Now we can use Jensen's inequality to show that
\(\mathcal{KL}(f \| g) \geq 0\)
\begin{align*}
\mathrm{KL}(f \| g) &= - \sum_i f_i \, \logg{\frac{g_i}{f_i}}
=- \av[f]{\logg{\frac{g_i}{f_i}}}\\
&\geq -\logg{\av[f]{\frac{g_i}{f_i}}} \\
&= - \logg{\sum_i f_i \frac{g_i}{f_i}} = -\logg{\sum_i g_i} = -\log(1)  = 0
\end{align*}
\begin{itemize}
\item Here we are assuming we have random variable that take
values \(X_i=g_i/f_i\) that occur with probability \(f_i\)
\item The KL-divergence is therefore equal to \(\av{-\log(X_i)}\)
\item Since \(-\log(x)\) is convex up we have by Jensen's
inequality that the KL-divergences is greater than or equal
to \(-\logg{\av{X_i}} = -\logg{\sum_i f_i\,X_i}\)
\item But \(X_i=g_i/f_i\) so the KL-divergence is greater than
\(-\logg{\sum_i g_i}\)
\item But \(g_i\) is a probability so \(\sum_i g_i=1\) giving us our result
\end{itemize}
\item This is known as the Gibbs' inequality after the mathematical
physicist, J. Willard Gibbs, (founder of modern statistical
mechanics) who first proved it
\item We often use KL-divergences when we want to choose the
parameters of one probability distribution so that it
approximates a second probability distribution
\end{itemize}
\end{itemize}
\end{itemize}

\section{Exercises}
\label{sec:orga59cdf1}

\subsection{Positive quadrant}
\label{sec:orgcb16fdf}
\begin{itemize}
\item Prove that the set of vectors with non-negative elements form a
convex set
\end{itemize}

\subsection{Inverse of Convex Functions}
\label{sec:org3e18f11}
\begin{enumerate}
\item Use the chain rule to compute the second derivative of \(f(g(x))\)
\item If \(g(x) = f^{-1}(x)\) show that the second derivative of
\(f(g(x))\) vanishes
\item Use these results to derive an identity for the second derivative
of \(f^{-1}(x)\)
\item Derive a condition for \(f^{-1}(x)\) to be a
convex-down function given that \(f(x)\) is convex-up
\item Use this to show
\begin{enumerate}
\item \(\sqrt{x}\) is a convex-down function
\item \(\log(x)\) is a convex-down function
\end{enumerate}
\end{enumerate}

\subsection{Cumulant Generating Function}
\label{sec:orge826acd}
\begin{itemize}
\item Here is something a bit harder (which you don't need to know)
\item The cumulant generating function of a probability distribution
\(p(x)\) is defined as
$$ G(\lambda) = \logg{\av{\e{\lambda\,x}}} $$
\begin{itemize}
\item the expectation is over the random variable \(x\) drawn from \(p(x)\)
\end{itemize}
\item It is called the cumulant generating function because it we take
then \(n^{th}\) derivative and set \(\lambda\) to zero we obtain the
\(n^{th}\) cumulant (i.e. \(\kappa_n = G^{(n)}(0)\))
\item The first cumulant is the mean, the second the variance while the
third and forth are proportional to the skewness and kurtosis
\item Cumulant generating functions appear a lot when you work with
probabilities, but go beyond this course
\item Nevertheless let's show they are convex
\begin{enumerate}
\item Find the second derivative
\item Show that if \(p(x)\) is a probability distribution then \(q(x) =
        p(x)\,\e{\lambda\,x}/\av{\e{\lambda\,x}}\) is also a
probability distribution
\item Hence show that the cumulant generating function is convex
\end{enumerate}
\item See answers
\end{itemize}


\section{Answers}
\label{sec:org7d3d70b}

\subsection{Positive quadrant}
\label{sec:org8728ec0}
\begin{itemize}
\item Let \(\mathcal{P}\) be the set of vectors with non-negative elements
\item If \(\bm{x}\in\mathcal{P}\) then if \(c\geq0\) we have
\(\bm{v} = c\,\bm{x}\in\mathcal{P}\) since each element of \(\bm{v}\) will
be non-negative (i.e. \(v_i=c\,x_i\geq0\))
\item Also for any two vectors \(\bm{x},\bm{y}\in\mathcal{P}\) clearly
\(\bm{w} = \bm{x} + \bm{y} \in \mathcal{P}\) since \(w_i=x_+y_i\)
\item Thus for any two vectors  \(\bm{x},\bm{y}\in\mathcal{P}\) and any
\(a\in\{0,1\}\) the vector \(\bm{z} = a\,\bm{x} +(1-a)\,\bm{y}\) will
be in \(\mathcal{P}\)
\end{itemize}

\subsection{Inverse of Convex Functions}
\label{sec:org0df6b63}
\begin{enumerate}
\item Taking derivatives
$$ \frac{\dd^2 f(g(x))}{\dd x^2} = \frac{\dd
      f'(g(x))\,g'(x)}{\dd x} = f''(g(x))\, (g'(x))^2 + f'(g(x))\,
      g''(x) $$
\item If \(g(x) = f^{-1}(x)\) then \(f(g(x))=x\) and the second derivative vanishes
\item Using 1. and 2. we find (writing \(f^{-1}(x)\) as \(g(x)\))
$$ g''(x) = - \frac{f''(g(x))\, (g'(x))^2}{f'(g(x))} $$
\item If \(f(x)\) is convex then \(f''(y)\geq0\) for any \(y\) (including
\(y=f^{-1}(x)\)) also \((g'(x))^2\geq0\) so for the inverse of
\(f(x)\) to be convex down we require \(f'(f^{-1}(x))>0\)
\item Use this to show
\begin{enumerate}
\item Let \(f(x)=x^2\), so that \(f''(x)=2>0\) and \(f'(y)=y\) which is
non-negative if \(y\geq0\), but \(f^{-1}(x)=\sqrt{x}>0\) so
\(f'(f^{-1}(x))\geq0\) and consequently \(\sqrt{x}\) is
convex-down
\item Let \(f(x)=\exp(x)\), so that \(f''(x) = \exp(x) >0\).  But
\(f'(y) = \exp(y)>0\) for all \(y\) so \(f'(f^{-1}(x))>0\) which is
sufficient to show \(f^{-1}(x)=\log(x)\) is a convex-down
function
\end{enumerate}
\end{enumerate}

\subsection{Cumulant Generating Function}
\label{sec:orgcbf8de7}
\begin{enumerate}
\item If \(G(\lambda) = \logg{\av{\e{\lambda\,x}}}\) then
$$ G'(\lambda) = \frac{ \av{x\,\e{\lambda\,x}} }{
      \av{\e{\lambda\,x}} } $$
and
$$ G''(\lambda) = \frac{ \av{x^2\,\e{\lambda\,x}} }{
      \av{\e{\lambda\,x}} } - \frac{ \av{x\,\e{\lambda\,x}}^2 }{
      \av{\e{\lambda\,x}}^2 } $$
\item \begin{itemize}
\item Now if \(p(x)\) is a probability distribution is will be
non-negative for all \(x\) and sum or integrate to 1
\item But then \(q(x) = p(x)\,\e{\lambda\,x}/\av{\e{\lambda\,x}}\)
will be non-negative as \(\e{\lambda\,x}>0\) and
\(\av{\e{\lambda\,x}}>0\) (the expectation of positive
quantities will be positive)
\item But
$$ \int q(x) \, \dd x = \frac{1}{\av{\e{\lambda\,x}}} \int
        p(x)\, \e{\lambda\,x}\,\dd x=
        \frac{\av{\e{\lambda\,x}}}{\av{\e{\lambda\,x}}} = 1 $$
\item So \(q(x)\) is non-negative and normalised so is a well defined
probability distribution
\end{itemize}
\item Using the result of 1. and 2
$$ G''(\lambda) = \frac{ \av[p]{x^2\,\e{\lambda\,x}} }{
      \av[p]{\e{\lambda\,x}} } - \frac{ \av[p]{x\,\e{\lambda\,x}}^2 }{
      \av[p]{\e{\lambda\,x}}^2 } = \av[g]{x^2} - \av[g]{x}^2 \geq 0 $$
\begin{itemize}
\item since variances are non-negative
\end{itemize}
\end{enumerate}
\end{document}
