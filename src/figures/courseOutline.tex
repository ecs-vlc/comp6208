%Master File:lectures.tex

\lesson{Course Outline}
\vspace{-1cm}
\begin{center}
\noindent\includegraphics[height=100mm]{cnn1.jpeg}
\end{center}
\keywords{Course Details and Topics}
%%%%%%%%%%%%%%%%%%%%%%% Next Slide %%%%%%%%%%%%%%%%%%%%%%%
\renewcommand{\Outline}{%
\begin{slide}
\section[1]{Outline}

\begin{minipage}{12cm}\raggedright
  \begin{enumerate}\squeeze
    \outlineitem{Course Outline}{outline}
  \end{enumerate}
\end{minipage}\hfill
\begin{minipage}{10cm}
  \includegraphics[width=10cm]{cnn1.jpeg}
\end{minipage}
\end{slide}
\addtocounter{outlineitem}{1}
}

\setcounter{outlineitem}{1}


%%%%%%%%%%%%%%%%%%%%%%% Next Slide %%%%%%%%%%%%%%%%%%%%%%%
%\Outline % Outline
\toptarget{firstoutline}
%%%%%%%%%%%%%%%%%%%%%%% Next Slide %%%%%%%%%%%%%%%%%%%%%%%

\begin{slide}
\section{Course Structure}

\begin{PauseHighLight}
  \begin{itemize}
  \item Lectures
    \begin{itemize}
    \item 15:00 Tuesday 35/1001
    \item 11:00 Wednesday 02/1089
    \item 15:00 Friday (was 02A/2077---needs to be changed)\pause
    \end{itemize}
  \item Assessment
    \begin{itemize}
    \item 80\% exam
    \item 20\% Problem Sheets\pause
    \end{itemize}
  \end{itemize}
\end{PauseHighLight}

\end{slide}

%%%%%%%%%%%%%%%%%%%%%%% Next Slide %%%%%%%%%%%%%%%%%%%%%%%

\begin{slide}
\section{Problem Sheets}

\begin{PauseHighLight}
  \begin{itemize}
  \item I am changing the assessment from a group project to problem sheets\pause
  \item I will give out two problem sheets each worth 10\%\pause
  \item They will help you understand the mathematical material\pause
  \item They should prepare you for the exam\pause
  \end{itemize}
\end{PauseHighLight}

\end{slide}

%%%%%%%%%%%%%%%%%%%%%%% Next Slide %%%%%%%%%%%%%%%%%%%%%%%

\begin{slide}
  \section[-1]{What's in the Course}

  \begin{PauseHighLight}
    \begin{itemize}
    \item This course is going to cover the core principles and
      mathematics behind machine learning\pause
    \item It is not going to explicitly teach different machine
      learning algorithms\pause
    \item We are not looking at advanced algorithms but cover the
      principles\pause
    \item There are very good implementation available (e.g.{}
      scikit-learn)\pause
    \item Along the way though we will meet (often many times)
      particular algorithms\pause
    \end{itemize}
  \end{PauseHighLight}

\end{slide}


%%%%%%%%%%%%%%%%%%%%%%% Next Slide %%%%%%%%%%%%%%%%%%%%%%%

\begin{slide}
\section{Topics}

\begin{PauseHighLight}
  \begin{itemize}
  \item Learning Theory
    \begin{itemize}\squeeze
    \item Bias-Variance\pause
    \item Overfitting, structure and regularisation\pause
    \item Ensembling, bagging and boosting\pause
    \end{itemize}
  \item Mathematics
    \begin{itemize}\squeeze
    \item Function Spaces: Kernel Methods and Gaussian Processes\pause
    \item Linear Algebra, embeddings, positive definiteness, subspace,
      determinants\pause
%    \item Non-linear Embeddings: tSNE\pause
    \end{itemize}
  \end{itemize}
\end{PauseHighLight}

\end{slide}

%%%%%%%%%%%%%%%%%%%%%%% Next Slide %%%%%%%%%%%%%%%%%%%%%%%

\begin{slide}
  \section[-2]{Topics Continued}

  \begin{PauseHighLight}
    \begin{itemize}
    \item Optimisation
      \begin{itemize}\squeeze
      \item Newton/Quasi-Newton Methods: convergence rates\pause
      \item SGD, momentum, ADAM\pause
      \end{itemize}
    \item  Constrainted Optimisation
      \begin{itemize}\squeeze
      \item KKT conditions\pause
      \item Duality Linear/Quadratic Programming\pause
      \item SVMs\pause
      \end{itemize}
    \item Convexity
      \begin{itemize}\squeeze
      \item Convex sets: linear constraints, PD matrices\pause
      \item Convex functions\pause
      \item SVMs, Lasso\pause
      \item Jensen's inequality\pause
      \end{itemize}
    \end{itemize}
  \end{PauseHighLight}
\end{slide}

%%%%%%%%%%%%%%%%%%%%%%% Next Slide %%%%%%%%%%%%%%%%%%%%%%%

\begin{slide}
\section{Topics Continued}

\begin{PauseHighLight}
  \begin{itemize}
  \item Probability
    \begin{itemize}
    \item Naive Bayes\pause
    \item Gaussian Processes\pause
    \item Dependencies and Graphical Models\pause
    \item Expectations and MCMC\pause
    \end{itemize}
  \item Variational Methods
    \begin{itemize}
    \item Divergences: KL and Wasserstein\pause
    \item VAEs and GANs\pause
    \item Variational Approximation\pause
    \end{itemize}
  \end{itemize}
\end{PauseHighLight}

\end{slide}


%%% Local Variables:
%%% TeX-master: "lectures"
%%% End:
