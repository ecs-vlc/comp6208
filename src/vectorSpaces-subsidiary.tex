% Created 2024-01-17 Wed 17:31
% Intended LaTeX compiler: pdflatex
\documentclass[11pt]{article}
\usepackage[utf8]{inputenc}
\usepackage[T1]{fontenc}
\usepackage{graphicx}
\usepackage{grffile}
\usepackage{longtable}
\usepackage{wrapfig}
\usepackage{rotating}
\usepackage[normalem]{ulem}
\usepackage{amsmath}
\usepackage{textcomp}
\usepackage{amssymb}
\usepackage{capt-of}
\usepackage{hyperref}
\usepackage{minted}
\usepackage[a4paper,margin=20mm]{geometry}
\usepackage{amsmath}
\usepackage{amsfonts}
\usepackage{stmaryrd}
\usepackage{bm}
\usepackage{minted}
\usemintedstyle{emacs}
\usepackage[T1]{fontenc}
\usepackage[scaled]{beraserif}
\usepackage[scaled]{berasans}
\usepackage[scaled]{beramono}
\newcommand{\tr}{\textsf{T}}
\newcommand{\grad}{\bm{\nabla}}
\newcommand{\av}[2][]{\mathbb{E}_{#1\!}\left[ #2 \right]}
\newcommand{\Prob}[2][]{\mathbb{P}_{#1\!}\left[ #2 \right]}
\newcommand{\logg}[1]{\log\!\left( #1 \right)}
\newcommand{\pred}[1]{\left\llbracket { \small #1} \right\rrbracket}
\newcommand{\e}[1]{{\rm e}^{#1}}
\newcommand{\dd}{\mathrm{d}}
\DeclareMathAlphabet{\mat}{OT1}{cmss}{bx}{n}
\newcommand{\normal}[2]{\mathcal{N}\!\left(#1 \big| #2 \right)}
\newcounter{eqCounter}
\setcounter{eqCounter}{0}
\newcommand{\explanation}{\setcounter{eqCounter}{0}\renewcommand{\labelenumi}{(\arabic{enumi})}}
\newcommand{\eq}[1][=]{\stepcounter{eqCounter}\stackrel{\text{\tiny(\arabic{eqCounter})}}{#1}}
\newcommand{\argmax}{\mathop{\mathrm{argmax}}}
\newcommand{\Dist}[2][Binom]{\mathrm{#1}\left( \strut {#2} \right)}
\author{Adam Prügel-Bennett}
\date{\today}
\title{Advanced Machine Learning Subsidary Notes\\\medskip
\large Lecture 5: Vector Spaces}
\hypersetup{
 pdfauthor={Adam Prügel-Bennett},
 pdftitle={Advanced Machine Learning Subsidary Notes},
 pdfkeywords={},
 pdfsubject={},
 pdfcreator={Emacs 27.1 (Org mode 9.3)}, 
 pdflang={English}}
\begin{document}

\maketitle



\section{Keywords}
\label{sec:orga4e4ae1}
\begin{itemize}
\item Vectors, vector spaces, operators
\end{itemize}

\section{Main Points}
\label{sec:org5b8e772}

\subsection{Vector Spaces}
\label{sec:orgfef20bd}
\begin{itemize}
\item Any set of objects with addition between members of the set and
scalar multiplication forms a vector space if they satisfies 8 axioms
\item Most of these axioms arise naturally if addition and scale multiplication
behave normally
\item The only additional axiom is closure
\item Normal vectors, matrices and functions all form vector spaces
\end{itemize}

\subsection{Distances}
\label{sec:org462e5de}
\begin{itemize}
\item A \emph{proper distance} or \emph{metric} between objects in a vector space
satisfies 4 conditions
\begin{enumerate}
\item \(d(\bm{x},\bm{y})\geq0\) (non-negativity)
\item \(d(\bm{x},\bm{y}) = 0\) if and only if \(\bm{x}=\bm{y}\) (identity of indiscernibles)
\item \(d(\bm{x},\bm{y}) = d(\bm{y},\bm{x})\) (symmetry)
\item \(d(\bm{x},\bm{y}) \leq d(\bm{x},\bm{z}) + d(\bm{z},\bm{y})\) (triangular inequality)
\end{enumerate}
\item You can define different distances for the same set of objects
\item Often we use \emph{pseudo-metrics} that breaks one or other of the conditions
\item Consider a function (mapping) \(f:\mathbb{R}\right\mathbb{R}\) then
a useful quantification of how big a change a mapping can produce
is given by the Lipschitz condition
\begin{align*}
 d(f(x), f(y)) \leq K\, d(x,y)
\end{align*}
This is useful if for all \(x\) and \(y\) there exists some fixed \(K\)
\item Lipschitz functions are continuous (have no jumps)
\item If \(K<1\) the mapping is said to be a contractive mapping
\end{itemize}
\subsection{Norms}
\label{sec:org27f46fb}
\begin{itemize}
\item Norms provide a measure of the size of a vector
\item They satisfy three conditions
\begin{enumerate}
\item \(\| \bm{v} \| >0\) if \(\bm{v}\neq\bm{0}\) (non-negativity)
\item \(\| a\,\bm{v} \| = a \| \bm{v} \|\) (linearity)
\item \(\| \bm{u} + \bm{v} \| \leq \| \bm{u} \| + \| \bm{v} \|\) (triangular inequality)
\end{enumerate}
\item Again if not all of these conditions are true we have \emph{pseudo-norms}
\item Norms provide a metric \(d(\bm{x}, \bm{y}) = \|\bm{x}-\bm{y}\|\)
\item We will meet norms very often in this course
\item \textbf{Vector Norms}
\begin{itemize}
\item There are a large number of norms for normal vectors that people use
\begin{enumerate}
\item Euclidean or 2-norm: \(\| \bm{v} \|_2 = \sqrt{v_1^2 + v_2^2 + \cdots + v_n^2}\)
\item \(p\)-norm: \(\| \bm{v} \|_p = \left(\sum_{i=1}^n | v_i |^p \right)^{1/p}\)
\item 1-norm: \(\| \bm{v} \|_1 &= \sum\limits_{i=1}^n | v_i |\)
\item \(\infty\)-norm or max-norm: \(\| \bm{v} \|_{\infty} &= \max_i |v_i|\)
\end{enumerate}
\item Note the 1-norm, 2-norm and \(\infty\)-norm are all \(p\)-norms with different \(p\)
\item The 0-norm counts the number of non-zero components (it is a
pseudo-norm as it is not linear)
\end{itemize}
\item \textbf{Matrix Norms}
\begin{itemize}
\item Matrices also have norm
\begin{enumerate}
\item The Frobenius norm is \(\| \mat{A} \|_F = \sqrt{\sum_{i=1}^m \sum_{j=1}^n |A_{ij}|^2}\)
\item Also have 1-norm, max-norm, Hilbert-norm (the maximum absolute eigenvalue), nuclear-norm, etc.
\end{enumerate}
\item Note that the determinant is not a norm because it can be negative and is not linear
\item Many of the commonly used matrix norms satisfy
\begin{align*}
   \| \mat{A}\,\mat{B} \| \leq \| \mat{A} \| \times \| \mat{B} \|\pause
\end{align*}
\item This is really useful because we can quickly bound norms of products of matrices
\item Many matrix and vector norms are compatible
\begin{align*}
    \| \mat{M} \bm{v} \|_b \leq \| \mat{M} \|_a \times \| \bm{v} \|_b
 \end{align*}
\item E.g. Frobenius and Euclidean norms are compatible
\item One of the main uses of matrix norms is to understand by how much it
can potentially increase the size of a vector
\end{itemize}
\item \textbf{Function Norms}
\begin{itemize}
\item The most common function norms are
\begin{enumerate}
\item The \(L_2\)-norm
\begin{align*}
\| f \|_{L_2} = \sqrt{\int_{\bm{x}\in\mathcal{R}} f^2(\bm{x}) \, \dd \bm{x}}
\end{align*}
where \(\mathcal{R}\) is the region over which the function is define
\item The \(L_1\)-norm
\begin{align*}
\| f \|_{L_1} = \int_{\bm{x}\in\mathcal{R}} |f(\bm{x})| \, \dd \bm{x}
\end{align*}
\item The \(\infty\) or max-norm
\begin{align*}
\| f \|_{\infty} = \max_{\bm{x}\in\mathcal{R}} f(\bm{x})
\end{align*}
\end{enumerate}
\item Function norms are also used to define vector spaces
\begin{enumerate}
\item The \(L_2\) vector space is the set of functions such that all 
functions satisfy \(\| f \|_{L_2}<\infty\)
\item The \(L_1\) vector space is the set of functions such that all 
functions satisfy \(\| f \|_{L_1}<\infty\)
\end{enumerate}
\item In these vector spaces we only consider functions that measurable in
the sense that \(\|f\|>0\) for any non-zero function
\end{itemize}
\end{itemize}
\end{document}
