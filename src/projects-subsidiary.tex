% Created 2021-01-28 Thu 17:14
% Intended LaTeX compiler: pdflatex
\documentclass[11pt]{article}
\usepackage[utf8]{inputenc}
\usepackage[T1]{fontenc}
\usepackage{graphicx}
\usepackage{grffile}
\usepackage{longtable}
\usepackage{wrapfig}
\usepackage{rotating}
\usepackage[normalem]{ulem}
\usepackage{amsmath}
\usepackage{textcomp}
\usepackage{amssymb}
\usepackage{capt-of}
\usepackage{hyperref}
\usepackage{minted}
\usepackage[a4paper,margin=20mm]{geometry}
\usepackage{amsmath}
\usepackage{amsfonts}
\usepackage{stmaryrd}
\usepackage{bm}
\usepackage{minted}
\usemintedstyle{emacs}
\usepackage[T1]{fontenc}
\usepackage[scaled]{beraserif}
\usepackage[scaled]{berasans}
\usepackage[scaled]{beramono}
\newcommand{\tr}{\textsf{T}}
\newcommand{\grad}{\bm{\nabla}}
\newcommand{\av}[2][]{\mathbb{E}_{#1\!}\left[ #2 \right]}
\newcommand{\Prob}[2][]{\mathbb{P}_{#1\!}\left[ #2 \right]}
\newcommand{\logg}[1]{\log\!\left( #1 \right)}
\newcommand{\pred}[1]{\left\llbracket { \small #1} \right\rrbracket}
\newcommand{\e}[1]{{\rm e}^{#1}}
\newcommand{\dd}{\mathrm{d}}
\DeclareMathAlphabet{\mat}{OT1}{cmss}{bx}{n}
\newcommand{\normal}[2]{\mathcal{N}\!\left(#1 \big| #2 \right)}
\newcounter{eqCounter}
\setcounter{eqCounter}{0}
\newcommand{\explanation}{\setcounter{eqCounter}{0}\renewcommand{\labelenumi}{(\arabic{enumi})}}
\newcommand{\eq}[1][=]{\stepcounter{eqCounter}\stackrel{\text{\tiny(\arabic{eqCounter})}}{#1}}
\newcommand{\argmax}{\mathop{\mathrm{argmax}}}
\newcommand{\Dist}[2][Binom]{\mathrm{#1}\left( \strut {#2} \right)}
\author{Adam Prügel-Bennett}
\date{\today}
\title{Advanced Machine Learning Subsidary Notes\\\medskip
\large Lecture 3: Project}
\hypersetup{
 pdfauthor={Adam Prügel-Bennett},
 pdftitle={Advanced Machine Learning Subsidary Notes},
 pdfkeywords={},
 pdfsubject={},
 pdfcreator={Emacs 26.3 (Org mode 9.1.9)}, 
 pdflang={English}}
\begin{document}

\maketitle



\section{Keywords}
\label{sec:org4013321}
\begin{itemize}
\item Details, Ideas, Research Methods
\end{itemize}

\section{Main Points}
\label{sec:org1b48498}

\subsection{Project}
\label{sec:org1ea62cc}
\begin{itemize}
\item You are going to projects in groups of 3-5
\item This is worth 40\% of the course
\item You should submit two report of no more than 4 pages
\begin{enumerate}
\item A description of your exploration of the data
\item A description of running machine learning
\end{enumerate}
\item A good place for finding projects is \texttt{//www.kaggle.com/}
\end{itemize}

\subsection{How to do a machine learning project}
\label{sec:orgd65e395}

This is taken from the course text \emph{Hands-on Machine Learning with Scikit-Learn, Keras, and TensorFlow: Concepts, Tools, and Techniques to Build Intelligent Systems} by Aurélien Géron

\begin{enumerate}
\item Frame the Problem and Look at the Big Picture
\begin{enumerate}
\item Define the objective in business terms
\item How will your solution be used?
\item What are the current solutions/workarounds (if any)?
\item How should you frame this problem (supervised/unsupervised, online/offline, etc.)?
\item How should performance be measured?
\item Is the performance measure aligned with the business objective?
\item What would be the minimum performance needed to reach the business objective?
\item What are comparable problems? Can you reuse experience or tools?
\item Is human expertise available?
\item How would you solve the problem manually?
\item List the assumptions you (or others) have made so far
\item Verify assumptions if possible
\end{enumerate}
\item Get the Data
\begin{enumerate}
\item List the data you need and how much you need
\item Find and document where you can get that data
\item Check how much space it will take
\item Check legal obligations, and get authorization if necessary
\item Get access authorisations
\item Create a workspace (with enough storage space)
\item Get the data
\item Convert the data to a format you can easily manipulate (without changing the data itself)
\item Ensure sensitive information is deleted or protected (e.g., anonymized)
\item Check the size and type of data (time series, sample, geographical, etc.)
\item Sample a test set, put it aside, and never look at it (no data snooping!)
\end{enumerate}
\item Explore the Data\\
\textbf{Note: try to get insights from a field expert for these steps}
\begin{enumerate}
\item Create a copy of the data for exploration (sampling it down to a manageable size if necessary)
\item Create a Jupyter notebook to keep a record of your data exploration
\item Study each attribute and its characteristics:
\begin{itemize}
\item Name
\item Type (categorical, int/float, bounded/unbounded, text, structured, etc)
\item Number of missing values
\item Noisiness and type of noise (stochastic, outliers, rounding errors, etc.)
\item Possibly useful for the task?
\item ype of distribution (Gaussian, uniform, logarithmic, etc.)
\end{itemize}
\item For supervised learning tasks, identify the target attribute(s)
\item Visualise the data
\item Study the correlations between attributes
\item Study how you would solve the problem manually
\item Identify the promising transformations you may want to apply
\item Identify extra data that would be useful
\item Document what you have learned
\end{enumerate}
\item Prepare the Data
\begin{enumerate}
\item Notes:
\begin{itemize}
\item Work on copies of the data (keep the original dataset intact)
\item Write functions for all data transformations you apply, for five reasons:
∗ So you can easily prepare the data the next time you get a fresh dataset
∗ So you can apply these transformations in future project
∗ To clean and prepare the test set
∗ To clean and prepare new data instances once your solution is live
∗ To make it easy to treat your preparation choices as hyperparameters
\end{itemize}

\item Data cleaning
\begin{itemize}
\item Fix or remove outliers (optional)
\item Fill in missing values (e.g., with zero, mean, median. . . ) or drop their rows (or columns)
\end{itemize}
\item Feature selection (optional):
\begin{itemize}
\item Drop the attributes that provide no useful information for the task
\end{itemize}
\item Feature engineering, where appropriate:
\begin{itemize}
\item Discretise continuous features
\item Decompose features (e.g., categorical, date/time, etc.)
\item Add promising transformations of features (e.g., \(\log(x)\), etc.)
\item Aggregate features into promising new features
\end{itemize}
\item Feature scaling: standardise or normalise features
\end{enumerate}
\item Short-List Promising Models
\begin{enumerate}
\item Train many quick and dirty models from different categories (e.g., linear, naive Bayes, SVM, Random Forests, neural net, etc.) using standard parameters
\item Measure and compare their performance
\begin{itemize}
\item For each model, use N-fold cross-validation and compute the mean and standard deviation of the performance measure on the N folds
\end{itemize}
\item Analyse the most significant variables for each algorithm
\item Analyse the types of errors the models make
\begin{itemize}
\item What data would a human have used to avoid these errors?
\end{itemize}
\item Have a quick round of feature selection and engineering
\item Have one or two more quick iterations of the five previous steps
\item Short-list the top three to five most promising models, preferring models that make different types of errors
\end{enumerate}
\item Fine-Tune the System
Note 1: You will want to use as much data as possible for this step, especially as you move toward the end of fine-tuning
Note 2: As always automate what you can
\begin{enumerate}
\item Fine-tune the hyperparameters using cross-validation
\begin{itemize}
\item Treat your data transformation choices as hyperparameters, especially when you are not sure about them (e.g., should I replace missing values with zero or with the median value? Or just drop the rows?)
\item Unless there are very few hyperparameter values to explore, prefer random search over grid search. If training is very long, you may prefer a Bayesian optimisation approach
\end{itemize}
\item Try Ensemble methods. Combining your best models will often perform better than running them individually
\item Once you are confident about your final model, measure its performance on the test set to estimate the generalization error
\end{enumerate}
\item Present Your Solution
\begin{enumerate}
\item Document what you have done
\item Create a nice presentation
\begin{itemize}
\item Make sure you highlight the big picture first
\end{itemize}
\item Explain why your solution achieves the business objective
\item Don’t forget to present interesting points you noticed along the way
\begin{itemize}
\item Describe what worked and what did not
\item List your assumptions and your system’s limitations
\end{itemize}
\item Ensure your key findings are communicated through beautiful visualisations or easy-to-remember statements (e.g., “the median income is the number-one predictor of housing prices”)
\end{enumerate}
\item Launch
\begin{enumerate}
\item Get your solution ready for production (plug into production data inputs, write unit tests, etc.)

\item Write monitoring code to check your system’s live performance at regular intervals and trigger alerts when it drops
\begin{itemize}
\item Beware of slow degradation too: models tend to “rot” as data evolves
\item Measuring performance may require a human pipeline (e.g., via a crowd-sourcing service)
\item Also monitor your inputs’ quality (e.g., a malfunctioning sensor sending random values, or another team’s output becoming stale). This is particularly important for online learning systems
\end{itemize}
\item Retrain your models on a regular basis on fresh data (automate as much as possible)
\end{enumerate}
\end{enumerate}
\end{document}
