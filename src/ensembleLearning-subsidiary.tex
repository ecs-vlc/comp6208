% Created 2024-01-19 Fri 14:01
% Intended LaTeX compiler: pdflatex
\documentclass[11pt]{article}
\usepackage[utf8]{inputenc}
\usepackage[T1]{fontenc}
\usepackage{graphicx}
\usepackage{grffile}
\usepackage{longtable}
\usepackage{wrapfig}
\usepackage{rotating}
\usepackage[normalem]{ulem}
\usepackage{amsmath}
\usepackage{textcomp}
\usepackage{amssymb}
\usepackage{capt-of}
\usepackage{hyperref}
\usepackage{minted}
\usepackage[a4paper,margin=20mm]{geometry}
\usepackage{amsmath}
\usepackage{amsfonts}
\usepackage{stmaryrd}
\usepackage{bm}
\usepackage{minted}
\usemintedstyle{emacs}
\usepackage[T1]{fontenc}
\usepackage[scaled]{beraserif}
\usepackage[scaled]{berasans}
\usepackage[scaled]{beramono}
\newcommand{\tr}{\textsf{T}}
\newcommand{\grad}{\bm{\nabla}}
\newcommand{\av}[2][]{\mathbb{E}_{#1\!}\left[ #2 \right]}
\newcommand{\Prob}[2][]{\mathbb{P}_{#1\!}\left[ #2 \right]}
\newcommand{\logg}[1]{\log\!\left( #1 \right)}
\newcommand{\pred}[1]{\left\llbracket { \small #1} \right\rrbracket}
\newcommand{\e}[1]{{\rm e}^{#1}}
\newcommand{\dd}{\mathrm{d}}
\DeclareMathAlphabet{\mat}{OT1}{cmss}{bx}{n}
\newcommand{\normal}[2]{\mathcal{N}\!\left(#1 \big| #2 \right)}
\newcounter{eqCounter}
\setcounter{eqCounter}{0}
\newcommand{\explanation}{\setcounter{eqCounter}{0}\renewcommand{\labelenumi}{(\arabic{enumi})}}
\newcommand{\eq}[1][=]{\stepcounter{eqCounter}\stackrel{\text{\tiny(\arabic{eqCounter})}}{#1}}
\newcommand{\argmax}{\mathop{\mathrm{argmax}}}
\newcommand{\Dist}[2][Binom]{\mathrm{#1}\left( \strut {#2} \right)}
\author{Adam Prügel-Bennett}
\date{\today}
\title{Advanced Machine Learning Subsidary Notes\\\medskip
\large Lecture 5: Ensemble Methods}
\hypersetup{
 pdfauthor={Adam Prügel-Bennett},
 pdftitle={Advanced Machine Learning Subsidary Notes},
 pdfkeywords={},
 pdfsubject={},
 pdfcreator={Emacs 27.1 (Org mode 9.3)}, 
 pdflang={English}}
\begin{document}

\maketitle

\section{Keywords}
\label{sec:org83d2245}
\begin{itemize}
\item Decision Trees, Averaging, Bagging
\end{itemize}

\section{Main Points}
\label{sec:org0bbc1a3}

\subsection{Decision Trees}
\label{sec:org6c88125}
\begin{itemize}
\item Decision trees are binary tree where a set of data is
partitioned into two at each node of the tree
\item Rule for partition depends on a single feature
\item Trees are grown greedily
\item All possible rules are tried to find the one that maximise the
purity of the leaves
\item Gini, entropy or variance used to measure purity
\item Decision trees can handle categorical or numerical data
\item They can handle missing data
\begin{itemize}
\item after deciding on a new rule (ignoring missing data), a
rule to send the missing data right or left is selected to
maximise purity
\end{itemize}
\item Decision trees are useful for understanding new data sets
\begin{itemize}
\item By examining the rules at the top of the tree we get to see
the most important variables
\end{itemize}
\item Decisions trees are often not competitive
\begin{itemize}
\item they can have high bias because they can only split the data
on single variables (limits their expressiveness)
\item they can also have high variance because a small change to the
data set that changes a rule at the top of the tree can lead
to very different predictions
\end{itemize}
\end{itemize}

\subsection{Ensembling}
\label{sec:orge81e15e}
\begin{itemize}
\item This is used to reduces variance by averaging many different machines
\item This only works if the machines make weakly correlated classifications
\item We ensemble machines by coming to a consensus
\begin{itemize}
\item Vote on a class in classification
\item Average for regression
\end{itemize}
\end{itemize}
\subsubsection{Estimating the Mean}
\label{sec:orgeb7693f}
\begin{itemize}
\item Suppose we want to estimate the mean of a distribution, \(f_{X}(x)\) from
random samples of the distribution.  We assume the distribution
has mean \(\mu = \av{X}\) and variance \(\sigma^2 = \av{(X-\mu)^2}\)
\item Our estimated mean is
\[ \hat{\mu} = \frac{1}{n} \sum_{i=1}^n X_i \]
where \(X_i\sim f_{X}(x)\) (i.e. the \(X_i\) are drawn at random
form the distribution)
\item In expectation
\[ \av{\hat{\mu}} = \frac{1}{n} \sum_{i=1}^n \av{X_i} = \frac{1}{n} \sum_{i=1}^n \mu = \mu\]
\item However, each \(\hat{\mu}\) will differ from the true \(\mu\) (some
will e greater and some less than \(\mu\))
\item We want to measure the variance in \(\hat{\mu}-\mu\)
\[ \hat{\mu}-\mu = \left( \frac{1}{n} \sum_{i=1}^n X_i \right) -\mu = \frac{1}{n} \sum_{i=1}^n (X_i-\mu)  \]
\item Thus
\[ \left( \hat{\mu}-\mu \right)^2 = \left(\frac{1}{n} \sum_{i=1}^n
      (X_i-\mu)\right)^{2} = \left(\frac{1}{n} \sum_{i=1}^n
      (X_i-\mu)\right)\left(\frac{1}{n} \sum_{j=1}^n
      (X_j-\mu)\right) \]
\begin{itemize}
\item Note that \(i\) and \(j\) are just dummy indices (it doesn't
matter what they are called but they are different)
\end{itemize}
\item Or
\[ \left( \hat{\mu}-\mu \right)^2 = \frac{1}{n^2} \sum_{i=1}^n
          \sum_{j=1}^n (X_i-\mu) (X_j-\mu) \]
\begin{itemize}
\item Now there are times when \(i=j\) and times when this isn't true
 so we separate these two out
\[ \left( \hat{\mu}-\mu \right)^2 = \frac{1}{n^2} \sum_{i=1}^n
         (X_i-\mu)^{2} + \sum_{i=1}^n \sum_{j=1\above j\neq i}^n (X_i-\mu) (X_j-\mu)\]
\end{itemize}
\item Taking expectations
\[ \av{\left( \hat{\mu}-\mu \right)^2} = \frac{1}{n^2} \sum_{i=1}^n
         \av{(X_i-\mu)^{2}} + \sum_{i=1}^n \sum_{j=1\above j\neq i}^n \av{(X_i-\mu) (X_j-\mu)}\]
\begin{itemize}
\item where I have used \(\av{A + B} = \av{A} + \av{B}\) (so that
\(\av{\sum_{i} A_i} = \sum_{i} \av{A_i})\)
\end{itemize}
\item Now because \(X_i\) and \(X_j\) are by assumption independently
chosen 
\[ \av{(X_i-\mu) (X_j-\mu)} =  \av{X_i-\mu}\,  \av{X_j-\mu} \]
but \(\av{X_i-\mu} = \av{X_i}-\mu = \mu -\mu = 0\)
so
\[ \av{(X_i-\mu) (X_j-\mu)} =  0\]
and
 \[ \av{\left( \hat{\mu}-\mu \right)^2} = \frac{1}{n^2} \sum_{i=1}^n
         \av{(X_i-\mu)^{2}} = \frac{1}{n^2} \sum_{i=1}^n \sigma^2 = \frac{\sigma^{2}}{n}\]
\end{itemize}
\subsubsection{Bagging}
\label{sec:org5f3f3cc}
\begin{itemize}
\item Uses bootstrap aggregation
\begin{itemize}
\item that is we sample the training set with replacement to obtain
slightly different datasets
\end{itemize}
\item Random Forest
\begin{itemize}
\item Uses decision trees with bootstrap aggregation
\item But also use different random subsets of features
\item Often gives state-of-the-art performance
\end{itemize}
\end{itemize}

\section{Exercises}
\label{sec:orgd24ed90}

\subsection{Compute expected error in mean for correlated variables}
\label{sec:orgb3ec70a}
\begin{itemize}
\item We look at estimating the mean by sampling \(n\) random variables
\begin{equation*}
\hat{\mu} = \frac{1}{n} \sum_{i=1}^n X_i
\end{equation*}
\item In expectation \(\mathbb{E}[X_i]=\mu\)
\item Thus in expectation \(\mathbb{E}[\hat{\mu}]=\mu\)
\item But there will typically be fluctuations from the mean which we can compute using
\begin{equation*}
\sigma^2_{\hat{\mu}} = \mathbb{E}\left[ \left(\strut \hat{\mu} - \mu \right)^2 \right]
\end{equation*}
\item \textbf{Compute this} using
\begin{align*}
\mathbb{E}[(X_i-\mu)^2] &= \sigma^2 & 
   \mathbb{E}[(X_i-\mu)\,(X_j-\mu)] &= \rho\,\sigma^2
\end{align*}
\item Answer given in lecture notes
\item Note that the estimated error in the mean is \(\sigma_{\hat{\mu}\) (this is
what you use when you compute error-bars)
\end{itemize}

\section{Experiments}
\label{sec:org25a6ddb}

\subsection{Visualise a decision tree}
\label{sec:org196b1cf}

\begin{minted}[]{python}
from sklearn.datasets import load_iris
from sklearn import tree
X, y = load_iris(return_X_y=True)
clf = tree.DecisionTreeClassifier()
clf = clf.fit(X, y)

tree.plot_tree(clf.fit(iris.data, iris.target))
\end{minted}
\end{document}
