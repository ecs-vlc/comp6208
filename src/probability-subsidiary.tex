% Created 2024-03-12 Tue 17:22
% Intended LaTeX compiler: pdflatex
\documentclass[11pt]{article}
\usepackage[utf8]{inputenc}
\usepackage[T1]{fontenc}
\usepackage{graphicx}
\usepackage{longtable}
\usepackage{wrapfig}
\usepackage{rotating}
\usepackage[normalem]{ulem}
\usepackage{amsmath}
\usepackage{amssymb}
\usepackage{capt-of}
\usepackage{hyperref}
\usepackage{minted}
\usepackage[a4paper,margin=20mm]{geometry}
\usepackage{amsmath}
\usepackage{amsfonts}
\usepackage{bm}
\usepackage{minted}
\usemintedstyle{emacs}
\usepackage[T1]{fontenc}
\usepackage[scaled]{beraserif}
\usepackage[scaled]{berasans}
\usepackage[scaled]{beramono}
\newcommand{\tr}{\textsf{T}}
\newcommand{\grad}{\bm{\nabla}}
\newcommand{\av}[2][]{\mathbb{E}_{#1\!}\left[ #2 \right]}
\newcommand{\Prob}[2][]{\mathbb{P}_{#1\!}\left[ #2 \right]}
\newcommand{\logg}[1]{\log\!\left( #1 \right)}
\newcommand{\bra}[1]{\left( #1 \right)}
\newcommand{\e}[1]{{\rm e}^{#1}}
\newcommand{\dd}{\mathrm{d}}
\DeclareMathAlphabet{\mat}{OT1}{cmss}{bx}{n}
\newcommand{\normal}[2]{\mathcal{N}\!\left(#1 \big| #2 \right)}
\newcounter{eqCounter}
\setcounter{eqCounter}{0}
\newcommand{\explanation}{\setcounter{eqCounter}{0}\renewcommand{\labelenumi}{(\arabic{enumi})}}
\newcommand{\eq}[1][=]{\stepcounter{eqCounter}\stackrel{\text{\tiny(\arabic{eqCounter})}}{#1}}
\newcommand{\argmax}{\mathop{\mathrm{argmax}}}
\newcommand{\pd}[3][\,]{\frac{\partial^{#1}{#2}}{\partial\,{#3}}}
\author{Adam Prügel-Bennett}
\date{\today}
\title{Advanced Machine Learning Subsidary Notes\\\medskip
\large Lecture 20: Probability}
\hypersetup{
 pdfauthor={Adam Prügel-Bennett},
 pdftitle={Advanced Machine Learning Subsidary Notes},
 pdfkeywords={},
 pdfsubject={},
 pdfcreator={Emacs 29.1 (Org mode 9.6.6)}, 
 pdflang={English}}
\begin{document}

\maketitle

\section{Keywords}
\label{sec:org8fe2874}
\begin{itemize}
\item Probability, Random variable, Expectations
\end{itemize}

\section{Main Points}
\label{sec:orgfe3601a}
\subsection{Basic Language}
\label{sec:org5265ddd}
\begin{itemize}
\item Probability is the process of modelling a world with uncertain outcomes
\item The model is not perfect, but reflects what we care about
\item Mathematicians model the possible outcomes of the world they are
considering in terms of a set of \emph{elementary events}
\item This set is frequently called \(\Omega\) (but you are free to give
it any name)
\item The elementary events are \emph{mutual exclusive} so that if
\(\omega_{i}, \omega_{j} \in \Omega\) then \(\omega_{i} \cap
     \omega_{j} = \emptyset\) (this just means there is no intersection
between elementary events)
\item The set is exhaustive so that \(\bigcup_i \omega_i= \Omega\)
\item This means in our model of the world every possible outcome is
included in \(\Omega\)
\item Now we are often not interested in elementary events (they may be
too small)
\item Instead we consider \textbf{events} \(\mathcal{E}=
      \bigcup\limits_{i\in\mathcal{I}}\omega_i\) where \(\mathcal{I}\) is
some \emph{index set}
\item Suppose we are modelling the probability of a darts match
\item \(\Omega\) might represent every location where a dart might land
\item \(\omega_{i}\in \Omega\) is a particular position
\item I might be interested in the probability of a bull's eye
\(\mathcal{B}= \bigcup\limits_{i\in\mathcal{I}}\omega_i\)
\item In this particular case this is a union of an uncountable
infinity of points
\item In this case I might represent the elementary events by a 2-d vector
\(\bm{x}\)
\item The event of getting a bull's eye corresponds to \(\bm{x}\) being
within the region that defines the bull's eye
\end{itemize}
\subsection{Probability and Random Variables}
\label{sec:org1875892}
\begin{itemize}
\item We can attribute a \textbf{probability},  \(\Prob{\mathcal{E}}\), to an event,
\(\mathcal{E}\)
\item Probabilities are number between 0 and 1: \(0\leq
     \Prob{\mathcal{E}} \leq 1\)
\item 0 means the event never occurs, 1 means the event definitely will occur
\item \(\Prob{\mathcal{E}} + \Prob{\neg\mathcal{E}} = 1\) where
\(\neg\mathcal{E} = \Omega \setminus \mathcal{E}\)
\item That is, the probability of an event occurring and the event not
occurring is 1
\item In some cases we can interpret \(\Prob{\mathcal{E}}\) as the
expected frequency of occurrence of a repetitive trial
\item E.g. If the event, \(\mathcal{E}\) is rolling an (honest) dice and getting
an even number then \(\Prob{\mathcal{E}}=\tfrac{1}{2}\) can be seen
as saying that on average this will happen half the time if you
repeat the experiment often enough
\item However how to we interpret \(\Prob{\text{Pass COMP6208 exam}}\)?
\item Hopefully, this is something you only attempt once
\item We can think of probability as encoding our informed belief that
something will happen
\item When our knowledge changes our probability will change
\item Random variables, \(X\), are numbers we assign to each elementary
event (many events can have the same number assigned)
\item That is we partition the set of outcomes \(\Omega\) and assign a
numbers to each partition
\item E.g. for a dice
\begin{align*}
  X =
  \begin{cases}
    0 & \text{if $\omega\in\{1,3,5\}$}\\
    1 & \text{if $\omega\in\{2,4,6\}$}
  \end{cases}
\end{align*}
\item If we let \(\mathcal{E}_i\) be the event that corresponding to the
partition with value \(x_i\) then \(\Prob{X=x_i} = \Prob{\mathcal{E}_i}\)
\item It is common practice (though not universal) to denote random
variables with capital letters
\item The correspond to events with uncertain outcomes (things that
happen in the future)
\item We distinguish the scalar that we assign to random variables by
writing them in lower case, e.g. \(x_{i}\)
\item When we write \(\Prob{X}\) we can view this as short-hand for
\begin{align*}
  \bra{\Prob{X=x} \mid x \in \mathcal{X}}
  = \bra{\Prob{X=x_1}, \Prob{X=x_2},\ldots\Prob{X=x_n}}
\end{align*}
where \(\mathcal{X}\) is the set of possible values that \(X\)
can take
\item That is, \(\Prob{X}\), is the probability mass function which tells
us the probability of the random variable \(X\) for all possible
values it can take
\item We differentiate between random variables (capitals) and scalars
(lower case) because they behave very differently when we take expectations
\item A function, \(Y=g(X)\), of a random variable \(X\) is a random
variable
\item \textbf{Continuous random variables}
\begin{itemize}
\item When we consider random variables with continuous outcomes (e.g. the
time I go to bed, \(T\)) then there is an uncountable infinity of possible
outcomes and \(\Prob{T=t}=0\)
\item Here we can consider the probability of a random variable,
\(\bm{X}\) occurring in a ball, \(\mathcal{B}(\bm{x}, \epsilon)\),
of radius \(\epsilon\) around the point \(\bm{x}\)
\item Taking the ratio of that probability to the volume of the ball
in the limit \(\epsilon\rightarrow0\) defines a \textbf{probability
density}
\begin{align*}
f_{\bm{X}}(\bm{x}) = \lim_{\epsilon\rightarrow 0}
\frac{\Prob{\bm{X}\in\mathcal{B}(\bm{x},
\epsilon)}}{|\mathcal{B}(\bm{x}, \epsilon)|}
\end{align*}
\item Probability densities are not probabilities (they are positive,
but not necessarily less than 1)
\item Because they are densities, if I change the way I measure
volumes then the numeric values of the density will change
(probability density measured in units of probability
per cm is different to that measured in units of probability
per inch)
\item A consequent is that if we do a change of variables then
often the probability density will change
\item Consider a region \$\mathcal{R}\$---we can describe this using
different coordinate systems \(x\) or \(y= g(x)\)
\begin{align*}
\Prob{X\in\mathcal{R}}
= \int_{\mathcal{R}} f_{X}(x) \, \dd x
=  \Prob{Y\in\mathcal{R}}
= \int_{\mathcal{R}} f_{Y}(y) \, \dd y
\end{align*}
(that is the probability of the event occurring in some region
should not depend on coordinate region we use)
\item Thus \(f_{X}(x)\, |\dd x| = f_{Y}(y) \, |\dd y|\) or
\begin{align*}
 f_{X}(x)  = f_{Y}(y) \, \left| \frac{\dd y}{\dd x}
 \right| = f_{Y}(g(x)) \, | g'(x) |
\end{align*}
\item In high dimensions if we make a change of variables
\(\bm{x} \rightarrow \bm{y}(\bm{x})\) (which can be seen as a change
of random variables \(\bm{X}\rightarrow\bm{Y}(\bm{X})\)) then
\begin{align*}
f_{\bm{X}}(\bm{x}) = f_{\bm{Y}}(\bm{y}) \, |\det(\mat{J})| 
\end{align*}
where \(\mat{J}\) is the Jacobian matrix
\begin{align*}
  \mat{J} =
  \begin{pmatrix}
   \pd{y_1}{x_1} & \pd{y_1}{x_2} & \cdots & \pd{y_1}{x_n} \\
   \pd{y_2}{x_1} & \pd{y_2}{x_2} & \cdots & \pd{y_2}{x_n} \\
   \vdots & \vdots & \ddots & \vdots \\
   \pd{y_n}{x_1} & \pd{y_n}{x_2} & \cdots & \pd{y_n}{x_n}
  \end{pmatrix}
\end{align*}
\item An alternative definition of probability densities in 1-d is
\begin{align*}
f_X(x) = \lim_{\delta x\rightarrow 0}
\frac{\Prob{x \leq X < x + \delta x}}{\delta x}\pause
\end{align*}
\item Thus
$$f_X(x)\,\delta x \approx \Prob{x \leq X < x + \delta x}$$
so that \(f_X(x)\,\delta x\leq 1\) (since this is a probability)
\end{itemize}
\item \textbf{Cumulative Distribution Functions (CDF)}
\begin{align*}
 F_X(x) = \Prob{X\leq x} =
 \begin{cases}
   \sum\limits_{i: x_i\leq x} \Prob{X=x_i} \\
   \int_{-\infty}^x f_X(y) \, \dd y
 \end{cases}
\end{align*}
\begin{itemize}
\item This is a function that goes from 0 to 1 as \(x\) goes from 
\(-\infty\) to \(\infty\)
\item For continuous random variables
\begin{align*}
  f_X(x) = \frac{\dd F_X(x)}{\dd x}
\end{align*}
\item Sometimes it is easy working with the CDF than the PDF
\end{itemize}
\end{itemize}
\subsection{Expectations}
\label{sec:org2fecfd8}
\begin{itemize}
\item Expectations compute the average value of a quantity (it is a number)
\item Mathematicians write expectations as \(\av[\bm{X}]{\cdots}\) (note
that physicist often use \(\langle \cdots \rangle\))
\item We can define the expectation of \(\bm{Y}=g(\bm{X})\) as
\begin{align*}
  \av[\bm{X}]{g(\bm{X})} =
  \begin{cases}
    \displaystyle \sum_{\bm{x}\in\mathcal{X}} g(\bm{x})\, \Prob{\bm{X}=\bm{x}} \\
    \displaystyle \int g(\bm{x})\, f_{\bm{X}}(\bm{x}) \, \dd \bm{x}
  \end{cases}
\end{align*}
\item The expectation of a constant \(c\) is \(c\)
\begin{align*}
  \av[\bm{X}]{c} =
  \begin{cases}
    \displaystyle \sum_{\bm{x}\in\mathcal{X}} c\,
    \Prob{\bm{X}=\bm{x}}
    = c \sum_{\bm{x}\in\mathcal{X}}  \Prob{\bm{X}=\bm{x}}
    = c \\
    \displaystyle \int c\, f_{\bm{X}}(\bm{x}) \, \dd \bm{x}
    = c \int  f_{\bm{X}}(\bm{x}) \, \dd \bm{x} =c 
  \end{cases}
\end{align*}
\item A consequence of this is that \(\av[X]{\av[X]{g(X)}} =
    \av[X]{g(X)}\)
\item Summations and integrals are linear operators
\begin{align*}
  \sum_i \bra{a\,x_i + b\,y_i}
  &= a\,\bra{ \sum_i  x_i} +  b\,\bra{\sum_i y_i} \\
  \int \bra{a\,f(\bm{x}) + b\, g(\bm{x})} \dd \bm{x}
  &=a \, \bra{ \int f(\bm{x})\, \dd \bm{x}} + b \,  \bra{ \int g(\bm{x})\, \dd \bm{x}}
\end{align*}
from which it follows that expectations are linear
\begin{align*}
  \av{a\,X + b\,Y} = a\,\av{X} + b\, \av{Y}
\end{align*}
\item Beware usually \(\av{X\,Y}\neq \av{X}\,\av{Y}\) (unless \(X\) and
\(Y\) are independent)
\item \textbf{Indicator Functions}
\begin{itemize}
\item Indicator functions are functions that take on the value of 0 or 1
\item They are written in different ways.  One nice form is the
Iverson notation
 \begin{align*}
   \pred{\text{\it predicate}} =
   \begin{cases}
     1 & \text{if \textit{predicate }is True}\\
     0 & \text{if \textit{predicate} is False}
   \end{cases}
\end{align*}
\item Sometimes it is written \(\bm{I}_A(x)\) where \(A(x)\) is the predicate
\item Donald Kunth is a champion of the Iverson notation (it is easy
to manipulate)
\item For probabilities
\begin{align*}
  \Prob{\text{\it predicate}} = \av{\pred{\text{\it predicate}}}\pause
\end{align*}
it is not too difficult to convince yourself that this is
correct as you are summing the probabilities where the predicate
is true
\item As an example the CDF is given by
\begin{align*}
  F_X(x) = \Prob{X\leq x} = \av{\pred{X\leq x}}
\end{align*}
\end{itemize}
\end{itemize}
\subsection{Calculus of Probabilities}
\label{sec:org7e9dfc9}
\begin{itemize}
\item When we model the world it is common to consider different random variables
\item In this case we can consider the \textbf{joint probability}
\begin{align*}
  p_{X,Y}(x, y) = \Prob{X=x, Y=y}
\end{align*}
i.e. the probability of the event where both \(X=x\) and \(Y=y\)
\item \(\Prob{X=x,Y=y} = \Prob{Y=y,X=x}\) the order in which you write the
random variables in a joint distribution function doesn't matter
\item There are really only a couple of ways you can manipulate
probabilities (this defines the calculus of probability)
\item The first is known as \emph{marginalisation}.  Given \(\Prob{X,Y}\)
\begin{align*}
  \Prob{X=x} = \sum_{y\in\mathcal{Y}} \Prob{X=x, Y=y}
\end{align*}
where \(\mathcal{Y}\) is the set of values that the random variable
\(Y\) takes
\item Often we will just write
\begin{align*}
  \Prob{X} = \sum_{Y} \Prob{X, Y}
\end{align*}
\item \textbf{Conditional Probability}
\begin{itemize}
\item The other rule of probability involves introducing \textbf{conditional
probabilities}
\item We can also define the probability of an event \(X\) given that
\(Y=y\) has occurred
\begin{align*}
  \Prob{X\mid Y=y} = \frac{\Prob{X,Y=y}}{\Prob{Y=y}}
\end{align*}
\item This is a probability over \(X\).  If we marginalise over \(X\) then
$$ \sum_{x\in \mathcal{X}} \Prob{X=x\mid Y=y}
      =  \frac{\sum_{x\in \mathcal{X}} \Prob{X=x,Y=y}}{\Prob{Y=y}}
      = \frac{\Prob{Y=y}}{\Prob{Y=y}} = 1 $$
\item In constructing a model it is often much easier to specify
conditional probabilities (because you know something) rather
than joint probabilities
\item When manipulating probabilities it is often easier to work
with joint probabilities because we can simplify them by
marginalising out random variables we are not interested in
\item To obtain the joint probability from a conditional probability
we use
\begin{align*}
  \Prob{X,Y} = \Prob{X|Y}\,\Prob{Y} = \Prob{Y|X}\,\Prob{X}\pause
\end{align*}
\item This generalises to more random variables
\begin{align*}
  \Prob{X,Y,Z} = \Prob{X,Y|Z}\, \Prob{Z}
  = \Prob{X|Y,Z}\,\Prob{Y|Z}\,\Prob{Z}
\end{align*}
\item This way of breaking down probabilities in not unique. For
example, we could have written
\begin{align*}
  \Prob{X,Y,Z} = \Prob{Y,Z|X}\, \Prob{X}
  = \Prob{Z|Y,X}\,\Prob{Y|X}\,\Prob{X}\pause
\end{align*}
\item Note that \(\Prob{A,B \mid X,Y}\) means the probability of
random variables \(A\) and \(B\) given that \(X\) and \(Y\) take
particular values
\item \textbf{Beware}: conditional probabilities, \(\Prob{X\mid Y}\) are probabilities
for \(X\), but not \(Y\)
\begin{align*}
  \sum_{x\in\mathcal{X}} \Prob{X=x\mid Y} &= 1 \\
  \sum_{y\in\mathcal{Y}} \Prob{X\mid Y=y} &\neq 1
\end{align*}
(in general)
\item Note that
\begin{align*}
  \av[Y]{\Prob{X\mid Y}} &= \sum_{y\in\mathcal{Y}} \Prob{Y=y}\,
                           \Prob{X|Y=y}\pause \\
  &= \sum_{y\in\mathcal{Y}} \Prob{X,Y=y}\pause = \Prob{X}
\end{align*}
that is taking the expectation of \(\Prob{X\mid Y}\) with respect
to \(Y\) is the same as marginalising the joint probability
\(\Prob{X,Y}\) with respect to \(Y\)
\item Joint probabilities does not imply causality
\item We might believe there is a causal relationship between getting
very cold and catching a cold \(\Prob{\text{Catch cold} \mid
      \text{freezing}} > \Prob{\text{Catch cold}}\)
\item However we can use Bayes' rule (see next lecture) to compute
\(\Prob{\text{freezing}\mid \text{Catch cold}}\)
\item That is, given all I know is that Bob has a cold I can deduce
the probability that he was freezing shortly before catching the
cold. This is a perfectly sensible thing to consider, but
clearly it does not imply that freezing yesterday was caused by
him catching a cold today
\end{itemize}
\item \textbf{Independence}
\begin{itemize}
\item Random variables \(X\) and \(Y\) are said to be \emph{independent} if
\begin{align*}
  \Prob{X, Y} = \Prob{X}\,\Prob{Y}
\end{align*}
\item Because \(\Prob{X,Y}= \Prob{X|Y}\,\Prob{Y}\) and \(\Prob{X,Y}=
      \Prob{Y|X}\,\Prob{X}\) independence implies
\begin{align*}
  \Prob{X|Y} &= \Prob{X} & \Prob{Y|X} = \Prob{Y}
\end{align*}
\item Probabilistic independence only implies a mathematical co-incident not
necessarily causal independence.  There are examples of events
that are causally dependent but nevertheless statistical
independent (although such cases are rare)
\item However causal independence always implies probabilistic independence
\item If \(X\in\{0,1\}\) represents the outcome of tossing a coin and
\(Y\in\{1,2,3,4,5,6\}\) the outcome of rolling a dice then \(X\) and \(Y\)
are independent
\item In well conducted experiments we expect the results we obtain
are independent (that is the outcome of one experiment should
not affect the outcome of another experiment)
\item Let \(\mathcal{D} = (X_1, X_2, \ldots, X_m)\) represents
possible outcomes from a set of \(m\) well conducted
experiments then
\begin{align*}
  \Prob{\mathcal{D}} = \prod_{i=1}^m \Prob{X_i}
\end{align*}
\item Denoting a possible sentence I might say by
\(\mathcal{S}=(W_1, W_2,\ldots,W_m)\) then
\begin{align*}
  \Prob{\mathcal{S}} \neq  \prod_{i=1}^m \Prob{W_i}
\end{align*}
\item If this is not the case they you expect my sentences to be
equally likely irrespective of the order of my words (I
certainly home this is not the case)
\item Independence is a very useful condition that substantially
simplifies probabilistic calculations
\item Independence is a strong condition that does not happen in many
models. However, a far more likely condition is \textbf{conditional
independence}.  For example, \(X\) and \(Y\) are conditionally
independent given \(Z\) if
$$ \Prob{X, Y |Z} = \Prob{X|Z} \, \Prob{Y|Z}$$
\item Me working today, \(W\), and me getting stuck in a traffic jam,
\(J\), are not statistically independent events, but they are
statistically independent (at least in a simple of model of the
world) given the day of the week, \(D\)
$$ \Prob{W, J |D} = \Prob{W|D} \, \Prob{J|D}$$
\end{itemize}
\end{itemize}



\section{Exercises}
\label{sec:org88f5b62}
\begin{enumerate}
\item Suppose I make an investment of capital \(C_{0}\) and each day I
get a random return \(R_{t}\) such that \(C_{t} =
     (1+r_{t})\,C_{t-1}\). Taking logs
$$ \logg{C_{t}} = \log(1 + R_{t}) + \log{C_{t-1}} \approx
     R_{r}+\logg{C_{t-1}}$$
where we assume \(|R_{t}| \ll 1\).  Rearranging \(\logg{C_{t}} -
     \logg{C_{t-1}} \approx R_{t}\) and summing
\begin{align*}
\sum_{t=1}^{T} \bra{\logg{C_{t}} - \logg{C_{t-1}}} =
\logg{C_{T}} - \logg{C_{0}} = \sum_{{t=1}}^{T} R_{t}
\end{align*}
Now \(\logg{C_{T}} - \logg{C_{0}} = \logg{C_{T}/C_{0}} =
     \logg{G_{T}}\) where \(G_{T}=C_{T}/C_{0}\) is the multiplicative
gain (or loss) in the investment. Let \(L_{T} = \logg{G_{T}}\).
Assuming we live in a random world so that \(R_{t} \sim
     \mathcal{N}(0,\epsilon)\). That is, the return each day is
normally distributed with mean 0 and variance \(\epsilon\).
Furthermore, let us assume that \(R_{t}\) is independent of
\(R_{t'}\) then \(L_{T}\) is the sum of \(T\) normal independent
variables so that \(L_{T}\sim \mathcal{N}(0, \sqrt{T}\,\epsilon)\).
That is, \(f_{L}(\ell) = \mathcal{N}(\ell\mid 0,
     \sqrt{T}\,\epsilon)\). Compute the PDF of \(G_{T}\) using
$$f_{G}(g) = f_{L}(\ell) \, \frac{\dd \ell}{\dd g}$$
where \(\ell = \log(g)\)
\end{enumerate}

\section{Answers}
\label{sec:org9647c37}
\begin{enumerate}
\item This may seem a crazy model, but it is very commonly used in
finance and models stock prices extremely well.  I'm not going to
give the answer, but the distribution is a log-normal
distribution which you can look up on Wikipedia.  It occurs
frequently when a random variable is equal to a product of random
variables (taking the logarithm and using the central limit
theorem---i.e. the sum of many random variables is approximately
normally distributed---gives rise to a log-normal distribution).
\end{enumerate}
\end{document}