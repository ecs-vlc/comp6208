% Created 2024-01-29 Mon 11:45
% Intended LaTeX compiler: pdflatex
\documentclass[11pt]{article}
\usepackage[utf8]{inputenc}
\usepackage[T1]{fontenc}
\usepackage{graphicx}
\usepackage{grffile}
\usepackage{longtable}
\usepackage{wrapfig}
\usepackage{rotating}
\usepackage[normalem]{ulem}
\usepackage{amsmath}
\usepackage{textcomp}
\usepackage{amssymb}
\usepackage{capt-of}
\usepackage{hyperref}
\usepackage{minted}
\usepackage[a4paper,margin=20mm]{geometry}
\usepackage{amsmath}
\usepackage{amsfonts}
\usepackage{bm}
\usepackage{minted}
\usemintedstyle{emacs}
\usepackage[T1]{fontenc}
\usepackage[scaled]{beraserif}
\usepackage[scaled]{berasans}
\usepackage[scaled]{beramono}
\newcommand{\tr}{\textsf{T}}
\newcommand{\grad}{\bm{\nabla}}
\newcommand{\av}[2][]{\mathbb{E}_{#1\!}\left[ #2 \right]}
\newcommand{\Prob}[2][]{\mathbb{P}_{#1\!}\left[ #2 \right]}
\newcommand{\logg}[1]{\log\!\left( #1 \right)}
\newcommand{\e}[1]{{\rm e}^{#1}}
\newcommand{\dd}{\mathrm{d}}
\DeclareMathAlphabet{\mat}{OT1}{cmss}{bx}{n}
\newcommand{\normal}[2]{\mathcal{N}\!\left(#1 \big| #2 \right)}
\newcounter{eqCounter}
\setcounter{eqCounter}{0}
\newcommand{\explanation}{\setcounter{eqCounter}{0}\renewcommand{\labelenumi}{(\arabic{enumi})}}
\newcommand{\eq}[1][=]{\stepcounter{eqCounter}\stackrel{\text{\tiny(\arabic{eqCounter})}}{#1}}
\newcommand{\argmax}{\mathop{\mathrm{argmax}}}
\author{Adam Prügel-Bennett}
\date{\today}
\title{Advanced Machine Learning Subsidary Notes\\\medskip
\large Lecture 4: Symmetry}
\hypersetup{
 pdfauthor={Adam Prügel-Bennett},
 pdftitle={Advanced Machine Learning Subsidary Notes},
 pdfkeywords={},
 pdfsubject={},
 pdfcreator={Emacs 27.1 (Org mode 9.3)}, 
 pdflang={English}}
\begin{document}

\maketitle

\section{Keywords}
\label{sec:orgf197648}
\begin{itemize}
\item Inductive Bias, Symmetry, Invariance, Group theory
\end{itemize}

\section{Main Points}
\label{sec:org26c9e2b}

\subsection{Inductive Bias}
\label{sec:org0e30ab9}
\begin{itemize}
\item Learning machines are not universal approximators, but rather have a propensity
to learn a specific set of rules
\item This is known as their \emph{inductive bias}
\item We want to choose the learning machine so the inductive bias reflects the properties
of the problem
\item In almost all problems we use some smoothness property to select slowly varying 
functions
\end{itemize}
\subsection{Invariances}
\label{sec:org85831ee}
\begin{itemize}
\item For many problems there are symmetries that you want your learning machine to 
respect
\item For example in image classification we consider the class of an object to be
invariant to its position in an image
\item Learning machines whose outputs are invariant to translations have a much smaller
set of functions to learn and therefore are much less likely to overfit
\item CNNs are equivariant in that if we translate the input the the output will be translated
in the same way
\item Building classifies using layers of CNN builds in a lot of translational invariance
\item Ideally we want a classifier system to also be scale and rotation invariant, although we
don't know how to this well
\item Sometimes we use pyramid systems where we rescale the image and run the same
operations on all scales
\item However, in my datasets the foreground object is rescaled to fill most of the image
with smaller object treated as the background so scale invariance is not necessarily
required
\item Often we try to learn invariance by augmenting the training set by artificially
transforming the image (e.g. by reflecting, cropping, shearing and rescaling an image)
\item Data augmentation usually gives a very significant improvement in performance
\item We can sometimes achieve invariance by suitably normalising our data.  Again when
done right this can give a significant improvement in performance
\end{itemize}
\subsection{Group Theory}
\label{sec:org8927e29}
\begin{itemize}
\item The mathematical language of symmetry is group theory
\item You should know group theory exists, but I'm not going to examine you on details
of group theory
\item This considers a set of objects (we can think of these as transformations)
\item Together with a binary operation we can think of as composition
\item That is if \(a,b\in\mathcal{G}\) are two transformations the \(a\cdot b\) can be 
interpreted as applying transformation \(b\) followed by transformation \(a\)
\item For the set, \(\matcal{G}\), and binary operation, "\(\cdot\)" to form a group they must satisfy four conditions
\begin{enumerate}
\item Closure: if \(a,b\in\mathcal{G}\) the \(a\cdot b \in \mathcal{G}\)
\item Associativity: for all \(a,b,c\in\mathcal{G}\), we have \((a\cdot b)\cdot c = a \cdot (b\cdot c)\)
\item Existence of an identity element where \(e\cdot a =a\)
\item Every element, \(a\in\mathcal{G}\) has an inverse such that \(a^{-1} a = e\)
\end{enumerate}
\item (Group theory doesn't really care if the elements of the group are transformations,
just that the axioms are satisfied)
\item An example is the cyclic group \(C_{2}\)
\begin{itemize}
\item We can think of this as tossing a coin with two actions
\begin{itemize}
\item flip the coin: \(f\)
\item leave the coin alone: \(e\)
\end{itemize}
\item These form a group as
\begin{enumerate}
\item Closure: \(f\cdot e=f\), \(f\cdot f = e\), \(e,\cdot e=e\) and \(e \cdot f =f\)
(if you don't believe this get out a coin and try.  The first identity is don't do anything
then flip the coin is the same as just flipping the coin)
\item Associatity; this follows automatically from the operation being a composition
(we will see this in innerProduct lecture where we look at matrix multiplication).
You can also check rather boringly, e,g, \((e\cdot f)\cdot f =f\cdot f = e\) while
\(e\cdot(f\cdot f) = e \cdot e = e\) thus \((e\cdot f)\cdot f =e\cdot(f\cdot f)\).  You can
try this for all triples
\item \(e\) is clearly the identity
\item e\textsuperscript{-1}=e and \(f^{-1}=f\)
\end{enumerate}
\item The cyclic group \(C_{3}\) corresponds to the rotations of an equilateral triangle maps
the triangle to itself.  While \(C_{n}\) are the rotations that keep an \(n\) sided polygon
with identical sides onto itself
\item The \emph{four group} is a second group of size 4 (besides \(C_{4}\)) that can be thought of
as describing the flipping around a vertical and horizontal axis a rectangle, or rotating
it by 180 degress around its centre of mass.  We can describe the group in terms of
a \emph{multiplication table} where the first row and first column corresponds to the elements
\(a_{i}\) and the \(i^{th}\) row and \(j^{th}\) column corresponds to
the multiplication \(a_{j}\cdot a_{i}\)
\begin{center}
\begin{tabular}{l|llll}
\hline
 & e & v & h & r\\
\hline
e & e & v & h & r\\
v & v & e & r & h\\
h & h & r & e & v\\
r & r & h & v & e\\
\hline
\end{tabular}
\end{center}
\begin{itemize}
\item This is an "Abelian group" meaning that for any elements \(a, b\) we have \(a\cdot b = b\cdot a\)
\end{itemize}
\item The group of permutations of \(n\) objects forms a group of size \(n!\) called the
symmetric group \(S_{n}\)
\begin{itemize}
\item This is a group as the composition of a permutation is a permutation
\item There is an identity (doing nothing)
\item For every element there is an inverse permutation (which is a permutation)
\item For \(n>2\) (where \(S_{2}=C_{2}\)) the group is non-Abelian meaning that there
are permutations \(p_{i}\) and \(p_{j}\) where \(p_{1}\neq p_{2}\)
\item Despite this composing permutations is associative
\item Permutations have a lot of structure and \(S_{n}\) is very well studied
\end{itemize}
\item There exists groups with an infinite number of elements.  The set of integers with the
operation of addition form a group with identity, 0, and the inverse of \(n\) is \(-n\)
\begin{itemize}
\item You can think of these describing an infinite set of discrete shifts
\item You first started studying this group when you first learn to count
\end{itemize}
\item The set of rational numbers under addition also forms a group
\item The reals under addition form a group describing any one dimensional translation
\item The set of reals excluding 0 forms a group under multiplication with an identity 1 
and where the inverse of \(x\) is \(1/x\)
\begin{itemize}
\item These can be thought of as describing scale transformations
\end{itemize}
\item Groups can be interpreted in different ways with their elements having different names
(e.g. )one, two, three, \(\cdots{}\) or un, deux, trois, \(\cdots{}\)).  The permutation of two elements
has the same structure (is isomorphic to) flipping a coin.  What counts is the structure of
the multiplication time (up to re-ordering the elements)
\end{itemize}
\end{itemize}
\subsubsection{Lie Groups}
\label{sec:org9614cdd}
\begin{itemize}
\item When we consider continuous transformation in high dimensional space then we find a
set of very interesting groups known as Lie groups
\item These describe things like general translational invariance, rotational invariance,
relativistic invariance
\item The set on transformations can be represented as matrices, with the composition operator
being matrix multiplication
\item Different groups correspond to different sets of matrices.
\item A well known group is the set of \(n\times n\) orthogonal matrices \(O(n)\)
\item By definition of orthogonal matrices \(\mat{M}^\tr \mat{M} = \mat{M}\, \mat{M}^\tr = \mat{I}\)
\item These correspond to rotations and reflections
\item Note that, in general,  \(\det(\mat{M}^{\tr}) = \det(\mat{M})\) and
\(\det(\mat{A}\,\mat{B}) = \det(\mat{A}) \times \det{\mat{B}}\)
\item As a consequence \(\det(\mat{M}^\tr \mat{M} ) = \det(\mat{I}) =1\) so
that \(\det(\mat{M})^{2}=1\) or \(\det(\mat{M}) = \pm 1\)
\item If \(\det(\mat{M}) = -1\) then the  transformation involves a reflection
\item The special orthogonal group, \(SO(n)\), corresponds to a subgroup
of matrices in \(O(2)\) with \(\det(\mat{M})=1\).  These correspond
to rotations only
\item The group \(SO(2)\) corresponds to 2-d rotations where the
elements of the group are matrices
 \begin{align*}
 \begin{pmatrix}
    \cos(\theta) & \sin(\theta) \\
   -\sin(\theta) & \cos(\theta)\pause
 \end{pmatrix}
\end{align*}
\end{itemize}
\end{document}
