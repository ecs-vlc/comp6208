%Master File:lectures.tex

\lesson{Constrained Optimisation}
\vspace{-1cm}
\begin{center}
  \includegraphics[width=0.5\linewidth]{ineqcon}
\end{center}
\keywords{Lagrangians, Inequalities, KKT, Linear Programming,
  Quadratic Programming, Duality}
%%%%%%%%%%%%%%%%%%%%%%% Next Slide %%%%%%%%%%%%%%%%%%%%%%%
\renewcommand{\Outline}{%
\begin{slide}
\section[1]{Outline}

\begin{minipage}{12cm}\raggedright
  \begin{enumerate}\squeeze
    \outlineitem{Constrained Optimisation}{constrained}
    \outlineitem{Inequalities}{inequalities}
    \outlineitem{Duality}{duality}
  \end{enumerate}
\end{minipage}\hfill
\begin{minipage}{10cm}
  \includegraphics[width=10cm]{ineqcon}
\end{minipage}
\end{slide}
\addtocounter{outlineitem}{1}
}

\setcounter{outlineitem}{1}

%%%%%%%%%%%%%%%%%%%%%%% Next Slide %%%%%%%%%%%%%%%%%%%%%%%
\Outline % Constrained Optimisation
\toptarget{firstoutline}
%%%%%%%%%%%%%%%%%%%%%%% Next Slide %%%%%%%%%%%%%%%%%%%%%%%

\begin{slide}
\section{Optimisation with Constraints}

\begin{PauseHighLight}
  \begin{itemize}
  \item There are a number of important applications where we wish to
    minimise an objective function subject to inequality
    constraints\pause
  \item A prominent example of this is support vector machines\pause
  \item More generally there are a large number of kernel models that
    involve constraints\pause
  \item However, constraints are ubiquitous in machine learning (e.g.{} in
    Wasserstein GANs)\pause
  \end{itemize}
\end{PauseHighLight}


\end{slide}

%%%%%%%%%%%%%%%%%%%%%%% Next Slide %%%%%%%%%%%%%%%%%%%%%%%

\begin{slide}
\section[-1]{Solving Constrained Optimisation Problems}

\begin{PauseHighLight}
  \begin{itemize}
  \item Suppose we have a problem
    \begin{align*}
      \min_{\bm{x}} f(\bm{x}) \quad \mbox{subject to $g(\bm{x})=0$}\pause
    \end{align*}
  \item A standard procedure is to define the Lagrangian
    \begin{align*}
      \mathcal{L}(\bm{x},\alpha) = f(\bm{x}) - \alpha\, g(\bm{x})
    \end{align*}
    where $\alpha$ is known as a Lagrange multiplier\pause
  \item In the extended space $(\bm{x},\alpha)$ we have to solve
    \begin{align*}
      \max_{\alpha} \, \min_{\bm{x}} \mathcal{L}(\bm{x},\alpha)\pause
    \end{align*}
  \end{itemize}
\end{PauseHighLight}

\end{slide}

%%%%%%%%%%%%%%%%%%%%%%% Next Slide %%%%%%%%%%%%%%%%%%%%%%%

\begin{slide}
\section[-2]{Conditions on Optimum}

\begin{PauseHighLight}
  \begin{itemize}
  \item The optimisation problem is
    \begin{align*}
      \max_{\alpha} \, \min_{\bm{x}} \mathcal{L}(\bm{x},\alpha)
      \quad \mbox{where} \quad \mathcal{L}(\bm{x},\alpha) = f(\bm{x}) - \alpha\,
      g(\bm{x}) \pause
    \end{align*}
  \item Assuming differentiability
    \begin{align*}
      \gradx \mathcal{L}(\bm{x},\alpha) &= \gradx f(\bm{x}) - \alpha\,\gradx g(\bm{x}) =0\\
      \pd{\mathcal{L}}{\alpha} &=  -g(\bm{x}) =0\pause
    \end{align*}
  \item The second condition is just the constraint\pause
  \item But what about first condition: $\gradx f(\bm{x}) = \alpha\,\gradx g(\bm{x})$?\pauseb
  \end{itemize}
\end{PauseHighLight}

\end{slide}


%%%%%%%%%%%%%%%%%%%%%%% Next Slide %%%%%%%%%%%%%%%%%%%%%%%

\begin{slide}
\section{Note on Gradients}

\begin{PauseHighLight}
  \begin{itemize}
  \item Note that for any function $f(\bm{x})$ we can Taylor expand
    around $\bm{x}_0$
    \begin{align*}
      f(\bm{x}) = f(\bm{x}_0) + (\bm{x}-\bm{x}_0)^\tr \gradx f(\bm{x}_0) + 
      \frac{1}{2}(\bm{x}-\bm{x}_0)^\tr \mat{H} (\bm{x}-\bm{x}_0) + \ldots
    \end{align*}
    where $\mat{H}$ is a matrix of second derivative known as the
    Hessian\pause
  \item If we consider the set of points perpendicular to $\gradx f(\bm{x}_0)$ which
    go through $\bm{x}_0$ (the tangent plane), these will have values
    \begin{align*}
      f(\bm{x}) &= f(\bm{x}_0) + O(\|\bm{x}-\bm{x}_0\|^2) & &
      \includegraphics[width=10cm]{perp}
    \end{align*}
    thus $\gradx f(\bm{x})$ is always orthogonal to the contour lines\pause
  \end{itemize}
\end{PauseHighLight}

\end{slide}


%%%%%%%%%%%%%%%%%%%%%%% Next Slide %%%%%%%%%%%%%%%%%%%%%%%

\begin{slide}
\section[-2]{Constrained Optima}
\pb
\pause
\begin{center}
  \includegraphics[width=0.8\linewidth]{constrainedopt0}\mypl{1}
  \multido{\ia=1+1,\ib=2+1}{2}{%
    \llap{\includegraphics[width=0.8\linewidth]{constrainedopt\ia}}\mypl{\ib}}
\end{center}
\end{slide}

%%%%%%%%%%%%%%%%%%%%%%% Next Slide %%%%%%%%%%%%%%%%%%%%%%%

\begin{slide}
\section[-2]{Example}

\begin{PauseHighLight}
  \begin{itemize}
  \item Minimise $f(\bm{x}) = x^2 + 2\,y^2 - x\,y$
  \item Subject to $g(\bm{x}) = x - 2\,y -3 = 0$\pause
  \item Writing $\mathcal{L} = f(\bm{x}) - \alpha\,g(\bm{x})$\pause
  \item Condition for minima is $\gradx \mathcal{L} = 0$
    \begin{align*}
      \gradx f(\bm{x}) = 
      \begin{pmatrix}
        2\,x -y \\ -x + 4\,y
      \end{pmatrix} =
      \alpha \gradx g(\bm{x}) = \alpha
      \begin{pmatrix}
        1 \\ -2
      \end{pmatrix}
    \end{align*}
    and $\pd{\mathcal{L}}{\alpha} = -g(\bm{x}) = -x + 2\,y +3 = 0$\pause
  \item Solving simultaneous equations gives minima at $(x,y)=(\frac{3}{4}, -
  \frac{9}{8})$ with $\alpha= \frac{21}{8}$\pause
  \end{itemize}
\end{PauseHighLight}

\end{slide}

%%%%%%%%%%%%%%%%%%%%%%% Next Slide %%%%%%%%%%%%%%%%%%%%%%%

\begin{slide}
\section{Surface}

\begin{center}
  \includegraphics[width=\linewidth]{lagrange3d}
\end{center}

\end{slide}

%%%%%%%%%%%%%%%%%%%%%%% Next Slide %%%%%%%%%%%%%%%%%%%%%%%

\begin{slide}
\section{Saddle-Point $y=-9/8$}

\begin{center}
  \includegraphics[width=0.7\linewidth]{lagrange3d-1}
\end{center}

\end{slide}



%%%%%%%%%%%%%%%%%%%%%%% Next Slide %%%%%%%%%%%%%%%%%%%%%%%

\begin{slide}
\section[-2.4]{Multiple Constraints}

\begin{PauseHighLight}
  \begin{itemize}
  \item Given an optimisation problem with multiple constraints
    \begin{align*}
      \min_{\bm{x}} f(\bm{x}) \quad \mbox{subject to $g_k(\bm{x})=0$ for
      $k=1,\,2,\,\ldots,\,m$}\pause
    \end{align*}
  \item We introduce multiple Lagrange multipliers
    \begin{align*}
      \mathcal{L}(\bm{x},\bm{\alpha}) 
      = f(\bm{x}) - \sum_{k=1}^m \alpha_k\, g_k(\bm{x})\pause
    \end{align*}
  \item The condition for an optima is $\gradx
    \mathcal{L}(\bm{x},\bm{\alpha}) =0$ which implies
    \begin{align*}
      \gradx f(\bm{x}) = \sum_{k=1}^m  \alpha_k \, \gradx g_k(\bm{x})\pause
    \end{align*}
    plus the original constraints
    $\pd{\mathcal{L}(\bm{x},\bm{\alpha})}{\alpha_k}= -g_k(\bm{x})=0$\pause
  \end{itemize}
\end{PauseHighLight}

\end{slide}

%%%%%%%%%%%%%%%%%%%%%%% Next Slide %%%%%%%%%%%%%%%%%%%%%%%

\begin{slide}
\section[-2]{Example}

\begin{PauseHighLight}
  \begin{itemize}
  \item Minimise $f(\bm{x}) = x^2 + 2\,y^2 + 5\,z^2- x\,y -x\,z$ subject
  to $g_1(\bm{x}) = x - 2\,y -z -3 = 0$ and $g_2(\bm{x}) = 
  2\,x + 3\,y +z - 2 = 0$ \pause
  \item Writing $\mathcal{L}(\bm{x},\alpha) = f(\bm{x}) - \alpha_1\,g_1(\bm{x}) 
    - \alpha_2\,g_2(\bm{x})$\pause
  \item Condition for minima is $\gradx \mathcal{L} = 0$ or $\gradx
  f(\bm{x}) = \sum_{k=1}^2 \alpha_k \gradx g_k(\bm{x})$
    \begin{align*}
      \begin{pmatrix}
        2\,x -y -z\\ -x + 4\,y \\ 10\,z -x
      \end{pmatrix}
      = \alpha_1 
      \begin{pmatrix}
        1 \\ -2 \\ -1
      \end{pmatrix}
      + \alpha_2
      \begin{pmatrix}
        2 \\ 3 \\ 1
      \end{pmatrix}
    \end{align*}
    and $\pd{\mathcal{L}}{\alpha_i} = -g_i(\bm{x}) = 0$\pause
  \item Solving simultaneous equations gives minima at $(\frac{37}{20}, -
  \frac{11}{20}, -\frac{1}{20})$ with $\alpha_1=3$ and
  $\alpha_2=\frac{13}{20}$\pause
  \end{itemize}
\end{PauseHighLight}

\end{slide}

%%%%%%%%%%%%%%%%%%%%%%% Next Slide %%%%%%%%%%%%%%%%%%%%%%%
\Outline % Inequality Constraints
%%%%%%%%%%%%%%%%%%%%%%% Next Slide %%%%%%%%%%%%%%%%%%%%%%%

\begin{slide}
\section{Inequality Constraints}

\begin{PauseHighLight}
  \begin{itemize}
  \item Suppose we have the problem
    \begin{align*}
      \min_{\bm{x}} f(\bm{x}) \quad \mbox{subject to $g(\bm{x})\geq 0$}\pause
    \end{align*}
  \item Looks much more complicated, but\pause
  \item Only two things can happen
    \begin{itemize}
    \item Either a minimum, $\bm{x}^*$, of $f(\bm{x})$ satisfies
    $g(\bm{x}^*)>0$ 
      \begin{itemize}
      \item We then have an unconstrained optimisation problem\pause
      \end{itemize}
    \item Otherwise, it satisfies  $g(\bm{x}^*)=0$
      \begin{itemize}
      \item We have a constrained optimisation problem\pause
      \end{itemize}
    \end{itemize}
  \end{itemize}
\end{PauseHighLight}

\end{slide}


%%%%%%%%%%%%%%%%%%%%%%% Next Slide %%%%%%%%%%%%%%%%%%%%%%%

\begin{slide}
\section[-2]{Inside Region}

\begin{center}
  \includegraphics[width=0.8\linewidth]{ineqcon1}
\end{center}
\end{slide}

%%%%%%%%%%%%%%%%%%%%%%% Next Slide %%%%%%%%%%%%%%%%%%%%%%%

\begin{slide}
\section[-2]{On the Boundary}

\begin{center}
  \includegraphics[width=0.8\linewidth]{ineqcon}
\end{center}
\end{slide}

%%%%%%%%%%%%%%%%%%%%%%% Next Slide %%%%%%%%%%%%%%%%%%%%%%%

\begin{slide}
\section[-1]{KKT Conditions}

\begin{PauseHighLight}
  \begin{itemize}
  \item To minimise $f(\bm{x})$ subject to $g(\bm{x})\geq0$
    \begin{align*}
      \mathcal{L}(\bm{x},\alpha) =  f(\bm{x}) - \alpha\, g(\bm{x})\pause
    \end{align*}
  \item Then $\gradx \mathcal{L}=0$ or
    \begin{align*}
      \gradx \mathcal{L} = \gradx f(\bm{x}) - \alpha\,\gradx g(\bm{x}) =0\pause
    \end{align*}
  \item where either
    \begin{itemize}
    \item $\alpha = 0$ and the solutions in the interior\pause\ or
    \item $\alpha>0$ and $g(\bm{x})=0$, i.e. the solution is on the
      boundary\pause
    \end{itemize}
  \item These conditions are known as the Karush-Kuhn-Tucker
    conditions\pause
  \end{itemize}
\end{PauseHighLight}

\end{slide}

%%%%%%%%%%%%%%%%%%%%%%% Next Slide %%%%%%%%%%%%%%%%%%%%%%%

\begin{slide}
\section[-2]{Many Inequalities}

\begin{PauseHighLight}
  \begin{itemize}
  \item Given the problem
    \begin{align*}
      \min_{\bm{x}} f(\bm{x}) \quad \mbox{subject to $g_k(\bm{x})\geq 0$ for
      $k=1,\,2,\,\ldots,\,m$}\pause
    \end{align*}
  \item We introduce multiple Lagrange multipliers
    \begin{align*}
      \mathcal{L}(\bm{x},\alpha) = f(\bm{x}) - \sum_{k=1}^m \alpha_k\,
      g_k(\bm{x})\pause 
    \end{align*}
  \item The condition for an optima is
    \begin{align*}
      \gradx f(\bm{x}) = \sum_{k=1}^m  \alpha_k \, \gradx g_k(\bm{x})\pause
    \end{align*}
  \item Plus the constraints that either $\alpha_k=0$ or $\alpha_k>0$ and
    $g_k(\bm{x})=0$\pause
  \end{itemize}
\end{PauseHighLight}

\end{slide}


%%%%%%%%%%%%%%%%%%%%%%% Next Slide %%%%%%%%%%%%%%%%%%%%%%%
\Outline %  Duality
%%%%%%%%%%%%%%%%%%%%%%% Next Slide %%%%%%%%%%%%%%%%%%%%%%%

\begin{slide}
\section{Solving the Lagrangian for $\bm{x}$}

\begin{PauseHighLight}
  \begin{itemize}
  \item Consider minimising a function $f(\bm{x})$ subject to a set of
    constraints $g_i(\bm{x})=0$ or $g_i(\bm{x})\leq 0$\pause
  \item We can consider this a double optimisation problem
    \begin{align*}
      \max_\alpha \, \min_{\bm{x}} \mathcal{L}(\bm{x},\bm{\alpha})
      = \max_\alpha \, \min_{\bm{x}} \left( f(\bm{x}) + \sum_i \alpha_i
      g_i(\bm{x}) \right)
    \end{align*}
    where there would be constraints on $\alpha_i$ if we had an
    inequality constraint\pause
  \end{itemize}
\end{PauseHighLight}

\end{slide}

%%%%%%%%%%%%%%%%%%%%%%% Next Slide %%%%%%%%%%%%%%%%%%%%%%%

\begin{slide}
\section{Dual Problem}

\begin{PauseHighLight}
  \begin{itemize}
  \item If $f(\bm{x})$ and $g_i(\bm{x})$ are simple we can sometimes
    find a set of variables $\bm{x}^*(\bm{\alpha})$ that minimises the
    Lagrangian
    \begin{align*}
      \grad_{\bm{x}} \mathcal{L}(\bm{x}^*(\bm{\alpha}),\bm{\alpha}) = 0\pause
    \end{align*}
  \item This leaves us with the \emph{dual problem}
    \begin{align*}
      \max_\alpha \mathcal{L}(\bm{x}^*(\bm{\alpha}),\bm{\alpha})\pause
    \end{align*}
  \item If we had an inequality constraint $g_i(\bm{x})\geq 0$ then we
    would have the additional constraint in the dual problem
    $\alpha_i\geq0$\pause
  \end{itemize}
\end{PauseHighLight}

\end{slide}

%%%%%%%%%%%%%%%%%%%%%%% Next Slide %%%%%%%%%%%%%%%%%%%%%%%

\begin{slide}
\section[-2]{Linear Programming}

\begin{PauseHighLight}
  \begin{itemize}
  \item In linear programming we minimise a linear objective function
    $\bm{c}^\tr \bm{x}$ subject to linear constraints $\bm{g}(\bm{x}) =
    \mat{M}\, \bm{x} - \bm{b} =0$ (or $\bm{g}(\bm{x})\geq0$)\pause
  \item The Lagrangian becomes
    \begin{align*}
      \mathcal{L}(\bm{x},\bm{\alpha}) = \bm{c}^\tr \bm{x} - \bm{\alpha}^\tr
      \left( \mat{M}\, \bm{x} - \bm{b} \right) \pause
    \end{align*}
  \item An equivalent way of writing the Lagrangian is
    \begin{align*}
      \mathcal{L}(\bm{x},\bm{\alpha}) 
      = \bm{b}^\tr\bm{\alpha} - \bm{x}^\tr\left( \mat{M}^\tr\, \bm{\alpha}
      - \bm{c}\right)\pause
    \end{align*}
  \item An entirely equivalent interpretation is that we maximise an
    objective function $\bm{b}^\tr\bm{\alpha}$ subject to constraints
    $\mat{M}^\tr \bm{\alpha} - \bm{c} = 0$ (or $\mat{M}^\tr
    \bm{\alpha} - \bm{c} \leq 0$)\pause
  \end{itemize}
\end{PauseHighLight}

\end{slide}


%%%%%%%%%%%%%%%%%%%%%%% Next Slide %%%%%%%%%%%%%%%%%%%%%%%

\begin{slide}
\section{Linear Programming Example}

\begin{PauseHighLight}
  \begin{itemize}
  \item Suppose we eat potatoes and rice and we want to ensure that we
    get enough vitamin A and C\pause
    \begin{center}
      \begin{tabular}{|r|c|c|c|}\hline
     &Potatoes& Rice& Daily Requirement \\ \hline
        Vitamin A & 3 & 5 & 20 \\ \hline
        Vitamin C & 5 & 2 & 24 \\ \hline
        Price & 5 & 4 & \\ \hline
      \end{tabular}\pause
    \end{center}
  \item We want to buy $P$ kg potatoes and $R$ kg of rice as cheaply as
    possible subject to fulfilling our vitamin requirement\pause
    \begin{align*}
      \min_{P,R} 5\,P + 4\,R\hspace{8cm} \\
      \text{subject to} \quad P,R\geq0, \quad 3 P + 5 R \geq 20 \quad \text{and} \quad
      5 P + 2 R \geq 24\pause
    \end{align*}
  \end{itemize}
\end{PauseHighLight}

\end{slide}

%%%%%%%%%%%%%%%%%%%%%%% Next Slide %%%%%%%%%%%%%%%%%%%%%%%

\begin{slide}
\section{Linear Programming}

\pb
\begin{minipage}{0.4\linewidth}
    \begin{itemize}
    \item Minimise $5\,P + 4\,R$
    \item Subject to
      \begin{itemize}
      \item $3 P + 5 R \geq 20$
      \item $5 P + 2 R \geq 24$
      \item $P, R \geq 0$\pause
      \end{itemize}
    \end{itemize}
\end{minipage}\hfill
\begin{minipage}{0.55\linewidth}
  \multipdf[width=\linewidth]{lp}\pause
\end{minipage}
\end{slide}

%%%%%%%%%%%%%%%%%%%%%%% Next Slide %%%%%%%%%%%%%%%%%%%%%%%

\begin{slide}
\section{Lagrangian}

\begin{PauseHighLight}
  \begin{itemize}
  \item We can write the problem as a Lagrange problem
    \begin{align*}
      \min_{P,R} \max_{A,C} \quad 5\,P + 4\,R - A\,(3 P + 5 R - 20)
      - C \, (5 P + 2 R - 24)\pause
    \end{align*}
  \item subject to $P,R,A,B \geq 0$\pause
  \item $A$ and $C$ are Lagrange multipliers for vitamin A and C\pause
  \item We can rearrange the Lagrangian to obtain
    \begin{align*}
      \max_{A,C} \min_{P,R} \quad 20\, A + 24\, C - P \, (3 A + 5 C - 5)
      - R \, (5 A + 2 C - 4)\pause
    \end{align*}
  \end{itemize}
\end{PauseHighLight}

\end{slide}


%%%%%%%%%%%%%%%%%%%%%%% Next Slide %%%%%%%%%%%%%%%%%%%%%%%

\begin{slide}
\section{Dual Problem}

\begin{PauseHighLight}
  \begin{itemize}
  \item The Lagrangian
    \begin{align*}
      \max_{A,C} \min_{P,R} \quad 20\, A + 24\, C - P \, (3 A + 5 C - 5)
      - R \, (5 A + 2 C - 4)\pause
    \end{align*}
    leads to the dual problem
    \begin{align*}
      \max_{A,C}\; 20\, A + 24\, C \hspace{6cm} \\
      \text{subject to} \quad 3 A + 5 C \leq 5 \quad 5 A + 2 C \leq 4
      \quad A, C \geq 0\pause
    \end{align*}
  \item Consider someone selling vitamins A and C.  They want
    to maximise the price of vitamins A and C, but their prices cannot
    exceed the price of the vitamins in potatoes or rice\pause
  \end{itemize}
\end{PauseHighLight}

\end{slide}

%%%%%%%%%%%%%%%%%%%%%%% Next Slide %%%%%%%%%%%%%%%%%%%%%%%

\begin{slide}
\section{Dual Linear Programme}

\pb
\begin{minipage}{0.4\linewidth}
    \begin{itemize}
    \item Maximise $20\,A + 24\,C$
    \item Subject to
      \begin{itemize}
      \item $3 A + 5 C \leq 5$
      \item $5 A + 2 C \leq 4$
      \item $A, C \geq 0$\pause
      \end{itemize}
    \end{itemize}
\end{minipage}\hfill
\begin{minipage}{0.55\linewidth}
  \multipdf[width=\linewidth]{dualLp}\pause
\end{minipage}
\end{slide}

%%%%%%%%%%%%%%%%%%%%%%% Next Slide %%%%%%%%%%%%%%%%%%%%%%%

\begin{slide}
\section{Why?}

\begin{PauseHighLight}
  \begin{itemize}
  \item Why are we bothered about translating one linear programme
    into another?\pause
  \item Sometime one form is massively easier to solve than the
    other\pause
  \item This is because the first linear programme depends on the
    dimensionality of $\bm{x}$ while the second linear programme
    depends on the number of constraints (or dimensionality of
    $\bm{\alpha}$)\pause
  \item This is important, for example, in Wasserstein GANs\pause
  \end{itemize}
\end{PauseHighLight}

\end{slide}


%%%%%%%%%%%%%%%%%%%%%%% Next Slide %%%%%%%%%%%%%%%%%%%%%%%

\begin{slide}
\section[-2]{Quadratic Programming}

\begin{PauseHighLight}
  \begin{itemize}
  \item A quadratic programme involves minimising a quadratic function
    $\bm{x}^\tr \mat{Q} \bm{x}$ (with $\mat{Q}\succ 0$) subject to
    linear constraints $\mat{M}\,\bm{x} = \bm{b}$ (or
    $\mat{M}\,\bm{x} \leq \bm{b}$)\pause
  \item We can define the Lagrangian
    \begin{align*}
      \mathcal{L}(\bm{x},\bm{\alpha}) = \bm{x}^\tr \mat{Q}\, \bm{x} -
      \bm{\alpha}^\tr \left( \mat{M}\,\bm{x} - \bm{b}\right)\pause
    \end{align*}
  \item Where the solution is given by $\max\limits_{\bm{\alpha}} \,
    \min\limits_{\bm{x}}  \mathcal{L}(\bm{x},\bm{\alpha})$\pause
  \item If the constraints are inequality constraints then
    $\alpha_i\geq0$\pause
  \end{itemize}
\end{PauseHighLight}

\end{slide}

%%%%%%%%%%%%%%%%%%%%%%% Next Slide %%%%%%%%%%%%%%%%%%%%%%%

\begin{slide}
\section{Solution to Quadratic Programming Problem}

\begin{PauseHighLight}
  \begin{itemize}
  \item Using
    \begin{align*}
      \mathcal{L}(\bm{x},\bm{\alpha}) = \bm{x}^\tr \mat{Q}\, \bm{x} -
      \bm{\alpha}^\tr \left( \mat{M}\,\bm{x} - \bm{b}\right)\pause
    \end{align*}
  \item Then
    \begin{align*}
      \grad_{\bm{x}} \mathcal{L}(\bm{x},\bm{\alpha}) =  2\, \mat{Q}\,
      \bm{x} - \mat{M}^\tr \bm{\alpha}\pause
    \end{align*}
  \item So $\grad_{\bm{x}} \mathcal{L}(\bm{x},\bm{\alpha}) = 0$
    implies
    \begin{align*}
      \bm{x}^* = \frac{1}{2} \mat{Q}^{-1} \mat{M}^\tr \bm{\alpha}\pause
    \end{align*}
  \end{itemize}
\end{PauseHighLight}

\end{slide}

%%%%%%%%%%%%%%%%%%%%%%% Next Slide %%%%%%%%%%%%%%%%%%%%%%%

\begin{slide}
\section{Dual Quadratic Programming Problem}

\begin{PauseHighLight}
  \begin{itemize}
  \item Substituting $\bm{x}^* = \frac{1}{2} \mat{Q}^{-1} \mat{M}^\tr
    \bm{\alpha}$ into
    \begin{align*}
      \mathcal{L}(\bm{x},\bm{\alpha}) = \bm{x}^\tr \mat{Q}\, \bm{x} -
      \bm{\alpha}^\tr \left( \mat{M}\,\bm{x} - \bm{b}\right)\pause
    \end{align*}
  \item We get the dual problem
    \begin{align*}
      \max_{\bm{\alpha}}
      -\frac{1}{4} \bm{\alpha}^\tr \mat{M} \mat{Q}^{-1} \mat{M}^\tr
      \bm{\alpha} + \bm{\alpha}^\tr \bm{b}\pause
    \end{align*}
  \item If the constraints were inequality constraints then we have
    $\alpha_i\geq 0$\pause
  \item We have exchanged one quadratic programme for another, but
    sometimes that very useful (e.g. SVMs)\pause    
  \end{itemize}
\end{PauseHighLight}

\end{slide}

%%%%%%%%%%%%%%%%%%%%%%% Next Slide %%%%%%%%%%%%%%%%%%%%%%%

\begin{slide}
\section{Lessons}
\label{last}

\begin{PauseHighLight}
  \begin{itemize}
  \item A useful tool for performing constrained optimisation is the
    introduction of Lagrange multipliers\pause
  \item This is particularly useful for problems with unique solutions
    (it will work when there are multiple solutions, but
    finding many saddle points is a pain)\pause
  \item For inequality constraints we need to satisfy KKT conditions\pause
  \item For simple situations (linear and quadratic programming) we can
    eliminate the original variables to obtain the dual problem\pause
  \end{itemize}
\end{PauseHighLight}

\end{slide}


%%% Local Variables:
%%% TeX-master: "lectures"
%%% End:
