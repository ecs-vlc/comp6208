%% ELEC2013 2007

%% ----------------------------------------------------------------
%% exam.tex
%% ----------------------------------------------------------------
\documentclass{sotonExamBoxes}    % use option [answers] to include them
%\documentclass[answers]{uosexamb}
%% ----------------------------------------------------------------
\usepackage{graphicx}
%\usepackage{pstricks}
\usepackage{alltt}
\usepackage{bm}
\usepackage{amssymb}
\usepackage{amsfonts}
\usepackage{amsmath}
%\usepackage{euler}
\thinmuskip=1mu
%\renewcommand{\ttdefault}{pcr}
\renewcommand{\ttdefault}{pcr}
\newcommand{\tr}{\textsf{T}}
\newcommand{\e}[1]{{\rm e}^{#1}}
\newcommand{\E}[1]{{\rm e}^{{\displaystyle #1}}}
\newcommand{\logg}[1]{\log\left(#1\right)}
\newcommand{\Prob}[1]{\mathbb{P}\left(#1\right)}
\newcommand{\Step}{\mathop{\mathrm{O}\hspace{-15pt}\rule[8.5pt]{11pt}{1.2pt}\hspace{2pt}}}
\newcommand\diag{\mathop{\mathrm{diag}}}
\newcommand{\grad}{\bm{\nabla}}
\newcommand{\dd}{\mathrm{d}}
\newcommand{\class}{\mathcal{C}}
\newcommand{\data}{\mathcal{D}}


%\DeclareMathAlphabet{\mat}{OT1}{cmss}{bx}{n}
%\DeclareMathAlphabet{\mat}{U}{eur}{b}{n}
\DeclareMathAlphabet{\mat}{U}{eur}{b}{n}

\newcommand{\pd}[2]{\ensuremath{\frac{\qpartial #1}{\qpartial #2}}\xspace}
\newcommand{\M}[1]{\ensuremath{\boldsymbol{#1}}\xspace}
\newcommand{\bst}{\rule{0mm}{5mm}}
\newcommand{\av}[2][\,]{\mathbb{E}_{#1}\!\left[ {\strut #2} \right]}
\newcommand{\len}[1]{\| #1 \|}
\graphicspath{{figures/} {../2018-19/figures}}
\newcommand{\inner}[2]{\left\langle #1, #2\right\rangle}
\graphicspath{{figures/}}
\newcommand{\squeeze}{\setlength{\itemsep}{-2pt}}
\renewcommand{\det}[1]{\textrm{det}({#1})}

% \usepackage{picture}
\begin{document}
\unitTitle{Gaussian Processes Problem Sheet}
\authors{Adam Pr\"ugel-Bennett}
\unitcode{COMP6208}
\semester{2}
\year{2022}
\duration{45}{0}


\maketitle

%\blankpage


%% ----------------------------------------------------------------
%\section
% -----------------------------------------------------------------------
% *************************  Section A  *********************************
% -----------------------------------------------------------------------

% ***************************** question 1 **************************

\begin{question}{10}
  Performing integrals over normal distributes takes practice. 
  \begin{qparts}
    \qpart[5]{Consider the integral
      \begin{align*}
        I_1 = \int_{-\infty}^{\infty} \e{-x^2/2} \, \dd x.
      \end{align*}
      Directly evaluating this is difficult, but there is a trick.
      Consider instead
      \begin{align*}
        I_1^2 &=  \int_{-\infty}^{\infty} \e{-x^2/2} \, \dd x
                \int_{-\infty}^{\infty} \e{-y^2/2} \, \dd y
                =   \int_{-\infty}^{\infty}  \int_{-\infty}^{\infty}
                  \e{-(x^2+y^2)/2} \, \dd x \, \dd y.
      \end{align*}
      By making the change of variables to polar coordinates where $r
      = \sqrt{x^2 + y^2}$ and $\theta = \arctan(y/x)$  (so that
      $x=r\,\cos(\theta)$, $y=r\,\sin(\theta)$) then  $\dd
      x\,\dd y = r\,\dd r\, \dd \theta$. Note that to integrate over
      all space we let $\theta$ vary from 0 to $2\pi$ and $r$ to vary
      from 0 to $\infty$.  Write down the integral in
      polar coordinate, make a further the change of variables
      $u=r^2/2$ to evaluate $I_1^2$ hence compute $I_1$
    }{\lines{6}}
    \begin{answer}
      \begin{align*}
        I_1^2 = \int_0^{2\,\pi} \dd \theta \int_{0}^\infty r\,e^{-r^2/2} \, \dd r
        = 2\,\pi \ \int_{0}^\infty r\,e^{-r^2/2} \, \dd r.
      \end{align*}
      Making the change of variables $u=r^2/2$ so that $\dd u/\dd r =
      r$ or $\dd u= r\,\dd r$ we get 
      \begin{align*}
        I_1^2 = 2\,\pi\ \int_{0}^\infty e^{-u} \, \dd u = 2\,\pi
      \end{align*}
      so $I_1 = \sqrt{2\,\pi}$.
    \end{answer}
    \qpart[5]{By making a change of variables compute
      \begin{align*}
        I_2 = \int_{-\infty}^{\infty} \e{-(x-\mu)^2/(2\sigma^2)} \dd x
      \end{align*}}{\lines{4}}
    \begin{answer}
      The trick is to let $z=(x-\mu)/\sigma$ so that $\dd z = \dd
      x/\sigma$ or $\dd x = \sigma \dd z$.  Then
      \begin{align*}
        I_2 = \sigma \int_{-\infty}^{\infty} \e{-z^2/2} \dd z =
        \sigma\,I_1 = \sqrt{2\,\pi}\,\sigma
      \end{align*}
      Note that the \textit{probability density function} (PDF) for a
      normally distributed random variable is given by
      \begin{align*}
        \mathcal{N}(x\mid \mu, \sigma^2) = \frac{1}{\sqrt{2\,\pi}\,\sigma} \e{-(x-\mu)^2/(2\sigma^2)} .
      \end{align*}
      (Observe that $ \mathcal{N}(0\mid \mu, \sigma^2) =
      1/(\sqrt{2\,\pi}\,\sigma)$ so when $\sigma<\sqrt{2\,\pi}$ then
      $\mathcal{N}(0\mid \mu, \sigma^2) >1$, showing that PDFs are not probabilities.)
    \end{answer}
    \newpage
    
    \qpart[3]{By using the identity $\e{a+b} = \e{a}\,\e{b}$, or more
      generally
      \begin{align*}
        \e{\sum_i a_i} = \prod_i \e{a_i}
      \end{align*}
      compute
      \begin{align*}
        I_3 = \int_{-\infty}^\infty \cdots \int_{-\infty}^\infty
        \e{-\frac{1}{2}\|\bm{x}\|_2^2} \dd x_1 \cdots \dd x_n
      \end{align*}
      where $\bm{x} = (x_1, x_2, \ldots, x_n)^\tr$.}{\lines{6}}
      \begin{answer}
        We note that $\|\bm{x}\|_2^2 = \sum\limits_{i=1}^n x_i^2$
        \begin{align*}
          I_3 &= \int_{-\infty}^\infty \cdots \int_{-\infty}^\infty
        \e{-\frac{1}{2}  \sum\limits_{i=1}^n x_i^2} \dd x_1 \cdots \dd
                x_n \\
          &= \prod_{i=1}^n \int_{-\infty}^\infty  \e{-x_i^2/2} \,
            \dd x_i =  \prod_{i=1}^n \sqrt{2\,\pi} = (2\,\pi)^{n/2}.
        \end{align*}
      \end{answer}

      \qpart[6]{By using the fact that for a positive semi-definite
        matrix, $\mat{\Xi}$, we can use the eigenvector decomposition
        $\mat{\Xi}^{-1} = \mat{V}\,\mat{\Lambda}^{-1}\,\mat{V}^\tr$ where
        $\mat{V}$ is an orthogonal matrix with determinant
        $\det{\mat{V}}=\pm1$ and $\mat{\Lambda}^{-1}$ is a diagonal matrix with
        elements $\lambda_i^{-1}$ compute
      \begin{align*}
        I_4 =  \int_{-\infty}^\infty \cdots \int_{-\infty}^\infty
        \e{-\frac{1}{2}  (\bm{x}-\bm{\mu})^\tr \mat{\Xi}^{-1} (\bm{x}-\mu)} \, \dd x_1 \cdots \dd x_n.
      \end{align*}}{\lines{9}}
      \begin{answer}
        This needs some confidence to push through to the end.
        \begin{align*}
          I_4 &=  \int_{-\infty}^\infty \cdots \int_{-\infty}^\infty
          \e{-\frac{1}{2}  (\bm{x}-\bm{\mu})^\tr 
                 \mat{V}\,\mat{\Lambda}^{-1}\,\mat{V}^\tr (\bm{x}-\bm{\mu})}
                 \, \dd x_1 \cdots \dd x_n.
        \end{align*}
        We make the change of variables $\bm{y}= \mat{V}^\tr(\bm{x}-\bm{\mu})$.
        However, to make a change of variables,
        $\bm{x}\rightarrow\bm{y}$, where we use
        \begin{align*}
           \iiint_{\mathcal{R}(\bm{x})}
          f(\bm{x}) \, \dd x_1 \cdots \dd x_n = \iiint_{\mathcal{R}(\bm{y})}
          f(\bm{x}(\bm{y})) \, |\mat{J}|\, \dd y_1 \cdots \dd y_n 
        \end{align*}
        where $\mathcal{R}(\bm{y})$ is the same region as
        $\mathcal{R}(\bm{x})$ but specified in $\bm{y}$-coordinates
        and $\mat{J}$ is the Jacobian matrix with elements
        $J_{ij} = \partial x_i/\partial y_j$.  In our case
        $\bm{x}(\bm{y}) = \mat{V}\,\bm{y} + \bm{\mu}$
        \begin{align*}
          J_{ij} = \frac{\partial x_i}{\partial y_j} = \frac{\partial
          \ }{\partial y_j} \left(\sum_{k=1}^n V_{ik}y_k + \mu_i)\right = V_{ij}
        \end{align*}
        so that $\mat{J} = \mat{V}$ and $|\det{\mat{J}}| =
        |\det{\mat{V}}| = 1$.  This makes sense as the matrix
        $\mat{V}$ corresponds to a rotation (with a possible
        reflection) which does not change the volume.

        Thus on making the change of variables
        \begin{align*}
           I_4 &=  \int_{-\infty}^\infty \cdots \int_{-\infty}^\infty
          \e{-\frac{1}{2}  \bm{y}^\tr \mat{\Lambda}^{-1} \bm{y}}
                  \, \dd y_1 \cdots \dd y_n.
                  = \prod_{i=1}^n  \int_{-\infty}^\infty
                  \e{-y_i^2/(2\,\lambda_i)} \, \dd y_i = \prod_i
                 \sqrt{2\,\pi\,\lambda_i}
        \end{align*}
        where we used $I_2$ with $\sigma = \sqrt{\lambda_i}$.  (Note
        if $\mat{\Xi}$ was not positive semi-definite $\lambda_i$
        would be negative and the integral would not converge.)
      \end{answer}
      \freshpage
      
      \qpart[1]{Using the facts, that $\mat{\Xi} =
        \mat{V}\,\mat{\Lambda}\,\mat{V}^\tr$, for any two square
        matrices $\mat{A}$ and $\mat{B}$ the determinants satisfy
        $\det{\mat{A}\,\mat{B}} = \det{\mat{A}}\,\det{\mat{B}}$, and
        $\det{\mat{V}}=\det{\mat{V}^\tr}=\pm{1}$ show that $\det{\mat{\Xi}}=\prod_i
        \lambda_i$.}{\lines{6}}
      \begin{answer}
        \begin{align*}
          \det{\mat{\Xi}} = \det{\mat{V}\,\mat{\Lambda}\,\mat{V}^\tr} =
          \det{\mat{V}}\,\det{\mat{\Lambda}}\,\det{\mat{V}^\tr}
          = \det{\mat{\Lambda}} = \prod_{i=1}^n \lambda_i
        \end{align*}
        We have used that $\det{\mat{V}}=\det{\mat{V}^\tr}$ which equals 1 or
        -1, but $\det{\mat{V}}\times\det{\mat{V}^\tr}=1$.  Also we use that
        the determinant of a diagonal matrix is equal to the product
        of the diagonal elements.  We note that
        \begin{align*}
          I_4 = (2\,\pi)^{n/2} \sqrt{\det{\mat{\Xi}}}
        \end{align*}
        There is another trick to simplify the notation a bit more.
        For an $n\times n$ matrix $\mat{M}$ then
        $\det{c\,\mat{M}}=c^n\,\det{\mat{M}}$ as the determinant of the matrix
        involves a sum of terms where each term involves a product of
        $n$ elements.  Multiplying each element by $c$ means that the
        determinant increases by $c^n$.  Thus, we can write
        \begin{align*}
            I_4 = (2\,\pi)^{n/2} \sqrt{\det{\mat{\Xi}}}= \sqrt{\det{2\,\pi\,\mat{\Xi}}}
        \end{align*}
        and the multivariate normal PDF is
        \begin{align*}
          \mathcal{N}(\bm{x} \mid \bm{\mu}, \mat{\Xi}) =
          \frac{1}{\sqrt{\det{2\,\pi\,\mat{\Xi}}}}  \, \e{-\frac{1}{2}  (\bm{x}-\bm{\mu})^\tr \mat{\Xi}^{-1} (\bm{x}-\mu)}
        \end{align*}
      \end{answer}
    \end{qparts}
  \end{question}
  \clearpage
% ***************************** question 2 **************************


\begin{question}{25}
  Consider a multivariate normal distribution
  \begin{align*}
    f_{\bm{X},\bm{Y}}(\bm{x}, \bm{y}) = \mathcal{N}
    \!\left( \begin{pmatrix}\bm{x}\\ \bm{y}\end{pmatrix}
    \,\middle|\, \begin{pmatrix} \bm{0} \\
      \bm{0} \end{pmatrix}, \begin{pmatrix} \mat{A} & \mat{B}\\
      \mat{B}^{\tr} & \mat{C} \end{pmatrix} \right) 
  \end{align*}
  where $\mat{A}$ and $\mat{C}$ are symmetric (positive definite)
  matrices.  The matrix
  \begin{align*}
    \mat{\Xi} &= \begin{pmatrix} \mat{A} & \mat{B}\\ \mat{B}^{\tr} & \mat{C} \end{pmatrix}
  \end{align*}
  is the covariance matrix.

  We want to compute the conditional probability density
  function $f_{\bm{X},\bm{Y}}(\bm{x}\mid \bm{y})$.  This is complicated because the
  normal distribution involve the inverse of the covariance matrix.  Let
      \begin{align*}
        \mat{U} &= \begin{pmatrix} \mat{I} & \mat{B}\,\mat{C}^{-1}\\
          \mat{0} & \mat{I} \end{pmatrix},
         & \mat{D} &=  \begin{pmatrix} \mat{A}-
           \mat{B}\,\mat{C}^{-1}\mat{B}^{\tr}
           & \mat{0}\\ \mat{0}^{\tr} & \mat{C}  \end{pmatrix}
      \end{align*}
      where $\mat{I}$ is the identity matrix.
  \begin{qparts}
    \qpart[3]{Compute $\mat{U}\,\mat{D}$}{\lines{5}}
    \begin{answer}
      \begin{align*}
        \mat{U}\,\mat{D}&=\begin{pmatrix} \mat{I} &
          \mat{B}\,\mat{C}^{-1}\\ \mat{0} &
          \mat{I} \end{pmatrix} \begin{pmatrix} \mat{A}-
          \mat{B}\,\mat{C}^{-1}\mat{B}^{\tr} & \mat{0}\\ \mat{0}^{\tr}
          & \mat{C} \end{pmatrix}\\
        &=\begin{pmatrix} \mat{A}- \mat{B}\,\mat{C}^{-1}\mat{B}^{\tr}
          & \mat{B}\\ \mat{0}^{\tr} & \mat{C} \end{pmatrix}
      \end{align*}
    \end{answer}
    
    \qpart[3]{Using the previous result compute
      $(\mat{U}\,\mat{D})\,\mat{U}^\tr$.  Hence show
      $\mat{\Xi} = \mat{U}\,\mat{D}\,\mat{U}^\tr$.}{\lines{5}}
    \begin{answer}
      \begin{align*}
        (\mat{U}\,\mat{D})\,\mat{U}^{\tr}&=\begin{pmatrix} \mat{A}-
          \mat{B}\,\mat{C}^{-1}\mat{B}^{\tr} & \mat{B}\\ \mat{0}^{\tr} &
          \mat{C} \end{pmatrix}\begin{pmatrix} \mat{I} & \mat{0}\\
          \mat{C}^{-1}\mat{B}^{\tr} & \mat{I} \end{pmatrix} =\begin{pmatrix}
          \mat{A} & \mat{B}\\ \mat{B}^{\tr} & \mat{C} \end{pmatrix} = \mat{\Xi}
      \end{align*}
    \end{answer}
    \clearpage
    
    \qpart[1]{Given that $\mat{\Xi} = \mat{U}\,\mat{D}\,\mat{U}^\tr$
      write down $\mat{\Xi}^{-1}$ in terms of $\mat{U}$ and
      $\mat{D}$}{\lines{1}}
    \begin{answer}
      \begin{align*}
        \mat{\Xi}^{-1} = (\mat{U}^{\tr})^{-1}\mat{D}^{-1} \mat{U}^{-1}
      \end{align*}
    \end{answer}
    
    \qpart[6]{Demonstrate by direct multiplication that
      \begin{align*}
        \mat{U}^{-1}
        &= \begin{pmatrix} \mat{I} &
          -\mat{B}\,\mat{C}^{-1} \\ \mat{0} & \mat{I} \end{pmatrix}
                                   &
         \mat{D}^{-1}
         &=\begin{pmatrix} (\mat{A}-
           \mat{B}\,\mat{C}^{-1}\mat{B}^{\tr})^{-1} & \mat{0}\\
           \mat{0}^{\tr} & \mat{C}^{-1} \end{pmatrix} 
      &
      (\mat{U}^\tr)^{-1} = \begin{pmatrix} \mat{I} &
        \mat{0} \\ -\mat{C}^{-1} \mat{B}^\tr & \mat{I} \end{pmatrix} 
      \end{align*}
      i.e. show $\mat{U}^{-1}\mat{U}=\mat{I}$,
      $\mat{D}^{-1}\mat{D}=\mat{I}$ and
      $(\mat{U}^\tr)^{-1}\mat{U}^\tr= \mat{I}$.
    }{\inumberedlines[3]{3}}
    \begin{answer}
    \begin{enumerate}
      \item
        \begin{align*}
          \mat{U}^{-1}\mat{U} = \begin{pmatrix} \mat{I} &
            -\mat{B}\,\mat{C}^{-1} \\ \mat{0} &
            \mat{I} \end{pmatrix} \begin{pmatrix} \mat{I} &
            \mat{B}\,\mat{C}^{-1}\\ \mat{0} & \mat{I} \end{pmatrix}
          = \begin{pmatrix} \mat{I} & \mat{0} \\ \mat{0} &
            \mat{I} \end{pmatrix} 
        \end{align*}
      \item
        \begin{align*}
          \mat{D}^{-1}\mat{D} =  \begin{pmatrix} (\mat{A}-
           \mat{B}\,\mat{C}^{-1}\mat{B}^{\tr})^{-1} & \mat{0}\\
           \mat{0}^{\tr} & \mat{C}^{-1} \end{pmatrix} \begin{pmatrix} \mat{A}-
           \mat{B}\,\mat{C}^{-1}\mat{B}^{\tr}
           & \mat{0}\\ \mat{0}^{\tr} &
           \mat{C}  \end{pmatrix}= \begin{pmatrix} \mat{I} & \mat{0}
           \\ \mat{0} &  \mat{I} \end{pmatrix}                                                                                   
        \end{align*}
        where we use $(\mat{A}-
           \mat{B}\,\mat{C}^{-1}\mat{B}^{\tr})^{-1}(\mat{A}-
           \mat{B}\,\mat{C}^{-1}\mat{B}^{\tr}) = \mat{I}$ and
           $\mat{C}^{-1}\mat{C}=\mat{I}$.
         \item
           \begin{align*}
              (\mat{U}^\tr)^{-1}\mat{U}^\tr =
           \begin{pmatrix} \mat{I} &
        \mat{0} \\ -\mat{C}^{-1} \mat{B}^\tr & \mat{I} \end{pmatrix} 
         \begin{pmatrix} \mat{I} & \mat{0}\\ 
           \mat{C}^{-1}\mat{B}^{\tr} & \mat{I} \end{pmatrix}
                                       =          \begin{pmatrix} \mat{I} & \mat{0}
           \\ \mat{0} &  \mat{I} \end{pmatrix}                                             
           \end{align*}
         \end{enumerate}
       \end{answer}
       \clearpage
       
       \qpart[4]{Letting $\bm{z} = \binom{\bm{x}}{\bm{y}}$ then we can
         write
         \begin{align*}
           f_{\bm{X},\bm{Y}}(\bm{x}, \bm{y}) = f_{\bm{Z}}(\bm{z}) = \mathcal{N}(\bm{z}\mid\bm{0},
           \mat{\Xi}) = \frac{1}{\sqrt{\det{2\,\pi\,\mat{\Xi}}}} \,
           \e{- \frac{1}{2}\, \bm{z}^{\tr}\mat{\Xi}^{-1} \mat{z}}.
         \end{align*}
         where $\det{2\,\pi\,\mat{\Xi}}$ is the determinant of the matrix
         $2\,\pi\,\mat{\Xi}$ and is introduced to ensures that
         $f_{\bm{X},\bm{Y}}(\bm{x},\bm{y})$ is normalised.  From parts (c) and (d)
         \begin{align*}
           \mat{\Xi}^{-1} = \begin{pmatrix} \mat{I} & \mat{0} \\
             -\mat{C}^{-1} \mat{B}^\tr&
             \mat{I} \end{pmatrix} \begin{pmatrix} (\mat{A}-
             \mat{B}\,\mat{C}^{-1}\mat{B}^{\tr})^{-1} & \mat{0}\\
             \mat{0}^{\tr} &
             \mat{C}^{-1} \end{pmatrix} \begin{pmatrix} \mat{I} &
             -\mat{B}\,\mat{C}^{-1} \\ \mat{0} &
             \mat{I} \end{pmatrix}. 
         \end{align*}
         Expand out $\bm{z}^{\tr}\mat{\Xi}^{-1} \mat{z} =
         (\bm{x}^\tr, \bm{y}^\tr)\mat{\Xi}^{-1}
         \binom{\bm{x}}{\bm{y}}$ (start by multiplying the vectors
         $\bm{z}^\tr$ by $(\mat{U}^\tr)^{-1}$ and $\bm{z}$ by $\mat{U}^{-1}$)
       }{\lines{10}}
       \begin{answer}
         \begin{align*}
           \bm{z}^{\tr}\mat{\Xi}^{-1}\bm{z} &=
                                              (\bm{x}^{\tr},\bm{y}^{\tr})\begin{pmatrix}
                                                \mat{I} & \mat{0} \\
                                                -\mat{C}^{-1}
                                                \mat{B}^{\tr} &
                                                \mat{I} \end{pmatrix} \begin{pmatrix}
                                                (\mat{A}-
                                                \mat{B}\,\mat{C}^{-1}\mat{B}^\tr)^{-1}
                                                & \mat{0}\\
                                                \mat{0}^{\tr} &
                                                \mat{C}^{-1} \end{pmatrix} \begin{pmatrix}
                                                \mat{I} &
                                                -\mat{B}\,\mat{C}^{-1}
                                                \\ \mat{0} &
                                                \mat{I} \end{pmatrix} \begin{pmatrix}\bm{x}\\
                                                \bm{y}\end{pmatrix}\\ 
                                            &=
                                              (\bm{x}^{\tr}-\bm{y}^\tr\mat{C}^{-1}\mat{B}^{\tr},\,\bm{y}^{\tr}) \begin{pmatrix}
                                                (\mat{A}-
                                                \mat{B}\,\mat{C}^{-1}\mat{B}^{\tr})^{-1}
                                                & \mat{0}\\
                                                \mat{0}^{\tr} &
                                                \mat{C}^{-1} \end{pmatrix} \begin{pmatrix}\bm{x}-\mat{B}\,\mat{C}^{-1}\bm{y}\\
                                                \bm{y}\end{pmatrix}\\ 
                                            &=
                                              (\bm{x}^{\tr}-\bm{y}^{\tr}\mat{C}^{-1}\mat{B}^{\tr})(\mat{A}-
                                              \mat{B}\,\mat{C}^{-1}\mat{B}^{\tr})^{-1}(\bm{x}-\mat{B}\,\mat{C}^{-1}\bm{y})
                                              + y^{\tr} \mat{C}^{-1}
                                              \bm{y}
         \end{align*}
       \end{answer}
       \clearpage
       
       In the next question we use the short-hand notation
       \begin{align*}
         \int f(\bm{z}) \, \dd \bm{z} = \int \cdots \int f(\bm{z}) \,
         \dd z_1\, \dd z_2 \ldots \dd z_n
       \end{align*}
       \qpart[5]{To compute $f_{\bm{X}\mid\bm{Y}}(\bm{x}\mid\bm{y}) = f_{\bm{X},\bm{Y}}(\bm{x},
         \bm{y})/f_{\bm{Y}}(\bm{y})$ we need to find
         \begin{align*}
           f_{\bm{Y}}(\bm{y})&= \int_{-\infty}^{\infty} f(\bm{x}, \bm{y}) \,\dd \bm{x}\\
                      &= \int_{-\infty}^{\infty}
                        \frac{1}{\sqrt{\det{2\,\pi\,\mat{\Xi}}}} \,
                        \e{- \frac{1}{2}\,
                        (\bm{x}^{\tr}-\bm{y}^{\tr}\mat{C}^{-1}\mat{B}^{\tr})(\mat{A}-
                        \mat{B}\,\mat{C}^{-1}\mat{B}^\tr)^{-1}(\bm{x}
                        -\mat{B}\,\mat{C}^{-1}\bm{y}) - \frac{1}{2}\,
                        y^{\tr} \mat{C}^{-1} \bm{y}} \dd \bm{x} \\
                      &=  \frac{\e{- \frac{1}{2}\,  y^{\tr} \mat{C}^{-1} \bm{y}}}{\sqrt{\det{2\,\pi\,\mat{\Xi}}}}
                        \int_{-\infty}^{\infty}
                        \e{- \frac{1}{2}\,
                        (\bm{x}^{\tr}-\bm{y}^{\tr}\mat{C}^{-1}\mat{B}^{\tr})(\mat{A}-
                        \mat{B}\,\mat{C}^{-1}\mat{B}^\tr)^{-1}(\bm{x}
                        -\mat{B}\,\mat{C}^{-1}\bm{y}) } \dd \bm{x} 
         \end{align*}
         By making a change of variable from $\bm{x}$ to $\bm{u} =
         \bm{x}-\mat{B}\,\mat{C}^{-1}\bm{y}$
         rewrite the integral
         \begin{align*}
           I = \int_{-\infty}^{\infty}
           \e{- \frac{1}{2}\,
           (\bm{x}^{\tr}-\bm{y}^{\tr}\mat{C}^{-1}\mat{B}^{\tr})(\mat{A}-
           \mat{B}\,\mat{C}^{-1}\mat{B}^\tr)^{-1}(\bm{x}
           -\mat{B}\,\mat{C}^{-1}\bm{y}) } \dd \bm{x} 
         \end{align*}
         then use
         \begin{align*}
           \int_{-\infty}^{\infty}
           \e{- \frac{1}{2}\, \bm{u}^\tr \mat{M}^{-1} \,
           \bm{u}} \dd \bm{u} = \sqrt{\det{2\,\pi\,\mat{M}}}  
         \end{align*}
         to evaluate $f_{\bm{Y}}(\bm{y})$.
       }{\lines{10}}
       \begin{answer}
         Making the change of variables $\bm{u} =
         \bm{x}-\mat{B}\,\mat{C}^{-1}\bm{y}$
          \begin{align*}
           I  =   \int_{-\infty}^{\infty}
           \e{- \frac{1}{2}\,\bm{u}^\tr (\mat{A}-
           \mat{B}\,\mat{C}^{-1}\mat{B}^\tr)^{-1}\bm{u}} \, \dd \bm{u}
            = \sqrt{\det{2\,\pi\,(\mat{A}-\mat{B}\,\mat{C}^{-1}\mat{B}^\tr)}}  
         \end{align*}
         Thus
         \begin{align*}
            f_{\bm{Y}}(\bm{y})&=\sqrt{\frac{\det{2\,\pi\,(\mat{A}-
                         \mat{B}\,\mat{C}^{-1}\mat{B}^\tr)}}{\det{2\,\pi\,\mat{\Xi}}}}
                         \, \e{- \frac{1}{2}\, y^{\tr} \mat{C}^{-1}
                         \bm{y}}.
         \end{align*}
         where $\det{\mat{\Xi}}$ denotes the determinant of a matrix $\mat{\Xi}$.
         In passing we recall that $\mat{\Xi}=\mat{U}^\tr\mat{D}\mat{U}$
         so $\det{\mat{\Xi}}=\det{\mat{U}^\tr}\,\det{\mat{D}}\,\det{\mat{U}}$,  But it
         is easy to show $\det{\mat{U}^{\tr}} = \det{\mat{U}} =1$  and because
         $\mat{D}$ is block diagonal $\det{\mat{D}} = \det{\mat{A}-
         \mat{B}\,\mat{C}^{-1}\mat{B}^{\tr}} \, \det{\mat{C}}$ so that  $\det{\mat{\Xi}} = \det{\mat{A}-
         \mat{B}\,\mat{C}^{-1}\mat{B}^{\tr}} \, \det{\mat{C}}$.  Thus
         \begin{align*}
           f_{\bm{Y}}(\bm{y}) = \frac{1}{\sqrt{\det{2\,\pi\,\mat{C}}}}\, \e{-
           \frac{1}{2}\, y^{\tr} \mat{C}^{-1} \bm{y}} = \mathcal{N}(\bm{y}
           \mid \bm{0}, \mat{C}).
         \end{align*}
         Recall that
           \begin{align*}
             f_{\bm{X},\bm{Y}}(\bm{x}, \bm{y}) = \mathcal{N}
             \!\left( \begin{pmatrix}\bm{x}\\ \bm{y}\end{pmatrix}
             \,\middle|\, \begin{pmatrix} \bm{0} \\
               \bm{0} \end{pmatrix}, \begin{pmatrix} \mat{A} & \mat{B}\\
               \mat{B}^{\tr} & \mat{C} \end{pmatrix} \right) 
           \end{align*}
           so we see $\bm{y}$ is normally distributed with covariance
           $\mat{C}$ as might be expected.
         \end{answer}
         \clearpage
         \qpart[3]{Using  $f_{\bm{X}\mid\bm{Y}}(\bm{x}\mid\bm{y}) = f_{\bm{X},\bm{Y}}(\bm{x},
         \bm{y})/f_{\bm{Y}}(\bm{y})$  write down
         $f_{\bm{X}|\bm{Y}}(\bm{x}\mid\bm{y})$.}{\lines{10}}
       \begin{answer}
         \begin{align*}
           f_{\bm{X}|\bm{Y}}(\bm{x}\mid\bm{y}) &= \frac{f_{\bm{X},\bm{Y}}(\bm{x}),
           \bm{y})}{f_{\bm{Y}}(\bm{y})}
           =
           \frac{\det{2\,\pi\,\mat{C}}^{\frac{1}{2}}}{\det{2\,\pi\,\mat{\Xi}}^{\frac{1}{2}}}
           \,\e{- \frac{1}{2}\,
           (\bm{x}^{\tr}-\bm{y}^{\tr}\mat{C}^{-1}\mat{B}^{\tr})\,(\mat{A}-
           \mat{B}\,\mat{C}^{-1}\mat{B}^\tr)^{-1}(\bm{x}-\mat{B}\,\mat{C}^{-1}\bm{y})}\\ 
           &= \frac{1}{\det{2\,\pi\,(\mat{A}- \mat{B}\,\mat{C}^{-1}\mat{B}^\tr)}^{\frac{1}{2}}}
             \,\e{- \frac{1}{2}\,
             (\bm{x}^{\tr}-\bm{y}^{\tr}\mat{C}^{-1}\mat{B}^{\tr})\,(\mat{A}-
             \mat{B}\,\mat{C}^{-1}\mat{B}^\tr)^{-1}(\bm{x}-\mat{B}\,\mat{C}^{-1}\bm{y})}\\  
           &= \mathcal{N}(\bm{x} \mid \mat{B}\,\mat{C}^{-1}\bm{y},\,\, \mat{A}-  \mat{B}\,\mat{C}^{-1}\mat{B}^\tr). 
         \end{align*}
         This is the conditioning result we use in deriving the
         Bayesian update for a Gaussian Process.
       \end{answer}
    \end{qparts}
\end{question}



\end{document}




