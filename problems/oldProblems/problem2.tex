%% ELEC2013 2007

%% ----------------------------------------------------------------
%% exam.tex
%% ----------------------------------------------------------------
\documentclass{sotonExamBoxes}    % use option [answers] to include them
%\documentclass[answers]{uosexamb}
%% ----------------------------------------------------------------
\usepackage{graphicx}
%\usepackage{pstricks}
\usepackage{alltt}
\usepackage{bm}
\usepackage{amssymb}
\usepackage{amsfonts}
\usepackage{amsmath}
\thinmuskip=1mu
%\renewcommand{\ttdefault}{pcr}
\renewcommand{\ttdefault}{pcr}
\newcommand{\tr}{\textsf{T}}
\newcommand{\e}[1]{{\rm e}^{#1}}
\newcommand{\E}[1]{{\rm e}^{{\displaystyle #1}}}
\newcommand{\logg}[1]{\log\left(#1\right)}
\newcommand{\bra}[1]{\left(#1\right)}
\newcommand{\Prob}[1]{\mathbb{P}\left(#1\right)}
\newcommand{\Step}{\mathop{\mathrm{O}\hspace{-15pt}\rule[8.5pt]{11pt}{1.2pt}\hspace{2pt}}}
\newcommand\diag{\mathop{\mathrm{diag}}}
\newcommand{\grad}{\bm{\nabla}}
\newcommand{\dd}{\mathrm{d}}
\newcommand{\class}{\mathcal{C}}
\newcommand{\data}{\mathcal{D}}
\DeclareMathAlphabet{\mat}{OT1}{cmss}{bx}{n}
\newcommand{\pd}[2]{\ensuremath{\frac{\partial #1}{\partial #2}}\xspace}
\newcommand{\M}[1]{\ensuremath{\boldsymbol{#1}}\xspace}
\newcommand{\bst}{\rule{0mm}{5mm}}
\newcommand{\av}[2][\,]{\mathbb{E}_{#1}\!\left[ {\strut #2} \right]}
\newcommand{\len}[1]{\| #1 \|}
\graphicspath{{figures/} {../2018-19/figures}}
\newcommand{\inner}[2]{\left\langle #1, #2\right\rangle}
\graphicspath{{figures/}}
\newcommand{\squeeze}{\setlength{\itemsep}{-2pt}}
\newcommand{\Tr}{\mathop{\mathrm{tr}}\,}

% \usepackage{picture}
\begin{document}
\unitTitle{Problem Sheet 2 for Advanced Machine Learning (COMP6208)}
\authors{Adam Pr\"ugel-Bennett}
\unitcode{COMP6208}
\semester{2}
\year{2022}
\duration{45}{0}


\maketitle

This problem sheet asks you to prove some well known results.
Although the algebra is easy the proofs are not entirely
straightforward.  There are marks assigned to the readability of the
solution and also how well laid out and explained the steps you make
are. (A good proof needs to be easy to follow: you need not comment on
trivial algebra, but there should not be steps that are difficult to
follow).

This looks very mathematical, but it helps to develop the tools and
language that is used to describe machine learning.

%\blankpage


%% ----------------------------------------------------------------
%\section
% -----------------------------------------------------------------------
% *************************  Section A  *********************************
% -----------------------------------------------------------------------

% ***************************** question 1 **************************


\begin{question}{10}
  \begin{qparts}
    \qpart[4]{Starting from the definition of a convex function
      \begin{align}\label{eq:convex1}
        f\bra{a\,x + (1-a)\,y} \leq a\,f(x) + (1-a) \, f(y)
      \end{align}
      Let $a = \epsilon/(x-y)$ and rearrange the
      inequality to give
      \begin{align*}
        (x-y)\,\bra{\frac{f(y+\epsilon) - f(y)}{\epsilon}}
      \end{align*}
      on the left-hand side.  Taking the limit $\epsilon \rightarrow 0$
      show that the function $f(x)$ lies above the tangent line
      $t(x) = f(y) + (x-y)\,f'(y)$ going through the point $y$.}{\lines{12}}
      \begin{answer}
        Rearranging the Equation~\ref{eq:convex1}
        \begin{align*}
          f\bra{y + a(x-y)} \leq f(y) + a\,\bra{f(x) - f(y)}.
        \end{align*}
        Or
        \begin{align*}
          \frac{f\bra{y + a(x-y)} - f(y)}{a} \leq f(x) - f(y).
        \end{align*}
        Letting $a =\epsilon/(x-y)$ then
        \begin{align*}
          (x-y) \,  \frac{f\bra{y + \epsilon} - f(y)}{\epsilon} \leq
          f(x) - f(y)
        \end{align*}
        Taking the limit $\epsilon \rightarrow 0$ then using
        \begin{align*}
          \lim_{\epsilon \rightarrow 0} \frac{f\bra{y + \epsilon} -
          f(y)}{\epsilon}  = f'(y)
        \end{align*}
        So that
        \begin{align*}
           (x-y) \, f'(y) \leq f(x) - f(y)
        \end{align*}
        or
        \begin{align*}
          f(x) \geq f(y) + (x-y)\,f'(y) = t(x).
        \end{align*}
      \end{answer}
      \qpart[1]{Sketch the tangent line, $t(x)$, at the point $y$ in the graph
        shown below.}{\qfig[7cm]{tangentline_q}}
      \begin{answer}
        \afig[7cm]{tangentline_a}
      \end{answer}
      \qpart[4]{Starting from the inequality for a convex function
        \begin{align}\label{eq:convex2}
          f(x) \geq f(y)  + (x-y)\, f'(y)
        \end{align}
        consider the case $y = x+\epsilon$, then by Taylor expanding $f(x+\epsilon)$
        and $f'(x+\epsilon)$ around $x$ and keeping all terms up to
        order $\epsilon^2$ show that for a convex function $f''(x)\geq0$.
      }{\lines{8}}
      \begin{answer}
        We use the expansions
        \begin{align*}
          f(x+\epsilon) &= f(x) + \epsilon\,f'(x) +
                          \frac{\epsilon^2}{2} f''(x) + O(\epsilon^3) \\
          f'(x+\epsilon) &= f'(x) + \epsilon\,f''(x) +
                           O(\epsilon^2) 
        \end{align*}
        Substituting into the Equation~(\ref{eq:convex2})
        \begin{align*}
          f(x) \geq f(x) + \epsilon\,f'(x) +
          \frac{\epsilon^2}{2} f''(x) + O(\epsilon^3)
          - \epsilon\,\bra{f'(x) + \epsilon\,f''(x) +
                           O(\epsilon^2) }
        \end{align*}
        Or subtraction $f(x)$ on both sides
        \begin{align*}
          0 \geq - \frac{\epsilon^2}{2} \, f''(x) + O(\epsilon^3)
        \end{align*}
        Since this has to be true for all $\epsilon>0$ this requires $f''(x)\geq0$.
      \end{answer}
      \qpart[1]{Prove that $x^4$ is convex.}{\lines{1}}
      \begin{answer}
        \begin{align*}
          \frac{\dd^2 x^4}{\dd \, x^2} = 12\,x^2 \geq 0
        \end{align*}
      \end{answer}
  \end{qparts}
\end{question}
\freshpage

% ***************************** question 2 **************************

\begin{question}{10}
  \begin{qparts}
    \qpart[2]{Show by writing out in component for that
      $\Tr \mat{A}\,\mat{B} = \Tr \mat{B}\,\mat{A}$ where
      $\Tr \mat{M} = \sum_i M_{ii}$ (i.e. the trace of a matrix is
      equal to the sum of terms down the diagonal).}{\lines{5}}
    \begin{answer}
      \begin{align*}
        \Tr \mat{A}\,\mat{B}  &= \sum_{i,j} A_{ij}\,B_{ji}
        &
          \Tr \mat{B}\,\mat{A} &= \sum_{k,l} B_{kl}\,A_{lk}
      \end{align*}
      These are identical (which is clearer if we let $l=i$ and
      $k=j$).
    \end{answer}
    \qpart[2]{Using the fact that we can write a symmetric matrix
      $\mat{M}$ as $\mat{M}= \mat{V}\,\mat{\Lambda}\,\mat{V}^\tr$
      where $\mat{V}$ is an orthogonal matrix and $\mat{\Lambda} =
      \mathrm{diag}(\lambda_1, \lambda_2, \ldots)$
      (i.e. a diagonal matrix with $\Lambda_{ii}=\lambda_i$). Show
      that $\Tr \mat{M} = \sum_{i} \lambda_i$.}{\lines{5}}
    \begin{answer}
      \begin{align*}
        \Tr \mat{M} = \Tr \bra{\mat{V}\mat{\Lambda}\mat{V}^\tr}
        = \Tr \bra{\mat{V}^\tr\mat{V}\,\mat{\Lambda}}
        = \Tr
        \mat{\Lambda} = \sum_i \lambda_i
      \end{align*}
      where we have used (1) $\Tr \mat{A}\,\mat{B} = \Tr
      \mat{B}\,\mat{A}$  with $\mat{A} = \mat{V}\mat{\Lambda}$ and
      $\mat{B}= \mat{V}^\tr$ and (2) $\mat{V}^\tr\mat{V}=\mat{I}$.
    \end{answer}
    \qpart[2]{Consider the matrix $\mat{X} = (\bm{x}_1, \bm{x}_2,
      \ldots, \bm{x}_n)$ where  the $i^{th}$ column of $\mat{X}$ is
      the vector $\bm{x}_i$.  Compute $\Tr \mat{X}^\tr
      \mat{X}$}{\lines{5}}
    \begin{answer}
      \begin{align*}
        \Tr \mat{X}^\tr \mat{X}
        = \Tr
        \begin{pmatrix}
          \bm{x}_1^\tr \bm{x}_1 & \bm{x}_1^\tr \bm{x}_2 & \ldots &
          \bm{x}_1^\tr \bm{x}_n \\
          \bm{x}_2^\tr \bm{x}_1 & \bm{x}_2^\tr \bm{x}_2 & \ldots &
          \bm{x}_2^\tr \bm{x}_n \\
          \vdots & \vdots & \ddots & \vdots \\
          \bm{x}_n^\tr \bm{x}_1 & \bm{x}_n^\tr \bm{x}_2 & \ldots &
          \bm{x}_n^\tr \bm{x}_n \\
        \end{pmatrix}
        = \sum_{i=1}^n \bm{x}_i^\tr \bm{x}_i = \sum_{i=1}^n
        \len{\bm{x}_i}^2
      \end{align*}
    \end{answer}
    \qpart[2]{The Frobenius norm, $\len{\mat{X}}_F$ for a matrix
      $\mat{X}$ is given by
      \begin{align*}
        \len{\mat{X}}_F =\sqrt{ \sum\limits_{i,j} X_{ij}^2}
      \end{align*}
      using the previous result show that $\len{\mat{X}}_F^2 = \Tr
      \mat{X}^\tr \mat{X}$
    }{\lines{6}}
    \begin{answer}
      \begin{align*}
        \len{\mat{X}}_F^2 = \sum\limits_{i,j} X_{ij}^2 = \sum_j \bra{
        \sum_i X_{ij}^2} = \sum_j \len{\bm{x}_j}^2 = \Tr \mat{X}^\tr \mat{X}
      \end{align*}
      where we have used the fact that  $\sum_i X_{ij}^2$ is a sum
      over all elements in the $j^{th}$ column but the $j^{th}$ column
      is equal to the vector $\bm{x}_j$.
    \end{answer}
    \qpart[2]{By using the SVD $\mat{X} = \mat{U}\,\mat{S}\,\mat{V}^\tr$
      where $\mat{S}=\mathrm{diag}(s_1,s_2,\ldots,x_n)$ (i.e. a
      diagonal matrix where $S_{ii} = s_i$---the $i^{th}$ singular
      value) show using the previous results that $\len{\mat{X}}_F^2 =
      \sum_i s_i^2$.}{\lines{10}}
    \begin{answer}
      Now
      \begin{align*}
        \mat{X}^\tr \mat{X} = \bra{\mat{V}\,\mat{S}^\tr \,\mat{U}^\tr}
         \bra{\mat{U}\,\mat{S}\,\mat{V}^\tr} = \mat{V}\,\mat{S}^\tr \mat{S}\,\mat{V}^\tr
      \end{align*}
      And so using
      \begin{align*}
        \Tr \mat{X}^\tr \mat{X} = \Tr \mat{S}^\tr \mat{S}
      \end{align*}
      but $\mat{S}^\tr \mat{S}$ is a diagonal matrix with elements
      $s_i^2$ so $\Tr \mat{X}^\tr \mat{X} = \sum_{i} s_i^2$.  But we
      have shown that $\len{\mat{X}}_F^2 = \Tr \mat{X}^\tr \mat{X}$,
      so $\len{\mat{X}}_F^2 = \sum_{i} s_i^2$.
    \end{answer}
  \end{qparts}
\end{question}

% ***************************** question 3 **************************


\begin{question}{10}
  The $p$-norm of a matrix $\mat{M}$ is defined to satisfy
  \begin{align}
    \len{\mat{M}}_p &= \max_{\bm{x}\neq \bm{0}}
    \frac{\len{\mat{M}\,\bm{x}}_p}{\len{\bm{x}}_p} \\
    &= \max_{\bm{x}:\len{\bm{x}}_p=1}\len{\mat{M}\,\bm{x}}_p \label{eq:pnorm2}
  \end{align}
  where $\len{\bm{x}}_p$ is the $p$ norm of a vector defined by
  \begin{align*}
    \len{\bm{x}}_p = \bra{ \sum_i |x_i|^p}^{1/p}.
  \end{align*}
  Note that with this definition $\len{\mat{M}\,\bm{x}}_p \leq
  \len{\mat{M}}_p \, \len{\bm{x}}_p$ (where the inequality is tight,
  i.e. there exists a vector where the inequality becomes an
  equality).
    

  \begin{qparts}
    \qpart[2]{If $\mat{U}$ is an orthogonal matrix show that for any
      vector $\bm{v}$ that $\len{\mat{U}\,\bm{v}}_2 = \len{\bm{v}}_2$.
      Use this to show
      $\len{\mat{U}\,\mat{A}}_2 = \len{\mat{A}}_2$.}{\lines{5}}
    \begin{answer}
      We note that
      \begin{align*}
        \len{\mat{U}\,\bm{v}}_2^2 = \bm{v}^\tr \,\mat{U}^\tr
        \mat{U}\,\bm{v} = \bm{v}^\tr\bm{v} = \len{\bm{v}}_2^2
      \end{align*}
      since norms are positive $\len{\mat{U}\,\bm{v}}_2 =
      \len{\bm{v}}_2$.
      We note that  $\len{\mat{U}\,\mat{A}\,\bm{x}}_2 =
      \len{\mat{A}\,\bm{x}}_2$, thus the unit vector $\bm{x}$ that
      maximises $\len{\mat{A}\,\bm{x}}_2$ will also maximise
      $\len{\mat{U}\,\mat{A}\,\bm{x}}_2 $.  By
      Equation~(\ref{eq:pnorm2}) this implies $\len{\mat{U}\,\mat{A}}_2 =
      \len{\mat{A}}_2$.
    \end{answer}
    \qpart[2]{If $\bm{V}$ is an orthogonal matrix show that
      $\len{\mat{A}\,\mat{V}^\tr}_2 = \len{\mat{A}}_2$.}{\lines{5}}
    \begin{answer}
      We consider a unit-vector $\bm{x}$ that maximises
      $\len{\mat{A}\,\bm{x}}_2$ then by definition~(\ref{eq:pnorm2})
      $\len{\mat{A}\,\bm{x}}_2 = \len{\mat{A}}_2$.  But we note that
      $\bm{y}=\mat{V}\,\bm{x}$ will be a unit vector that maximises
      $\len{\mat{A}\,\mat{V}^\tr\bm{y}}_2$ since
      \begin{align*}
        \len{\mat{A}\,\mat{V}^\tr\bm{y}}_2 =
        \len{\mat{A}\,\mat{V}^\tr\mat{V}\bm{x}}_2 =
        \len{\mat{A}\bm{x}}_2
      \end{align*}
      using $,\mat{V}^\tr\mat{V}=\mat{I}$.  But as $\bm{x}$ is a
      unit vector that maximises $ \len{\mat{A}\bm{x}}_2$ then
      $\bm{y}$ will be a unit vector that maximise
      $\len{\mat{A}\,\mat{V}^\tr\bm{y}}_2 $ and
      $\len{\mat{A}\,\mat{V}^\tr}=\len{\mat{A}}$.
    \end{answer}
    \qpart[1]{Using the SVD  $\mat{M} = \mat{U}\,\mat{S}\,\mat{V}^\tr$
      and the results of part (a) and part (b)
      show that $\len{\mat{M}}_2 = \len{\mat{S}}_2$.}{\lines{3}}
    \begin{answer}
      We note that $\len{\mat{M}}_2 =
      \len{\mat{U}\,\mat{S}\,\mat{V}^\tr}_2$.  But by part (a)
      $\len{\mat{U}\,\mat{S}\,\mat{V}^\tr}_2=\len{\mat{S}\,\mat{V}^\tr}_2$
      and by part (b) $\len{\mat{S}\,\mat{V}^\tr}_2=\len{\mat{S}}_2$.
    \end{answer}
      \freshpage
    \qpart[1]{Compute $\len{\mat{S}\,\bm{x}}_2^2$ where
      $\mat{S}=\mathrm{diag}(s_1, s_2, \ldots, s_n)$ is the
      diagonal matrix of singular values}{\lines{2}}
    \begin{answer}
      Since $\mat{S}= \diag(s_1, s_2, \ldots, s_n)$ then
      \begin{align*}
        \len{\mat{S}\,\bm{x}}_2^2 = \sum_{i=1}^n s_i^2\,x_i^2
      \end{align*}
    \end{answer}
    \qpart[4]{Write down the Lagrangian
      to maximise $\len{\mat{S}\bm{x}}_2^2$ subject to
      $\len{\bm{x}}_2^2 =1$.  Compute the extrumum conditions given by
      $\partial L/\partial x_i=0$.  Let $(s_\alpha|\alpha=1,2,\ldots)$
      be the set of unique singular values and $I_\alpha$ the set of
      indices, $i$ such that $s_i=s_\alpha$.  Using the extrumum
      condition and the constraint write down the set of extremum
      values for $\len{\mat{S}\,\bm{x}}$ and hence show that
      $\len{\mat{M}}_2 = s_{max}$ where $s_{max}$ is the maximum
      singular value.}{\lines{12}}
    \begin{answer}
      The Lagrangian is equal to
      \begin{align*}
        L = \sum_{i=1}^n s_i^2 \, x_i^2 - \lambda\bra{\sum_{i=1}^n
        x_i^2 -1}
      \end{align*}
      The extremum conditions are
      \begin{align*}
        \pd{L}{x_i} = x_i\,\bra{s_i^2 - \lambda} = 0
      \end{align*}
      (and the constraint).  The extremum conditions are thus either
      $x_i=0$ or $s_i^2=\lambda$.   Thus the value of Lagrange
      multipliers are $\lambda = s_\alpha$ where $x_i=0$ if $i\not\in
      I_\alpha$.  Thus
      \begin{align*}
        \len{\mat{S}\,\bm{x}}_2^2 = s_\alpha^2 \sum_{i \in I_\alpha} x_i^2
        = s_\alpha^2
      \end{align*}
      where we have used $ \sum_{i} x_i^2=1$.  The maximum condition
      is just given by the case $s_{max}$ (the largest maximum
      eigenvalue).  Thus using Equation~(\ref{eq:pnorm2}) we have
      $\len{\mat{S}}_2= s_{max}$.  But since $\len{\mat{M}}_2 =
      \len{\mat{S}}_2$ it follows that $\len{\mat{M}}_2 = s_{max}$.
    \end{answer}
  \end{qparts}
\end{question}
\freshpage

% ***************************** question 4 **************************

\begin{question}{10}
  \begin{qparts}
    \qpart[2]{We consider the mapping $\bm{y}= \mat{M}\,\bm{x}$ where
      $\mat{M}$ is an $n\times n$ matrix.  Suppose there is some noise
      in $\bm{x}$ so that $\bm{x}' = \bm{x} + \bm{\epsilon}$ so that
      under the mapping $\bm{y}'= \mat{M}\,\bm{x}'$.  Compute an upper
      bound on $\len{\bm{y}'-\bm{y}}_2$ in terms of $\len{\epsilon}$
      and $s_{max}$.}{\lines{3}}
    \begin{answer}
      \begin{align*}
        \len{\bm{y}'-\bm{y}}_2 = \len{\mat{M}\,\bm{\epsilon}}_2 \leq
        \len{\mat{M}}_2 \, \len{\epsilon}_2 = s_{max} \, \len{\epsilon}_2
      \end{align*}
    \end{answer}

    \qpart[3]{For a matrix $\mat{M} = \mat{U}\,\mat{S}\,\mat{V}^\tr$
      show that
      \begin{align*}
         \len{\mat{M}\,\bm{x}}_2 =  \len{\mat{S}\,\bm{a}}_2 \, \len{\bm{w}}_2
      \end{align*}
      where $\bm{a} = \mat{V}^\tr \bm{x}/\len{\bm{x}}_2$ so that
      $\len{a}_2$.  Show that we can lower bound
      $\len{\mat{S}\,\bm{a}}_2^2$ by $s_{min}^2$ hence prove
      \begin{align*}
        \len{\mat{M}\,\bm{x}}_2 \geq s_{min}\,\len{\bm{x}}_2.
      \end{align*}
    }{\lines{8}}
    \begin{answer}
      \begin{align*}
        \len{\mat{X}\,\bm{x}}_2 =
        \len{\mat{U}\,\mat{S}\,\mat{V}^\tr\,\bm{x}}_2
        = \len{\mat{S}\,\mat{V}^\tr\,\bm{x}}_2
      \end{align*}
      but using $\bm{a} = \mat{V}^\tr \bm{x}/\len{\bm{x}}_2$
      \begin{align*}
        \len{\mat{M}\,\bm{x}}_2 =  \len{\mat{S}\,\bm{a}}_2 \, \len{\bm{w}}_2
      \end{align*}
      where $\len{a}_2=1$.  But
      \begin{align*}
        \len{\mat{S}\,\bm{a}}_2^2 = \sum_i a_i^2\,s_i^2 \geq
        s_{min}^2 \sum_i a_i^2 = s_{min}^2\, = s_{min}^2
        \len{\bm{a}}^2 = s_{min}^2.
      \end{align*}
      Thus $\len{\mat{S}\,\bm{a}}_2 \geq s_{min}$ and
      \begin{align*}
         \len{\mat{M}\,\bm{x}}_2 \geq s_{min}\,\len{\bm{x}}_2.
      \end{align*}
    \end{answer}
    
    \qpart[1]{Using the previous results obtain an upper bound for the relative error
      \begin{align*}
        \frac{\len{\bm{y}' - \bm{y}}_2}{\len{\bm{y}}_2}
      \end{align*}
      in terms of $s_{max}$, $s_{min}$, $\len{\bm{\epsilon}}_2$ and
      $\len{\bm{x}}$.}{\lines{2}}
      \begin{answer}
        \begin{align*}
          \frac{\len{\bm{y}' - \bm{y}}_2}{\len{\bm{y}}_2} \leq
          \frac{s_{max}}{s_{min}} \, \frac{\len{\bm{\epsilon}}_2}{\len{\bm{x}}_2}
        \end{align*}
      \end{answer}
      \freshpage
      
      \qpart[1]{The condition number for an invertible square matrix $\mat{M}$ is
        given by $\kappa_2(\mat{M}) =
        \len{\mat{A}}_2\,\len{\mat{A}^{-1}}_2$ (there are different
        condition numbers for different norms.)  Write done the
        condition number in terms of $s_{max}$ and $s_{min}$.}{\lines{4}}
      \begin{answer}
        We observe that $\len{\mat{A}}_2=s_{max}$ and
        $\len{\mat{A}^{-1}}_2 = s_{min}^{-1}$ so that
        \begin{align*}
          \kappa_2(\mat{M}) =
        \len{\mat{A}}_2\,\len{\mat{A}^{-1}}_2 = \frac{s_{max}}{s_{min}}
        \end{align*}
      \end{answer}

      \qpart[3]{In linear regression we make predictions $\hat{y} =
        \bm{x}^\tr \bm{w}$ given an input $\bm{x}$ where $\bm{w} =
        \mat{X}^+\,\bm{y}$ where $\mat{X}^+=
        (\mat{X}^\tr\,\mat{X})^{-1}\mat{X}^\tr$ is the pseudo inverse
        of the design matrix $\mat{X}$ and $\bm{y}$ is a vector of
        training examples.  There are bounds on the accuracy of
        linear regression depending on $\av{s_{max}/s_{min}}$ where
        $s_{max}$ and $s_{min}$ are the maximum and minimum no-zero
        singular value of the design matrix.  Consider randomly drawn
        feature vectors
        \begin{align*}
          \bm{x}_i \sim \mathcal{N}(\bm{0},\mat{I}).
        \end{align*}
        Using python generate the $m\times p$ dimensional design
        matrix $\mat{X}$ with rows $\bm{x}_i^\tr$.  By computing the
        singular values for $\mat{X}$ for $m=i\times p$ where
        $i=1, 2, \ldots, 10$ find $s_{max}/s_{min}$.   Repeat this 10
        times to obtain an estimate of $\av{s_{max}/s_{min}}$.  Plot a
        graph of you estimate for $\av{s_{max}/s_{min}}$ (on a log-axis) versus $m/p$
        for $p=10, 50$ and $100$. }{\qfig[14cm]{condition_number_q}}
      \begin{answer}
        \afig[10cm]{condition_number}
      \end{answer}
  \end{qparts}
\end{question}


% ***************************** End of exam *****************************
\end{document}


% ***************************** question 4 **************************

\begin{question}{10}
  \begin{qparts}
    \qpart[3]{We consider the least squares solution to a linear
      regression problem with data $\mathcal{D} =
      \{(\bm{x}_i,y_i) \mid i=1,2,\ldots, m\}$
      \begin{align*}
        \bm{w}^* = \mat{X}^+ \bm{y} = \bra{\mat{X}^\tr\mat{X}}^{-1} \mat{X} \, \bm{y}
      \end{align*}
      where $\mat{X}$ is the design matrix with rows equal to
      $\bm{x}_i$ and $\bm{y}$ is the vector of targets with elements
      $y_i$.  Assuming $p>m$ and using the SVD
      $\mat{X} = \mat{U}\,\mat{S}\,\mat{V}^\tr$ compute
      $\len{\mat{X}^+}_2$ (see lecture 8).}{\lines{7}}
    \begin{answer}
      Using the properties of orthogonal matrices $\mat{X}^+ =
      \mat{V}\mat{S}^+\mat{U}$ where $\mat{S}^+$ is a matrix with
      non-zero elements on the diagonal equal to $1/s_i$.  Using the
      result from the previous question $\len{\mat{X}^+}_2 =
      1/s_{min}$ where $s_{min}$ is the smallest singular value of the
      design matrix $\mat{X}$.
    \end{answer}
    \qpart[2]{Assume an error, $\bm{\epsilon}$, in the targets
      $\bm{y}=\bm{y}_{true} + \bm{\epsilon}$.  Let
      $\bm{w}_{true} = \mat{X}^+ \bm{y}_{true}$ and
      $\bm{w} = \mat{X}^+ \bm{y} = \mat{X}^+ (\bm{y}+\bm{\epsilon})$.
      In linear regression given a feature vector $\bm{x}$ we make a
      prediction $\hat{y} = \bm{x}^\tr\,\bm{w}$.  Using the
      Cauchy-Schwarz inequality
      $|\bm{a}^\tr\bm{b}| \leq \len{\bm{a}}_2\,\len{\bm{b}}_2$ and the
      norm inequality
      $\len{\mat{A}\bm{x}}_2 \leq \len{\mat{A}}_2 \, \len{\bm{x}}_2$
      and obtain a bound on the absolute error in the prediction
      $|\bm{x}^\tr\,\bm{w} - \bm{x}^\tr\,\bm{w}_{true}|$ involving
      $\len{\bm{x}}_2$, $\len{\bm{\epsilon}}_2$ and $s_{min}$ (the
      minimum singular value of the design matrix
      $\mat{X}$).}{\lines{7}}
    \begin{answer}
      \begin{align*}
        |\bm{x}^\tr\,\bm{w} - \bm{x}^\tr\,\bm{w}_{true}|
        = |\bm{x}^\tr \mat{X}^+ \bm{\epsilon}|
        \leq \len{\bm{x}}_2 \, \len{\mat{X}^+\,\bm{\epsilon}}_2
        \leq \len{\bm{x}}_2 \,
        \len{\mat{X}^+}_2\,\len{\bm{\epsilon}}_2
        = \frac{1}{s_{min}}\len{\bm{x}}_2 \, \len{\bm{\epsilon}}_2
      \end{align*}
      where the first inequality results from applying the
      Cauchy-Schwarz inequality and the second inequality comes from
      the norm inequality.
    \end{answer}
    \freshpage
    \qpart[2]{Using the SVD, $\mat{X}=\mat{U}\mat{S}\mat{V}^\tr$, show that
      \begin{align*}
        \len{\bm{x}^\tr \mat{X}^+}_2 \leq \len{\bm{w}}_2\,\len{\bm{a}^\tr\mat{S}^+}_2
      \end{align*}
      where $\bm{a}^\tr = \bm{x}^\tr\,\mat{V}/\len{\mat{w}}_2$ so that
      $\len{\bm{a}}_2=1$.  Find the structure of the matrix $ \mat{S}^+
      (\mat{S}^+)^\tr$ and thus show that
      \begin{align*}
        \len{\bm{a}^\tr\mat{S}^+}_2^2 = 
        \sum_i s_i^{-2}\,a_i^2
      \end{align*}
      and therefore $\len{\bm{a}^\tr\mat{S}^+}_2^2  \geq s_{max}^{-2}$ where
      $s_{max}$ is the maximum singular value of the design matrix $\mat{X}$.
      From this result show
      \begin{align*}
        \len{\bm{x}^\tr \mat{X}^+}_2 \leq \frac{1}{s_{max}} \len{\bm{x}}_2.
      \end{align*}}{\lines{10}}
      \begin{answer}
        We note that $\mat{X}^+ = \mat{V}\,\mat{S}^+\mat{U}^\tr$ so
        that
        \begin{align*}
          \len{\bm{x}^\tr \mat{X}^+}_2 =
          \len{\bm{x}^\tr\mat{V}\mat{S}^+\mat{U}^\tr} _2
          = \len{\bm{x}^\tr\mat{V}\mat{S}^+}_2
        \end{align*}
        since for any vector $\bm{b}$ if $\mat{U}$ is an orthogonal
        matrix $\len{\bm{b}^\tr \mat{U}^\tr}_2^2 = \bm{b}^\tr\mat{U}^\tr\mat{U}\,\bm{b} =
        \len{\bm{b}}_2^2$.
        If we  let $\bm{a}^\tr = \bm{x}^\tr\,\mat{V}/\len{\mat{w}}$ then
        \begin{align*}
          \len{\bm{x}^\tr \mat{X}^+}_2 = \len{\bm{w}}_2 \, \len{\bm{a}^\tr\,\mat{S}^+}_2.
        \end{align*}
        Now $ \mat{S}^+ (\mat{S}^+)^\tr= \mathrm{diag}(s_1^{-2},
        s_2^{-2}, \ldots s_n^{-2})$.   So that
        \begin{align*}
          \len{\bm{u}^\tr\,\mat{S}^+}_2^2 =
          \bm{a}^\tr \mat{S}^+ (\mat{S}^+)^\tr \bm{a} 
          =   \sum_i s_i^{-2} a_i ^2
          \geq s_{max}^{-2} \sum_i a_i^2
          = s_{max}^{-2}
        \end{align*}
        where we have used the fact that $\len{\bm{a}}=1$.  Using this
        and the previous result
        \begin{align*}
           \len{\bm{x}^\tr \mat{X}^+}_2 > s_{max}^{-1} \len{\bm{w}}_2 .
        \end{align*}
      \end{answer}
      \freshpage

      We note that given a feature vector $\bm{x}$ we can use linear
      regression to make a prediction $\bm{x}^\tr \bm{w}= \bm{x}^\tr
      \mat{X}^+ \bm{y}$.  We have shown that an error of size
      $\bm{\epsilon}$ in the training targets $\bm{y}$ leads to a
      possible error no larger than
      $s_{min}^{-1}\,\len{\bm{x}}_2\,\len{\bm{\epsilon}}_2$.  The
      prediction, $\bm{x}^\tr\bm{w}$ is smaller than $s_{max}^{-1}\,\len{\bm{x}}_2
      \bm{a}^\tr\bm{y}$ where $\len{\bm{a}}_2=1$.  The fractional
      error is
      \begin{align*}
        \frac{\bm{x}^\tr \bm{w} - \bm{x}^\tr
        \bm{w}_{true}}{\bm{x}^\tr\bm{w}}
        \leq \frac{s_{max}}{s_{min}}
        \frac{\len{\bm{\epsilon}}}{\bm{a}^\tr\bm{y}}.
      \end{align*}
      for some unit vector $\bm{a}$.
      This is not a useful bound as $\bm{a}^\tr\bm{y}$ can be
      arbitrarily small (reflecting the fact that $\bm{x}^\tr\bm{w}$
      can be arbitrarily small).  Nevertheless, assuming that
      $\bm{x}$ is not correlated in anyway with $\mat{X}^+ \bm{y}$
      then we would expect $\bm{a}^\tr\bm{y}$ to be proportional to
      $\len{\bm{y}}_2$.  We note that in this case the fractional error
      is amplified by $s_{max}/s_{min}$.  This is an example of a
      condition number (there are different condition numbers
      depending on what norm we use).  We can obtain rigorous error
      bounds for linear regression (they involve slightly more
      algebra), but they all tend to involve condition numbers.
      \qpart[3]{Assume that
        \begin{align*}
          \bm{x}_i \sim \mathcal{N}(\bm{0},\mat{I})
        \end{align*}
        using python form the $m\times p$ matrix $\mat{X}$ with rows
        $\bm{x}_i^\tr$.  Compute the singular values for $\mat{X}$
        for $m=i\times p$ where $i=1, 2, \ldots, 10$.  Compute
        $s_{max}/s_{min}$ and repeat 10 times to obtain an estimate of
        $\av{s_{max}/s_{min}}$.  Plot a graph of you estimate for
        $\av{s_{max}/s_{min}}$ versus $m/p$ for $p=10, 50$ and $100$.
      }{\lines{10}}
  \end{qparts}
\end{question}


