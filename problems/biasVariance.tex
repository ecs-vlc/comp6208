%% ELEC2013 2007

%% ----------------------------------------------------------------
%% exam.tex
%% ----------------------------------------------------------------
\documentclass{sotonExamBoxes}    % use option [answers] to include them
%\documentclass[answers]{uosexamb}
%% ----------------------------------------------------------------
\usepackage{graphicx}
%\usepackage{pstricks}
\usepackage{alltt}
\usepackage{bm}
\usepackage{amssymb}
\usepackage{amsfonts}
\usepackage{amsmath}
\thinmuskip=1mu
%\renewcommand{\ttdefault}{pcr}
\renewcommand{\ttdefault}{pcr}
\newcommand{\tr}{\textsf{T}}
\newcommand{\e}[1]{{\rm e}^{#1}}
\newcommand{\E}[1]{{\rm e}^{{\displaystyle #1}}}
\newcommand{\logg}[1]{\log\left(#1\right)}
\newcommand{\Prob}[1]{\mathbb{P}\left(#1\right)}
\newcommand{\Step}{\mathop{\mathrm{O}\hspace{-15pt}\rule[8.5pt]{11pt}{1.2pt}\hspace{2pt}}}
\newcommand\diag{\mathop{\mathrm{diag}}}
\newcommand{\grad}{\bm{\nabla}}
\newcommand{\dd}{\mathrm{d}}
\newcommand{\class}{\mathcal{C}}
\newcommand{\data}{\mathcal{D}}
\DeclareMathAlphabet{\mat}{OT1}{cmss}{bx}{n}
\newcommand{\pd}[2]{\ensuremath{\frac{\qpartial #1}{\qpartial #2}}\xspace}
\newcommand{\M}[1]{\ensuremath{\boldsymbol{#1}}\xspace}
\newcommand{\bst}{\rule{0mm}{5mm}}
\newcommand{\av}[2][\,]{\mathbb{E}_{#1}\!\left[ {\strut #2} \right]}
\newcommand{\len}[1]{\| #1 \|}
\graphicspath{{figures/} {../2018-19/figures}}
\newcommand{\inner}[2]{\left\langle #1, #2\right\rangle}
\graphicspath{{figures/}}
\newcommand{\squeeze}{\setlength{\itemsep}{-2pt}}

% \usepackage{picture}
\begin{document}
\unitTitle{Bias Variance Problem Sheet}
\authors{Adam Pr\"ugel-Bennett}
\unitcode{COMP6208}
\semester{2}
\year{2022}
\duration{45}{0}


\maketitle

When modelling systems with uncertainty it is convenient to define
\textit{random variables}.  The are numbers that we associate with the
outcome of some stochastic event.  We associate a probability (or
probability density) with the set out outcomes such that the random
variables take a particular value.  We often write random
variables using capital letters (e.g. $X$) while the actual values the
$X$ takes we write with small letters $x$.  Thus $\Prob{X=x}$ is the
probability that the random value, $X$, takes value, $x$.  We write
non-random variables (scalars) with a small letter (e.g. $c$).  Note
that for most continuous random variables $\Prob{X=x}=0$ so instead we
define the probability density
\begin{align*}
  f_X(x) = \lim_{\delta x \rightarrow 0} \frac{\Prob{x\leq X \leq
  x+\delta x}}{\delta x}.
\end{align*}

The expectation (or average value) of some function, $g$, of $X$ is
written as
\begin{align*}
  \av[X]{g(X)} =
  \begin{cases}
    \sum_{x\in \mathcal{X}} \Prob{X=x}\, g(x) \\
    \int f_X(x) \, g(x)\, \dd x
  \end{cases}
\end{align*}
depending on whether $X$ is a continuous or discrete random variable.
Note $\mathcal{X}$ is the possible values that the random variable can take.
When it is clear what random variables we are taking expectation with
respect to then we will often write $\av{\cdot}$ for $\av[X]{\cdot}$.


%\blankpage


%% ----------------------------------------------------------------
%\section
% -----------------------------------------------------------------------
% *************************  Section A  *********************************
% -----------------------------------------------------------------------

% ***************************** question 1 **************************


\begin{question}{10}
  \begin{qparts}
    \qpart[3]{Let $X$ be outcome of an honest dice
      ($\mathcal{X}=\{1,2,3,4,5,6\}$).  What is
      \begin{enumerate}
      \item $\av{X}$
      \item $\av{2\,X}$
      \item $\av{X^2}$
      \end{enumerate}}{\inumberedlines[2]{3}}
    \begin{answer}
      \begin{enumerate}
      \item $\av{X} = \frac{1+2+3+4+5+6}{6} = 3.5$
      \item $\av{2\,X} = 2\,\av{X} = 7$
      \item $\av{X^2} = \frac{1+4+9+16+25+36}{6} = \frac{91}{6} = 15.1666$
      \end{enumerate}
    \end{answer}
    \freshpage
    \qpart[4]{Let $X$ be a random variable as before and $Y$ be a
      random variable for a second independent dice. What is
      \begin{enumerate}
      \item $\av[X]{X + Y}$
      \item $\av[X,Y]{X + Y}$
      \item $\av[X]{X\,Y}$
      \item $\av[X,Y]{X\,Y}$
      \end{enumerate}
    }{\inumberedlines[1]{4}}
    \begin{answer}
      \begin{enumerate}
      \item $\av[X]{X + Y} = \av[X]{X} + Y = 3.5 + Y$
      \item $\av[X,Y]{X + Y} = \av[X]{X} + \av[Y]{Y} = 7$
      \item $\av[X]{X\,Y} = \av[X]{X}\,Y = 3.5\,Y$
      \item $\av[X,Y]{X\,Y} = \av[X]{X} \, \av[Y]{Y} = 3.5^2 = 12.25$
      \end{enumerate}
    \end{answer}
    \qpart[3]{Let $X$ be the random variable as before and $E$ be a
      random variable equal to 0 if $X$ is odd and 1 if $X$ is even.
      Note that $E$ is not independent of $X$.  What is
      \begin{enumerate}
      \item $\Prob{E=1} (= \av[X]{E})$
      \item $\av[X]{X + E}$
      \item $\av[X]{X\,E}$
      \end{enumerate}
    }{\inumberedlines[1]{3}}
    \begin{answer}
      \begin{enumerate}
      \item $\Prob{E=1} = \av[X]{E} = \frac{0+1+0+1+0+1}{6} = 0.5$
      \item $\av[X]{X + E} = \av[X]{X} + \av[X]{E} = 3.5 + 0.5 = 4$
      \item $\av[X]{X\,E} = \frac{0 +2 + 0 +4 +0 +6}{6} = 2$
      \end{enumerate}
    \end{answer}
    \end{qparts}
\end{question}
\begin{answer}
  Note that expectations are \textit{linear operators} so that
  \begin{align*}
    \av[x]{a\,g(X) + b\,h(X)} = a\,\av[X]{g(X)} + b\,\av[x]{h(X)}.
  \end{align*}
  Note that this also means
  \begin{align*}
    \av[X]{\sum_{i=1}^n g_i(X)} = \sum_{i=1}^n \av[X]{g_i(X)}.
  \end{align*}
  If $X$ and $Y$ are independent (so $\Prob{X=x, Y=y}=
  \Prob{X=x}\,\Prob{Y=y}$) then
  \begin{align*}
    \av[X,Y]{g(X)\, h(Y)} = \av[X]{g(X)} \, \av[Y]{h[Y]}.
  \end{align*}
  However is $X$ and $Y$ are not independent random variables then
   \begin{align*}
    \av[X,Y]{g(X)\, h(Y)} = \sum_{x\in\mathcal{X}}
     \sum_{y\in\mathcal{Y}} \Prob{X=x,Y=y} g(X)\, h(Y)
   \end{align*}
   and usually $\av[X,Y]{g(X)\, h(Y)} \neq \av[X]{g(X)} \, \av[Y]{h[Y]}$.
\end{answer}

\end{document}




