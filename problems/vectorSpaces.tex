%% ELEC2013 2007

%% ----------------------------------------------------------------
%% exam.tex
%% ----------------------------------------------------------------
\documentclass{sotonExamBoxes}    % use option [answers] to include them
%\documentclass[answers]{uosexamb}
%% ----------------------------------------------------------------
\usepackage{graphicx}
%\usepackage{pstricks}
\usepackage{alltt}
\usepackage{bm}
\usepackage{amssymb}
\usepackage{amsfonts}
\usepackage{amsmath}
\thinmuskip=1mu
%\renewcommand{\ttdefault}{pcr}
\renewcommand{\ttdefault}{pcr}
\newcommand{\tr}{\textsf{T}}
\newcommand{\e}[1]{{\rm e}^{#1}}
\newcommand{\E}[1]{{\rm e}^{{\displaystyle #1}}}
\newcommand{\logg}[1]{\log\left(#1\right)}
\newcommand{\Prob}[1]{\mathbb{P}\left(#1\right)}
\newcommand{\Step}{\mathop{\mathrm{O}\hspace{-15pt}\rule[8.5pt]{11pt}{1.2pt}\hspace{2pt}}}
\newcommand\diag{\mathop{\mathrm{diag}}}
\newcommand{\grad}{\bm{\nabla}}
\newcommand{\dd}{\mathrm{d}}
\newcommand{\class}{\mathcal{C}}
\newcommand{\data}{\mathcal{D}}
\DeclareMathAlphabet{\mat}{OT1}{cmss}{bx}{n}
\newcommand{\pd}[2]{\ensuremath{\frac{\qpartial #1}{\qpartial #2}}\xspace}
\newcommand{\M}[1]{\ensuremath{\boldsymbol{#1}}\xspace}
\newcommand{\bst}{\rule{0mm}{5mm}}
\newcommand{\av}[2][\,]{\mathbb{E}_{#1}\!\left[ {\strut #2} \right]}
\newcommand{\len}[1]{\| #1 \|}
\graphicspath{{figures/} {../2018-19/figures}}
\newcommand{\inner}[2]{\left\langle #1, #2\right\rangle}
\graphicspath{{figures/}}
\newcommand{\squeeze}{\setlength{\itemsep}{-2pt}}

% \usepackage{picture}
\begin{document}
\unitTitle{Vector Space Problem Sheet}
\authors{Adam Pr\"ugel-Bennett}
\unitcode{COMP6208}
\semester{2}
\year{2022}
\duration{45}{0}


\maketitle

This paper asks you to prove some well known results.  Although the
algebra is easy the proofs are not entirely straightforward.  There
are marks assigned to the readability of the solution and also how
well laid out and explained the steps you make are. (A good proof
needs to be easy to follow: you need not comment on trivial algebra,
but there should not be steps that are difficult to follow).

This looks very mathematical, but it helps to develop the tools and
language that is used to describe machine learning.

%\blankpage


%% ----------------------------------------------------------------
%\section
% -----------------------------------------------------------------------
% *************************  Section A  *********************************
% -----------------------------------------------------------------------

% ***************************** question 1 **************************


\begin{question}{20}
  An inner product $\inner{\bm{x}}{\bm{y}}$ between vectors in a
  vector space $\mathcal{V}$ satisfies the following
  properties
  \begin{enumerate}\squeeze
  \item $\inner{\bm{x}}{\bm{x}} \geq 0$ for all $\bm{x}\in\mathcal{V}$
  \item $\inner{\bm{x}}{\bm{x}} = 0$  if and only if $\bm{x}=\bm{0}$
  \item $\inner{\alpha\, \bm{x}}{\bm{y}} = \alpha\,
    \inner{\bm{x}}{\bm{y}}$
  \item $\inner{\bm{x}}{\bm{y}+\bm{z}} = \inner{\bm{x}}{\bm{y}} +
    \inner{\bm{x}}{\bm{z}}$
  \item $\inner{\bm{x}}{\bm{y}} = \inner{\bm{y}}{\bm{x}}$
  \end{enumerate}
  The question explores properties of inner products.
  \clearpage
  \begin{qparts}
    \qpart[10]{Consider the quadratic
      \begin{align*}
        q(t ) = \inner{\bm{x} + t\, \bm{y}}{\bm{x} + t\, \bm{y}}
      \end{align*}
      By definition 1 of an inner product $q(t)$ must be non-negative
      and will only be 0 when $\bm{x} + t\, \bm{y}=0$.

      Expand $q(t)$ in the form $q(t) = A\,t^2 + 2\, B \, t +C$ to       
      find $A$, $B$ and $C$.  For $q(t)$ to not change sign its roots
      (values of $t$ such that $q(t)=0$) must be complex (i.e. have an
      imaginary part), or possible have a double root.  If there is a
      value of $t$ such that $q(t)=0$.  Use the
      standard solutions to a quadratic of the form $q(t) = A\,t^2 +
      2\, B t + C$ to show that for our $q(t)$ to never become
      negative then
      \begin{align*}
        \inner{\bm{x}}{\bm{y}}^2 \leq \inner{\bm{x}}{\bm{x}} \,
        \inner{\bm{y}}{\bm{y}}.
      \end{align*}
      This is the famous Cauchy-Schwarz inequality written in a very
      general form.}{\lines{17}}
    \begin{answer}
      \begin{align*}
        q(t) = \inner{\bm{y}}{\bm{y}}  t^2 + 2\,
        \inner{\bm{x}}{\bm{y}}\,t +  \inner{\bm{x}}{\bm{x}}
      \end{align*}
      So that $A= \inner{\bm{y}}{\bm{y}}$, $B= 2\,\inner{\bm{x}}{\bm{y}}$
      and $C=\inner{\bm{x}}{\bm{x}}$.  The roots to the quadratic
      equation $q(t)=A\,t^2 + B \, t + C = 0$ are
      \begin{align*}
        t = \frac{-B \pm \sqrt{B^2 - 4\,A\,C}}{2\,A}.
      \end{align*}
      The condition that these roots are complex is that
      $B^2-4\,A\,C<0$, and where $B^2-4\,A\,C=0$ indicating there is a
      double root.  Thus $B^2/4 \leq A\,C$.  Putting in the values of
      $A$, $B$ and $C$ then
      \begin{align*}
          \inner{\bm{x}}{\bm{y}}^2 \leq \inner{\bm{x}}{\bm{x}} \,
        \inner{\bm{y}}{\bm{y}}.
      \end{align*}
    \end{answer}
    \freshpage
    \qpart[10]{From an inner product we can define a norm $\len{\bm{x}} =
      \sqrt{\inner{\bm{x}}{\bm{x}}}$.  This clearly satisfies
      non-negativity and linearity.  The only non-trivial property to
      show is that this norm satisfies the triangular inequality.  Expand out
      $\len{\bm{x} + \bm{y}}$ and hence show that
      \begin{align*}
        \len{\bm{x}+\bm{y}}^2 = \len{\bm{x}}^2 + \len{\bm{y}}^2 + 2
        \inner{\bm{x}}{\bm{y}}
      \end{align*}
      Then use the Cauchy-Schwarz inequality to prove the triangular
      inequality.}{\lines{17}}
    \begin{answer}
      We start form
      \begin{align*}
        \len{\bm{x}+\bm{y}}^2
        &=  \inner{\bm{x} + \bm{y}}{\bm{x} + \bm{y}}
        \\
        &=  \inner{\bm{x}}{\bm{x}} + \inner{\bm{y}}{\bm{y}}
          + 2 \, \inner{\bm{x}}{\bm{y}}
      \end{align*}
      Using the Cauchy-Schwarz inequality $|\inner{\bm{x}}{\bm{y}}|^2
      \leq \inner{\bm{x}}{\bm{x}}\, \inner{\bm{y}}{\bm{y}}$ so that
      \begin{align*}
         \len{\bm{x}+\bm{y}}^2
        &\leq  \inner{\bm{x}}{\bm{x}} + \inner{\bm{y}}{\bm{y}}
          + 2 \, \sqrt{\inner{\bm{x}}{\bm{x}}\, \inner{\bm{y}}{\bm{y}}}
        \\
        &= \len{\bm{x}}^2 + \len{\bm{y}}^2 + 2 \,
          \len{\bm{x}}\,\len{\bm{y}}
        = \left( \len{\bm{x}} + \len{\bm{y}}\right)^2
      \end{align*}
      Taking the square root (which is monotonic so does not change
      the inequality)
      \begin{align*}
         \len{\bm{x}+\bm{y}} \leq  \len{\bm{x}} + \len{\bm{y}}
      \end{align*}
      Thus proving the triangular inequality for
      $\len{\bm{x}} = \sqrt{\inner{\bm{x}}{\bm{x}}}$. 
    \end{answer}
  \end{qparts}
\end{question}
\freshpage

% ***************************** question 2 **************************

\begin{question}{20}
  Random variables, $X$, $Y$, etc.{} form a vector space (i.e. they
  satisfy properties such as closure under addition and scalar
  multiplication).   Furthermore we can define an inner product
  between random variables as
  \begin{align*}
    \inner{X}{Y} = \av{X\,Y}
  \end{align*}
  (i.e. the expectation of the random variable $Z = X\,Y$)
  \begin{qparts}
    \qpart[10]{Use the Cauchy-Schwarz inequality to show that
      \begin{align*}
        \mathrm{Cov}(X,Y)^2 \leq \mathrm{Var}(X) \, \mathrm{Var}(Y)
      \end{align*}
      hence show that the Pearson correlation is between -1 and 1.
    }{\lines{17}}
    \begin{answer}
      Let $\mu = \av{X}$ and $\nu = \av{Y}$ then
      \begin{align*}
        \mathrm{Cov}(X,Y) = \av{(X-\mu)(Y-\nu)}
      \end{align*}
      From the Cauchy-Schwarz inequality
      \begin{align*}
        \mathrm{Cov}(X,Y)^2
        &= \av{(X-\mu)\,(Y-\nu)}^2 = \inner{X-\mu}{Y-\mu}^2
        \\
        &\leq \inner{X-\mu}{X-\mu} \, \inner{Y-\mu}{Y-\mu} =
          \av{(X-\mu)^2} \, \av{(Y-\nu)^2} = \mathrm{Var}(X)\,
          \mathrm{Var}(Y).
      \end{align*}
      which proves the inequality.  The Pearson correlation is defined
      as
      \begin{align*}
        \rho(X,Y) =  \frac{\mathrm{Cov}(X,Y)}
        {\sqrt{\mathrm{Var}(X) \, \mathrm{Var}(Y)}}
      \end{align*}
      Thus $\rho(X,Y)^2= \mathrm{Cov}(X,Y) ^2/\left( \mathrm{Var}(X) \,
        \mathrm{Var}(Y)\right)$, but  we have already shown that
      $\mathrm{Cov}(X,Y)^2 \leq \mathrm{Var}(X) \, \mathrm{Var}(Y)$ so
      $\rho(X,Y)^2\leq 1$ and $\rho(X,Y)$ will lie in the interval $[-1,1]$.
    \end{answer}
    \freshpage
    \qpart[10]{Show that for vectors $\bm{x}, \bm{y}\in\mathbb{R}^n$ that the
      inner product
      \begin{align*}
        \inner{\bm{x}}{\bm{y}} = \sum_{i=1}^n w_i \, x_i \,y_i
      \end{align*}
      satisfies the condition of an inner product provided $w_i>0$ for
      all $i$.  Write down the norm and distance induced by this
      inner product and provide an interpretation of what this
      distance means.}{\lines{19}}
    \begin{answer}
      We consider the five condition on the inner product
      \begin{enumerate}
      \item Non-negativity follows from
        \begin{align*}
          \inner{\bm{x}}{\bm{x}} = \sum_{i=1}^n w_i \, x_i ^2
          \geq 0
        \end{align*}
        since for each term in the sum $w_i \, x_i ^2\geq0$.
      \item For $ \inner{\bm{x}}{\bm{x}} = 0$ each term in the sum
        must be zero.  Since $w_i>0$ this implies $x_i=0$ for all $i$,
        that is, $\bm{x}=\bm{0}$.
      \item Linearity follows trivially since
        \begin{align*}
          \inner{a\,\bm{x}}{\bm{y}}
          &= \sum_{i=1}^n w_i \, (a\,x_i)  \,y_i \\
          &= a\, \sum_{i=1}^n w_i \, x_i \,y_i = a\,\inner{\bm{x}}{\bm{y}}
        \end{align*}
      \item The distributive law again follows trivially
        \begin{align*}
          \inner{\bm{x}}{\bm{y}+\bm{z}}
          &= \sum_{i=1}^n w_i \, x_i \,(y_i+z_i) \\
          &= \sum_{i=1}^n w_i \, x_i \,y_i + \sum_{i=1}^n w_i \, x_i \,z_i 
          \\
          &=   \inner{\bm{x}}{\bm{y}} +   \inner{\bm{x}}{\bm{z}}
        \end{align*}
      \item Finally symmetry follows from the commutativity of
        multiplication
        \begin{align*}
          \inner{\bm{x}}{\bm{y}} &=  \sum_{i=1}^n w_i \, x_i \,y_i
          =  \sum_{i=1}^n w_i \, y_i \,x_i =  \inner{\bm{y}}{\bm{x}} 
        \end{align*}
        The induced norm for this inner product is
        \begin{align*}
          \len{x} = \sqrt{\sum_{i=1}^n w_i \, x_i^2}
        \end{align*}
        while the induced distance is
        \begin{align*}
          d(\bm{x}, \bm{y}) = \sqrt{ \sum_{i=1}^n w_i \, (x_i-y_i)^2}
        \end{align*}
        A natural interpretation of this is that this is a
        generalisation of the Euclidean distance where we rescale each
        axis by a factor $w_i$ and then compute the normal Euclidean distance.
      \end{enumerate}
    \end{answer}
  \end{qparts}
  
\end{question}

% ***************************** question 3 **************************



% ***************************** End of exam *****************************
\end{document}




